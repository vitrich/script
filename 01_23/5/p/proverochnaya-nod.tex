\documentclass[a4paper,12pt]{article}
\usepackage[T2A]{fontenc}
\usepackage[russian]{babel}
\usepackage{geometry}
\usepackage{amsmath}
\usepackage{amssymb}
\usepackage{xcolor}
\usepackage{fancybox}

% Макет для А5 (два варианта на одном листе А4)
\geometry{a4paper, margin=0.5cm, left=0.5cm, right=0.5cm, top=0.5cm, bottom=0.5cm}

\pagestyle{empty}

\title{Проверочная работа: Текстовые задачи на НОД и НОК}
\author{}
\date{}

\begin{document}

% ВАРИАНТ 1 (левая половина)
\begin{minipage}[t]{0.48\textwidth}

\textbf{\Large Проверочная работа}

\textbf{\Large Вариант 1}

\vspace{0.2cm}

\noindent\hrulefill

\textbf{Время: 20 минут. Все ответы на обороте.}

\vspace{0.3cm}

\noindent\textbf{1.} В~сувенирном магазине 36~ракушек одного вида, 48~другого и~72~третьего. Какое наибольшее число одинаковых наборов?

\vspace{1cm}

\noindent\textbf{2.} В~конкурсе положили в~волшебные сундучки 18~жемчужин и~24~кристалла поровну. Сколько сундучков?

\vspace{1cm}

\noindent\textbf{3.} Марш солдат: 12~человек в~шеренге и~18~человек. Какое минимальное число солдат?

\vspace{1cm}

\noindent\textbf{4.} Лента разрезается на куски 35~см и~50~см. Какой наименьший размер? (в~см)

\vspace{1cm}

\noindent\textbf{5.} На прямоугольный участок 54~м на 48~м ставят столбы забора через равные расстояния. Максимальное расстояние между столбами?

\vspace{1cm}

\end{minipage}
\hfill
% ВАРИАНТ 2 (правая половина)
\begin{minipage}[t]{0.48\textwidth}

\textbf{\Large Проверочная работа}

\textbf{\Large Вариант 2}

\vspace{0.2cm}

\noindent\hrulefill

\textbf{Время: 20 минут. Все ответы на обороте.}

\vspace{0.3cm}

\noindent\textbf{1.} В~конкурсе положили 18~жемчужин и~24~кристалла в~сундучки поровну. Сколько сундучков заполнено?

\vspace{1cm}

\noindent\textbf{2.} Цветы: 156~ромашек, 234~василька, 390~травинок. Больше 50~одинаковых букетов. Сколько букетов?

\vspace{1cm}

\noindent\textbf{3.} Фейерверк: жёлтые каждые 2~сек, красные каждые 3~сек, белые каждые 4~сек. Через сколько секунд вспыхнут вместе?

\vspace{1cm}

\noindent\textbf{4.} Орхидеи: 1-й сорт через 15~см, 2-й через 18~см. Через какое расстояние окажутся рядом?

\vspace{1cm}

\noindent\textbf{5.} Желудей 36, орехов 48, веточек 72. Сколько разных поделок из одинакового материала?

\vspace{1cm}

\end{minipage}

\newpage

% ВАРИАНТ 3 (левая половина)
\begin{minipage}[t]{0.48\textwidth}

\textbf{\Large Проверочная работа}

\textbf{\Large Вариант 3}

\vspace{0.2cm}

\noindent\hrulefill

\textbf{Время: 20 минут. Все ответы на обороте.}

\vspace{0.3cm}

\noindent\textbf{1.} Бусинок 185~лиловых и~111~бирюзовых. Сколько браслетов поровну?

\vspace{1cm}

\noindent\textbf{2.} Шаги: утка 60~мм, гусь 75~мм. На каком расстоянии сделают целое число шагов?

\vspace{1cm}

\noindent\textbf{3.} Три друга ходят: 1-й раз в~5~дней, 2-й раз в~12~дней, 3-й раз в~10~дней. Когда встретятся?

\vspace{1cm}

\noindent\textbf{4.} На участок 54~м на 48~м ставят столбы через равные расстояния. Максимальное расстояние между столбами?

\vspace{1cm}

\noindent\textbf{5.} Соки: 210~л виноград, 126~л апельсин, 294~л ананас. Одинаковые упаковки коктейля. Сколько упаковок?

\vspace{1cm}

\end{minipage}
\hfill
% ВАРИАНТ 4 (правая половина)
\begin{minipage}[t]{0.48\textwidth}

\textbf{\Large Проверочная работа}

\textbf{\Large Вариант 4}

\vspace{0.2cm}

\noindent\hrulefill

\textbf{Время: 20 минут. Все ответы на обороте.}

\vspace{0.3cm}

\noindent\textbf{1.} Ткань 48~см на 40~см режут на квадратные лоскутки. Какой максимальный размер лоскута?

\vspace{1cm}

\noindent\textbf{2.} Театры: 145~мальчиков, 87~девочек. В~группе поровну м. и~д. Сколько групп?

\vspace{1cm}

\noindent\textbf{3.} Беспилотники: 1-й 8~мин, 2-й 12~мин, 3-й 18~мин. Когда вернутся одновременно?

\vspace{1cm}

\noindent\textbf{4.} Карандашей 48~синих, 48~жёлтых, 48~зелёных, 72~красных, 120~картинок. Наборы поровну. Сколько?

\vspace{1cm}

\noindent\textbf{5.} Конфет 185, игрушек 111. Одинаковые подарки. Сколько подарков?

\vspace{1cm}

\end{minipage}

\end{document}