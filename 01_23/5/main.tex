
\documentclass[t, aspectratio=169]{beamer}  % [t], [c], или [b] --- вертикальное выравнивание на слайдах (верх, центр, низ)
%\documentclass[handout]{beamer} % Раздаточный материал (на слайдах всё сразу)
%\documentclass[aspectratio=169]{beamer} % Соотношение сторон

%\usetheme{Berkeley} % Тема оформления
%\usetheme{Bergen}
%\usetheme{Szeged}
%\usetheme{Antibes}
\usetheme{default}
\usecolortheme{default}
\useinnertheme{default}
\useoutertheme{default}

% Базовая настройка без лишних элементов
\setbeamertemplate{navigation symbols}{} % Убираем навигационные символы
\setbeamertemplate{footline}{} % Убираем нижнюю панель
\setbeamertemplate{headline}{} % Убираем верхнюю панель

\useinnertheme{circles}

%\useinnertheme{rectangles}
 

%%% Работа с русским языком
\usepackage{cmap}					% поиск в PDF
\usepackage{mathtext} 				% русские буквы в формулах
\usepackage[T2A]{fontenc}			% кодировка
\usepackage[utf8]{inputenc}			% кодировка исходного текста
\usepackage[english,russian]{babel}	% локализация и переносы

%% Beamer по-русски
\newtheorem{rtheorem}{Теорема}
\newtheorem{rproof}{Доказательство}
\newtheorem{rexample}{Пример}

%%% Дополнительная работа с математикой
\usepackage{amsmath,amsfonts,amssymb,amsthm,mathtools} % AMS
\usepackage{icomma} % "Умная" запятая: $0,2$ --- число, $0, 2$ --- перечисление

%% Номера формул
\mathtoolsset{showonlyrefs=true} % Показывать номера только у тех формул, на которые есть \eqref{} в тексте.
%\usepackage{leqno} % Нумерация формул слева

%% Свои команды
\DeclareMathOperator{\sgn}{\mathop{sgn}}

%% Перенос знаков в формулах (по Львовскому)
\newcommand*{\hm}[1]{#1\nobreak\discretionary{}
{\hbox{$\mathsurround=0pt #1$}}{}}

%%% Работа с картинками
\usepackage{graphicx}  % Для вставки рисунков
\graphicspath{{images/}{images2/}}  % папки с картинками
\setlength\fboxsep{3pt} % Отступ рамки \fbox{} от рисунка
\setlength\fboxrule{1pt} % Толщина линий рамки \fbox{}
\usepackage{wrapfig} % Обтекание рисунков текстом

%%% Работа с таблицами
\usepackage{array,tabularx,tabulary,booktabs} % Дополнительная работа с таблицами
\usepackage{longtable}  % Длинные таблицы
\usepackage{multirow} % Слияние строк в таблице

%%% Программирование
\usepackage{etoolbox} % логические операторы

%%% Другие пакеты
\usepackage{lastpage} % Узнать, сколько всего страниц в документе.
\usepackage{soul} % Модификаторы начертания
\usepackage{csquotes} % Еще инструменты для ссылок
%\usepackage[style=authoryear,maxcitenames=2,backend=biber,sorting=nty]{biblatex}
\usepackage{multicol} % Несколько колонок

%%% Картинки
\usepackage{tikz, times} % Работа с графикой
\usepackage{pgfplots}
\usepackage{pgfplotstable}
\usetikzlibrary{mindmap,trees}
\usepackage{verbatim}
\usepackage{wrapfig}
\usetikzlibrary{shadows}


\title{Признаки делимости натуральных чисел}
\subtitle{5 класс}
%\author{Владимир Витриченко}
\date{\today}
\institute[]{Школа Летово Джуниор }



\usepackage{amsmath} % подключить в преамбуле

\setbeamercovered{dynamic} 
 

\begin{document}

\frame[plain]{\titlepage}	% Титульный слайд
\large

\begin{frame}
	
	\frametitle{Оглавление}
	\tableofcontents[pausesections]
	
	% pausesubsections, sections={<2-3>}]	
\end{frame}

\section{Признаки делимости на 10, 2 и на 5}	
\subsection{}

\begin{frame}
	\frametitle{\insertsection} 
%	\framesubtitle{\insertsubsection}	 
		{\Huge
		% Используем align* для выравнивания
		\begin{align*}
			% Переносим \pause перед \\ для корректного порядка
			a &= b \cdot c + q \pause \\
			a &= 10 \cdot c + q \pause \\
			a &= 100 \cdot c + q \pause \\
			a &= 1000 \cdot c + q 
		\end{align*}
	}	
\end{frame}


\begin{frame}
	
	
	\frametitle{\insertsection} 
	\framesubtitle{\insertsubsection}
	\begin{block}{Определение}
		Число делится на 10 в том и только в том случае, когда его последняя цифра равна 0.
	\end{block}\pause
	\begin{block}{или}
		Число делится на 10  \alert{$\Leftrightarrow $} когда его последняя цифра равна 0.
	\end{block}\pause
	\begin{block}{Определение}
		Число делится на 5 в том и только в том случае, когда  оно оканчивается цифрой 0 или 5.
	\end{block}\pause
	\begin{block}{или}
		Число делится на 5 \alert{$\Leftrightarrow $}  оканчивается цифрой 0 или 5.
	\end{block}\pause

	
	
	
\end{frame}
\begin{frame}
	
	\frametitle{\insertsection} 
	\framesubtitle{\insertsubsection}
	
	\begin{block}{Определение}
		Число делится на 2 в том и только в том случае, если последняя цифра числа четная.
	\end{block}\pause
	\begin{alertblock}{или}
		Число делится на 2  \alert{$\Leftrightarrow $}  последняя цифра числа четная.
	\end{alertblock}\pause
\end{frame}






\end{document}