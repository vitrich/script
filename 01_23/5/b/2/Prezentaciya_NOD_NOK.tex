\documentclass[t, aspectratio=169]{beamer}

\usetheme{default}
\usecolortheme{default}
\useinnertheme{default}
\useoutertheme{default}

% Базовая настройка без лишних элементов
\setbeamertemplate{navigation symbols}{}
\setbeamertemplate{footline}{}
\setbeamertemplate{headline}{}

\useinnertheme{circles}

%%% Работа с русским языком
\usepackage{cmap}
\usepackage{mathtext}
\usepackage[T2A]{fontenc}
\usepackage[english,russian]{babel}

%% Beamer по-русски
\newtheorem{rtheorem}{Теорема}
\newtheorem{rproof}{Доказательство}
\newtheorem{rexample}{Пример}

%%% Дополнительная работа с математикой
\usepackage{amsmath,amsfonts,amssymb,amsthm,mathtools}
\usepackage{icomma}

\mathtoolsset{showonlyrefs=true}

%% Свои команды
\DeclareMathOperator{\sgn}{\mathop{sgn}}

%% Перенос знаков в формулах
\newcommand*{\hm}[1]{#1\nobreak\discretionary{}
{\hbox{$\mathsurround=0pt #1$}}{}}

%%% Работа с картинками
\usepackage{graphicx}
\graphicspath{{images/}{images2/}}
\setlength\fboxsep{3pt}
\setlength\fboxrule{1pt}
\usepackage{wrapfig}

%%% Работа с таблицами
\usepackage{array,tabularx,tabulary,booktabs}
\usepackage{longtable}
\usepackage{multirow}

%%% Другие пакеты
\usepackage{lastpage}
\usepackage{soul}
\usepackage{csquotes}
\usepackage{multicol}

%%% Картинки
\usepackage{tikz, times}
\usepackage{pgfplots}
\pgfplotsset{compat=1.11}
\usepackage{pgfplotstable}
\usetikzlibrary{mindmap,trees}
\usepackage{verbatim}
\usetikzlibrary{shadows}

\title{Решение текстовых задач на НОД и НОК}

\subtitle{5--6 класс}

\date{\today}

\institute[]{Школа Летово}

\setbeamercovered{dynamic}

\begin{document}

\frame[plain]{\titlepage}

\large

\begin{frame}

\frametitle{Оглавление}

\tableofcontents[pausesections]

\end{frame}

\section{Основные понятия}

\begin{frame}

\frametitle{\insertsection}

\begin{block}{НОД --- Наибольший общий делитель}

Это наибольшее число, на которое делятся все данные числа \alert{нацело}.

\end{block}\pause

\textbf{Пример:} НОД(12; 18) = ?

\begin{align*}
\text{Делители 12:} & \quad 1, 2, 3, 4, 6, 12 \\
\text{Делители 18:} & \quad 1, 2, 3, 6, 9, 18 \\
\text{Общие:} & \quad 1, 2, 3, 6
\end{align*}

\alert{НОД(12; 18) = 6}

\end{frame}

\begin{frame}

\frametitle{\insertsection}

\begin{block}{НОК --- Наименьшее общее кратное}

Это наименьшее число, которое \alert{делится нацело} на все данные числа.

\end{block}\pause

\textbf{Пример:} НОК(4; 6) = ?

\begin{align*}
\text{Кратные 4:} & \quad 4, 8, 12, 16, 20, 24, \ldots \\
\text{Кратные 6:} & \quad 6, 12, 18, 24, 30, \ldots \\
\text{Общие:} & \quad 12, 24, \ldots
\end{align*}

\alert{НОК(4; 6) = 12}

\end{frame}

\section{Методы нахождения}

\begin{frame}

\frametitle{Способ 1: Разложение на простые множители}

\textbf{Алгоритм:}

\begin{enumerate}
	\item Разложить каждое число на простые множители \pause
	\item Для \alert{НОД}: выбрать общие множители \pause
	\item Для \alert{НОК}: выбрать все множители с максимальными степенями
\end{enumerate}

\end{frame}

\begin{frame}

\frametitle{Способ 1: Пример}

{\Large

\textbf{Найти НОД и НОК чисел 24 и 36}

\begin{align*}
24 &= 2^3 \cdot 3 \pause \\
36 &= 2^2 \cdot 3^2 \pause
\end{align*}

\alert{НОД(24; 36)} = $2^2 \cdot 3 = 4 \cdot 3 = 12$ \pause

\alert{НОК(24; 36)} = $2^3 \cdot 3^2 = 8 \cdot 9 = 72$

}

\end{frame}

\begin{frame}

\frametitle{Способ 2: Алгоритм Евклида}

\textbf{Алгоритм Евклида (для НОД):}

\begin{itemize}
	\item Большее число делим на меньшее \pause
	\item Затем делитель делим на остаток \pause
	\item Продолжаем, пока остаток не будет 0 \pause
	\item \alert{НОД} --- последний ненулевой остаток
\end{itemize}

\textbf{Пример:} НОД(48; 36)

{\Large
\begin{tabular}{c c c}
$48 : 36 = 1$ & остаток & $12$ \\
$36 : 12 = 3$ & остаток & $0$ \\
\end{tabular}
}

\alert{НОД = 12}

\end{frame}

\section{Определение типа задачи}

\begin{frame}

\frametitle{Когда применяем НОД?}

\alert{\textbf{Признаки задачи на НОД:}}

\begin{itemize}
	\item Нужно \alert{разделить} предметы на равные группы \pause
	\item Нужен \alert{наибольший размер} или \alert{максимальное количество} \pause
	\item Слова: ``одинаковые'', ``поровну'', ``разрезать'', ``разложить''
\end{itemize}

\begin{block}{Типичные ситуации:}
	\begin{itemize}
		\item Разрезание тканей, досок без отходов
		\item Раздача поровну (букеты, наборы, подарки)
		\item Составление одинаковых групп
		\item Размещение с равными расстояниями
	\end{itemize}
\end{block}

\end{frame}

\begin{frame}

\frametitle{Когда применяем НОК?}

\alert{\textbf{Признаки задачи на НОК:}}

\begin{itemize}
	\item События повторяются через разные промежутки \pause
	\item Нужно найти, \alert{когда они совпадут одновременно} \pause
	\item Слова: ``вместе'', ``одновременно'', ``встречаются'', ``повторится''
\end{itemize}

\begin{block}{Типичные ситуации:}
	\begin{itemize}
		\item Периодические события (фейерверки, вращение планет)
		\item Встречи людей с разной периодичностью
		\item Повторение движений (шестерни, спортсмены)
		\item Разрезание без остатков на куски разной длины
	\end{itemize}
\end{block}

\end{frame}

\section{Решение примеров}

\begin{frame}

\frametitle{Пример 1: Задача на НОД}

\textbf{\alert{Условие:}} На фабрике произвели 210 л виноградного сока, 126 л апельсинового и 294 л ананасового. Сколько упаковок коктейля?

\textbf{Решение:} Ищем НОД(210; 126; 294)

{\Large
\begin{align*}
210 &= 2 \cdot 3 \cdot 5 \cdot 7 \\
126 &= 2 \cdot 3^2 \cdot 7 \\
294 &= 2 \cdot 3 \cdot 7^2
\end{align*}

НОД = $2 \cdot 3 \cdot 7 = \alert{42}$ упаковки

}

\end{frame}

\begin{frame}

\frametitle{Пример 2: Задача на НОК}

\textbf{\alert{Условие:}} Салют: жёлтые через 2 сек, красные через 3 сек, белые через 4 сек. Через какое время вспыхнут одновременно?

\textbf{Решение:} Ищем НОК(2; 3; 4)

{\Large
\begin{align*}
2 &= 2 \\
3 &= 3 \\
4 &= 2^2
\end{align*}

НОК = $2^2 \cdot 3 = \alert{12}$ секунд

}

\textbf{Проверка:} 12 делится на 2, 3 и 4

\end{frame}

\begin{frame}

\frametitle{Пример 3: Ткань и лоскутки}

\textbf{\alert{Условие:}} Ткань 48 см на 40 см режут на квадратные лоскутки. Какой максимальный размер лоскута?

\textbf{Решение:} Ищем НОД(48; 40)

{\Large
\begin{align*}
48 &= 2^4 \cdot 3 \\
40 &= 2^3 \cdot 5 \\
\text{НОД} &= 2^3 = \alert{8} \text{ см}
\end{align*}

Количество лоскутков:

$\displaystyle \frac{48}{8} \cdot \frac{40}{8} = 6 \cdot 5 = \alert{30}$ лоскутков

}

\end{frame}

\section{Алгоритм решения}

\begin{frame}

\frametitle{\insertsection}

\textbf{\alert{Пошаговый алгоритм:}}

\begin{enumerate}
	\item \textbf{Прочитать} задачу внимательно \pause
	
	\item \textbf{Определить тип:}
	\begin{itemize}
		\item Разделение поровну? $\Rightarrow$ \alert{НОД}
		\item Периодические события? $\Rightarrow$ \alert{НОК}
	\end{itemize}
	\pause
	
	\item \textbf{Выписать числа} из условия \pause
	
	\item \textbf{Найти} НОД или НОК \pause
	
	\item \textbf{Проверить} ответ \pause
	
	\item \textbf{Записать ответ} предложением
\end{enumerate}

\end{frame}

\begin{frame}

\frametitle{Практические советы}

\begin{itemize}
	\item \alert{\textbf{Главное}} --- правильно определить тип задачи
	\vspace{0.3cm}
	
	\item Для быстрого разложения используйте малые простые числа: 2, 3, 5, 7, 11 \pause
	\vspace{0.3cm}
	
	\item Алгоритм Евклида удобен для двух больших чисел \pause
	\vspace{0.3cm}
	
	\item \textbf{Проверяйте всегда:}
	\begin{itemize}
		\item НОД должен делить все числа
		\item НОК должен делиться на все числа
	\end{itemize}
	\pause
	\vspace{0.3cm}
	
	\item \textbf{Полезная формула:} НОД(a;b) $\cdot$ НОК(a;b) = a $\cdot$ b
\end{itemize}

\end{frame}

\end{document}
