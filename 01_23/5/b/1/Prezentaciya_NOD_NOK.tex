\documentclass[14pt,aspectratio=169]{beamer}
\usepackage[utf-8]{inputenc}
\usepackage[russian]{babel}
\usepackage{amsmath}
\usepackage{amssymb}
\usepackage{geometry}
\usepackage{xcolor}
\usepackage{fancybox}

% Тема
\usetheme{Madrid}
\usecolortheme{default}
\setbeamerfont{frametitle}{size=\Large,series=\bfseries}

% Цвета
\definecolor{darkblue}{RGB}{25, 65, 124}
\definecolor{lightblue}{RGB}{173, 216, 230}
\definecolor{accent}{RGB}{220, 20, 60}

\setbeamercolor{structure}{fg=darkblue}
\setbeamercolor{alerted text}{fg=accent}

\title{Решение текстовых задач на НОД и НОК}
\author{Урок математики, 5--6 класс}
\date{}

\begin{document}

\frame{\titlepage}

% Слайд 1: Содержание
\begin{frame}
\frametitle{План урока}
\tableofcontents
\end{frame}

% Раздел 1: Основные понятия
\section{Основные понятия}

\begin{frame}
\frametitle{Что такое НОД?}
\begin{itemize}
    \item \textbf{НОД} -- наибольший общий делитель двух или нескольких чисел
    \item Это наибольшее число, на которое делятся все данные числа \textcolor{accent}{нацело}
    \item \textbf{Пример:} НОД(12; 18) = ?
    
    Делители 12: 1, 2, 3, 4, 6, 12
    
    Делители 18: 1, 2, 3, 6, 9, 18
    
    Общие делители: 1, 2, 3, 6
    
    \textcolor{accent}{НОД(12; 18) = 6}
\end{itemize}
\end{frame}

\begin{frame}
\frametitle{Что такое НОК?}
\begin{itemize}
    \item \textbf{НОК} -- наименьшее общее кратное двух или нескольких чисел
    \item Это наименьшее число, которое \textcolor{accent}{делится нацело} на все данные числа
    \item \textbf{Пример:} НОК(4; 6) = ?
    
    Кратные 4: 4, 8, 12, 16, 20, 24, ...
    
    Кратные 6: 6, 12, 18, 24, 30, ...
    
    Общие кратные: 12, 24, ...
    
    \textcolor{accent}{НОК(4; 6) = 12}
\end{itemize}
\end{frame}

% Раздел 2: Методы нахождения
\section{Методы нахождения НОД и НОК}

\begin{frame}
\frametitle{Способ 1: Разложение на простые множители}
\textbf{Алгоритм:}
\begin{enumerate}
    \item Разложить каждое число на простые множители
    \item Для \textcolor{accent}{НОД}: выбрать общие множители и перемножить их
    \item Для \textcolor{accent}{НОК}: выбрать все множители (с максимальными степенями)
\end{enumerate}

\vspace{0.5cm}

\textbf{Пример:} Найти НОД и НОК чисел 24 и 36

\begin{align*}
24 &= 2^3 \cdot 3 \\
36 &= 2^2 \cdot 3^2
\end{align*}

\textcolor{accent}{НОД(24; 36)} = $2^2 \cdot 3 = 4 \cdot 3 = 12$

\textcolor{accent}{НОК(24; 36)} = $2^3 \cdot 3^2 = 8 \cdot 9 = 72$
\end{frame}

\begin{frame}
\frametitle{Способ 2: Алгоритм Евклида (НОД)}
\textbf{Алгоритм Евклида:}
\begin{itemize}
    \item Большее число делим на меньшее
    \item Затем делитель делим на остаток
    \item Продолжаем, пока остаток не будет равен 0
    \item \textcolor{accent}{НОД} -- последний ненулевой остаток
\end{itemize}

\vspace{0.5cm}

\textbf{Пример:} НОД(48; 36)

\begin{tabular}{c c c c}
$48 : 36 = 1$ & остаток & $12$ & \\
$36 : 12 = 3$ & остаток & $0$ & \textbf{\textcolor{accent}{НОД = 12}} \\
\end{tabular}
\end{frame}

% Раздел 3: Применение в задачах
\section{Определение типа задачи}

\begin{frame}
\frametitle{Когда применяем НОД?}
\textcolor{accent}{\textbf{Признаки задачи на НОД:}}
\begin{itemize}
    \item Нужно \textcolor{accent}{\textbf{разделить}} предметы на равные группы
    \item Нужно найти \textcolor{accent}{\textbf{наибольший размер}} или \textcolor{accent}{\textbf{максимальное количество}} порций/наборов
    \item Слова: \textit{``одинаковые'', ``поровну'', ``разрезать'', ``разложить'', ``наибольший''}
\end{itemize}

\vspace{0.5cm}

\textbf{\textcolor{darkblue}{Типичные ситуации:}}
\begin{itemize}
    \item Разрезание тканей, досок без отходов
    \item Раздача поровну (букеты, наборы, подарки)
    \item Составление одинаковых групп
    \item Размещение с равными расстояниями
\end{itemize}
\end{frame}

\begin{frame}
\frametitle{Когда применяем НОК?}
\textcolor{accent}{\textbf{Признаки задачи на НОК:}}
\begin{itemize}
    \item События происходят через разные промежутки времени
    \item Нужно найти, \textcolor{accent}{\textbf{когда события совпадут}} одновременно
    \item Слова: \textit{``вместе'', ``одновременно'', ``встречаются'', ``через какое время'', ``повторится''}
\end{itemize}

\vspace{0.5cm}

\textbf{\textcolor{darkblue}{Типичные ситуации:}}
\begin{itemize}
    \item Периодические события (фейерверки, вращение планет)
    \item Встречи людей, которые ходят с разной периодичностью
    \item Повторение движений (шестерни, спортсмены на круге)
    \item Разрезание без остатков (лента на куски разной длины)
\end{itemize}
\end{frame}

% Раздел 4: Решение примеров
\section{Решение примеров}

\begin{frame}
\frametitle{Пример 1: Задача на НОД}
\textbf{\textcolor{darkblue}{Условие:}} На фабрике произвели 210 л виноградного сока, 126 л апельсинового и 294 л ананасового. Сколько упаковок фруктового коктейля произвели? Каков состав коктейля?

\vspace{0.3cm}

\textbf{\textcolor{accent}{Решение:}}
\begin{enumerate}
    \item Нужно разделить соки \textcolor{accent}{поровну} $\Rightarrow$ ищем \textbf{НОД}
    \item НОД(210; 126; 294) = ?
\end{enumerate}

\begin{align*}
210 &= 2 \cdot 3 \cdot 5 \cdot 7 \\
126 &= 2 \cdot 3^2 \cdot 7 \\
294 &= 2 \cdot 3 \cdot 7^2
\end{align*}

НОД = $2 \cdot 3 \cdot 7 = \textcolor{accent}{42}$ упаковки

Состав: $210:42=5$ л, $126:42=3$ л, $294:42=7$ л
\end{frame}

\begin{frame}
\frametitle{Пример 2: Задача на НОК}
\textbf{\textcolor{darkblue}{Условие:}} Салют: желтые хризантемы -- через 2 сек, красные сердечки -- через 3 сек, белые голуби -- через 4 сек. Через какое время вспыхнут одновременно?

\vspace{0.3cm}

\textbf{\textcolor{accent}{Решение:}}
\begin{enumerate}
    \item События повторяются через разные промежутки $\Rightarrow$ ищем \textbf{НОК}
    \item НОК(2; 3; 4) = ?
\end{enumerate}

\begin{align*}
2 &= 2 \\
3 &= 3 \\
4 &= 2^2
\end{align*}

НОК = $2^2 \cdot 3 = \textcolor{accent}{12}$ секунд

\textbf{\textcolor{darkblue}{Проверка:}} 12 : 2 = 6, 12 : 3 = 4, 12 : 4 = 3 ✓
\end{frame}

\begin{frame}
\frametitle{Пример 3: Нахождение с повторениями}
\textbf{\textcolor{darkblue}{Условие:}} Ткань 48 см × 40 см. Сколько одинаковых квадратных лоскутков? Какой размер?

\vspace{0.3cm}

\textbf{\textcolor{accent}{Решение:}}
\begin{enumerate}
    \item Нужен наибольший размер квадрата $\Rightarrow$ НОД(48; 40)
    \item $48 = 2^4 \cdot 3$, $40 = 2^3 \cdot 5$
    \item НОД = $2^3 = 8$ см -- сторона квадрата
\end{enumerate}

Количество лоскутков:
\begin{align*}
\text{По длине: } 48 : 8 &= 6 \text{ лоскутков} \\
\text{По ширине: } 40 : 8 &= 5 \text{ лоскутков} \\
\text{Всего: } 6 \times 5 &= \textcolor{accent}{30 \text{ лоскутков}}
\end{align*}
\end{frame}

% Раздел 5: Алгоритм решения
\section{Алгоритм решения}

\begin{frame}
\frametitle{Алгоритм решения задачи на НОД/НОК}

\begin{enumerate}
    \item \textbf{Прочитать задачу внимательно}
    \vspace{0.2cm}
    
    \item \textbf{Определить тип задачи:}
    \begin{itemize}
        \item Разделение поровну, одинаковые группы? $\Rightarrow$ \textcolor{accent}{НОД}
        \item Периодические события, совпадение? $\Rightarrow$ \textcolor{accent}{НОК}
    \end{itemize}
    \vspace{0.2cm}
    
    \item \textbf{Выписать числа} из условия задачи
    \vspace{0.2cm}
    
    \item \textbf{Найти} НОД или НОК (разложением или алгоритмом Евклида)
    \vspace{0.2cm}
    
    \item \textbf{Проверить} ответ с условием задачи
    \vspace{0.2cm}
    
    \item \textbf{Записать ответ} в виде предложения
\end{enumerate}
\end{frame}

\begin{frame}
\frametitle{Практические советы}

\begin{itemize}
    \item \textcolor{accent}{\textbf{Прежде всего}} -- понять, ищем ли мы НОД или НОК
    
    \vspace{0.3cm}
    
    \item Для \textbf{быстрого разложения} используйте малые простые множители: 2, 3, 5, 7, 11
    
    \vspace{0.3cm}
    
    \item \textbf{Алгоритм Евклида} удобен для двух чисел, когда они большие
    
    \vspace{0.3cm}
    
    \item Всегда \textbf{проверяйте}: НОД должен делить все числа, НОК должен делиться на все числа
    
    \vspace{0.3cm}
    
    \item \textbf{Полезная формула:} НОД(a;b) × НОК(a;b) = a × b
\end{itemize}
\end{frame}

\end{document}
