\documentclass[a4paper,12pt]{article}
\usepackage[T2A]{fontenc}
\usepackage[russian]{babel}
\usepackage{geometry}
\usepackage{amsmath}
\usepackage{amssymb}
\usepackage{xcolor}
\usepackage{fancybox}
\usepackage{multicol}

% Макет для А5 (два варианта на одном листе А4)
\geometry{a4paper, margin=0.5cm, left=0.5cm, right=0.5cm, top=0.5cm, bottom=0.5cm}

\pagestyle{empty}

% Стиль для задач
\newcommand{\task}[2]{%
\begin{shadowbox}
\textbf{Задача #1} (#2 минут)

\end{shadowbox}
}

\title{Самостоятельная работа: Текстовые задачи на НОД и НОК}
\author{}
\date{}

\begin{document}

% ВАРИАНТ 1 (левая половина + правая половина на одном листе)
% Левая половина (Вариант 1)
\begin{minipage}[t]{0.48\textwidth}

\textbf{\Large Вариант 1}

\vspace{0.2cm}

\noindent\hrulefill

\vspace{0.3cm}

\textbf{Время: 20 минут}

\vspace{0.3cm}

\noindent\textbf{1.} На кружке ребята делали куклы-мотанки из квадратных лоскутков ткани. Сколько куколок можно смотать из ткани 48 см на 40 см без отходов? Какой наибольший размер лоскута?

\vspace{1.5cm}

\noindent\textbf{2.} На фабрике произвели 210 л виноградного сока, 126 л апельсинового и 294 л ананасового. Сколько упаковок коктейля? Какой состав?

\vspace{1.5cm}

\noindent\textbf{3.} Фейерверки запущены: желтые через 2 сек, красные через 3 сек, белые через 4 сек. Через какое время вспыхнут одновременно?

\vspace{1.5cm}

\noindent\textbf{4.} На тур. слёт поехали 424 школьника в один лагерь и 477 в другой. Сколько автобусов? Сколько мест в каждом? (все места заняты)

\vspace{1.5cm}

\noindent\textbf{5.} Один спортсмен круг пробегает за 90 сек, второй за 106 сек. Через сколько времени они встретятся на финише?

\vspace{1.5cm}

\end{minipage}
\hfill
% Правая половина (Вариант 2)
\begin{minipage}[t]{0.48\textwidth}

\textbf{\Large Вариант 2}

\vspace{0.2cm}

\noindent\hrulefill

\vspace{0.3cm}

\textbf{Время: 20 минут}

\vspace{0.3cm}

\noindent\textbf{1.} Стол 195 см на 156 см украшают квадратными плитками. Какой наибольший размер плитки? Сколько плиток?

\vspace{1.5cm}

\noindent\textbf{2.} Во флористическую мастерскую привезли 156 ромашек, 234 василька и 390 травинок. Сколько букетов больше 50? Состав букета?

\vspace{1.5cm}

\noindent\textbf{3.} Три беспилотника: первый 8 мин, второй 12 мин, третий 18 мин. Через какое время все вернутся одновременно?

\vspace{1.5cm}

\noindent\textbf{4.} В наборе 185 бусин лилового цвета и 111 бусин бирюзового. Сколько браслетов? Сколько бусин каждого цвета в браслете?

\vspace{1.5cm}

\noindent\textbf{5.} На полке музея 100 экспонатов. Они расставлены по 3, по 4, по 5, по 6 штук. Сколько всего?

\vspace{1.5cm}

\end{minipage}

\newpage

% ВАРИАНТ 3 (левая половина + правая половина)
\begin{minipage}[t]{0.48\textwidth}

\textbf{\Large Вариант 3}

\vspace{0.2cm}

\noindent\hrulefill

\vspace{0.3cm}

\textbf{Время: 20 минут}

\vspace{0.3cm}

\noindent\textbf{1.} Для поделок использовали 36 желудей, 48 орехов и 72 веточки. Какое наибольшее число разных поделок из одинакового количества каждого материала?

\vspace{1.5cm}

\noindent\textbf{2.} На новогодний смотр театров приехало 145 мальчиков и 87 девочек. В каждой группе одинаковое число м. и д. Сколько групп? Состав группы?

\vspace{1.5cm}

\noindent\textbf{3.} Три друга в компьютерном клубе: один ходит 1 раз в 5 дней, второй раз в 12 дней, третий раз в 10 дней. Через какое время встретятся?

\vspace{1.5cm}

\noindent\textbf{4.} На участок 54 м на 48 м ставили столбы для забора через равные промежутки. Сколько столбов? На каком максимальном расстоянии?

\vspace{1.5cm}

\noindent\textbf{5.} Утка шагает 60 мм, гусь 75 мм. На каком наименьшем расстоянии они сделают целое число шагов?

\vspace{1.5cm}

\end{minipage}
\hfill
% ВАРИАНТ 4
\begin{minipage}[t]{0.48\textwidth}

\textbf{\Large Вариант 4}

\vspace{0.2cm}

\noindent\hrulefill

\vspace{0.3cm}

\textbf{Время: 20 минут}

\vspace{0.3cm}

\noindent\textbf{1.} Бусинок 185 лиловых и 111 бирюзовых. Сколько браслетов поровну? Сколько бусин каждого вида в браслете?

\vspace{1.5cm}

\noindent\textbf{2.} Ромашек 156, васильков 234, травинок 390. Больше 50 букетов одинакового состава. Сколько букетов и какой состав?

\vspace{1.5cm}

\noindent\textbf{3.} На морском побережье салют: каждые 2 сек жёлтые, каждые 3 сек красные, каждые 4 сек белые. Когда вспыхнут вместе?

\vspace{1.5cm}

\noindent\textbf{4.} Выпускники подарили первоклассникам: 69 карандашей и 46 ластиков поровну каждому. Сколько учеников в классе?

\vspace{1.5cm}

\noindent\textbf{5.} Лента разрезается на куски 35 см и 50 см без обрезков. Какой наименьший размер ленты? (в метрах)

\vspace{1.5cm}

\end{minipage}

\end{document}