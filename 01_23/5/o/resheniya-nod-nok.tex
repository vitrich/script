\documentclass[a4paper,12pt]{article}
\usepackage[T2A]{fontenc}
\usepackage[utf8]{inputenc}
\usepackage[russian]{babel}
\usepackage{geometry}
\usepackage{amsmath}
\usepackage{amssymb}
\usepackage{xcolor}
\usepackage{fancybox}
\usepackage{booktabs}
\usepackage{newunicodechar}
\newunicodechar{✓}{\ensuremath{\checkmark}}

\geometry{a4paper, margin=1cm}

\pagestyle{plain}

\title{\textbf{Решения и Ответы}\\[0.3cm] \textit{Текстовые задачи на НОД и НОК}}
\author{5--6 класс}
\date{}

\begin{document}
	
	\maketitle
	
	\section*{Самостоятельная работа -- Вариант 1}
	
	\subsection*{Задача 1}
	\textbf{Условие:} На кружке ребята делали куклы-мотанки из квадратных лоскутков ткани. Сколько куколок можно смотать из ткани 48~см на 40~см без отходов? Какой наибольший размер лоскута?
	
	\textbf{Решение:}
	Нужно разрезать ткань на квадраты без отходов. Размер квадрата должен быть делителем обеих размеров. Ищем наибольший такой делитель~-- НОД(48; 40).
	
	\begin{align*}
		48 &= 2^4 \cdot 3 = 16 \cdot 3 \\
		40 &= 2^3 \cdot 5 = 8 \cdot 5 \\
		\text{НОД}(48; 40) &= 2^3 = 8 \text{ см}
	\end{align*}
	
	Количество лоскутков:
	\begin{align*}
		\text{По длине: } 48 : 8 &= 6 \text{ лоскутков} \\
		\text{По ширине: } 40 : 8 &= 5 \text{ лоскутков} \\
		\text{Всего: } 6 \times 5 &= 30 \text{ куколок}
	\end{align*}
	
	\textbf{Ответ:} 30~куколок, сторона лоскута 8~см.
	
	\subsection*{Задача 2}
	\textbf{Условие:} На фабрике произвели 210~л виноградного сока, 126~л апельсинового и~294~л ананасового. Сколько упаковок коктейля? Какой состав?
	
	\textbf{Решение:}
	Нужно разделить соки поровну на одинаковые упаковки. Ищем НОД(210; 126; 294).
	
	\begin{align*}
		210 &= 2 \cdot 3 \cdot 5 \cdot 7 \\
		126 &= 2 \cdot 3^2 \cdot 7 \\
		294 &= 2 \cdot 3 \cdot 7^2 \\
		\text{НОД}(210; 126; 294) &= 2 \cdot 3 \cdot 7 = 42
	\end{align*}
	
	Состав упаковки:
	\begin{align*}
		\text{Виноград: } 210 : 42 &= 5 \text{ л} \\
		\text{Апельсин: } 126 : 42 &= 3 \text{ л} \\
		\text{Ананас: } 294 : 42 &= 7 \text{ л}
	\end{align*}
	
	\textbf{Ответ:} 42~упаковки, состав: 5~л, 3~л, 7~л соответственно.
	
	\subsection*{Задача 3}
	\textbf{Условие:} Фейерверки запущены: жёлтые через 2~сек, красные через 3~сек, белые через 4~сек. Через какое время вспыхнут одновременно?
	
	\textbf{Решение:}
	События повторяются периодически. Ищем НОК(2; 3; 4)~-- наименьшее время, когда все события совпадут.
	
	\begin{align*}
		2 &= 2 \\
		3 &= 3 \\
		4 &= 2^2 \\
		\text{НОК}(2; 3; 4) &= 2^2 \cdot 3 = 4 \cdot 3 = 12 \text{ сек}
	\end{align*}
	
	Проверка: $12 : 2 = 6$ ✓, $12 : 3 = 4$ ✓, $12 : 4 = 3$ ✓
	
	\textbf{Ответ:} Через 12~секунд.
	
	\subsection*{Задача 4}
	\textbf{Условие:} На тур. слёт поехали 424~школьника в~один лагерь и~477~в~другой. Сколько автобусов? Сколько мест в~каждом? (все места заняты, никто не стоит)
	
	\textbf{Решение:}
	Нужно разделить детей поровну на одинаковые автобусы. Ищем НОД(424; 477).
	
	Используем алгоритм Евклида:
	\begin{align*}
		477 : 424 &= 1 \text{ (остаток } 53\text{)} \\
		424 : 53 &= 8 \text{ (остаток } 0\text{)} \\
		\text{НОД}(424; 477) &= 53
	\end{align*}
	
	Количество автобусов:
	\begin{align*}
		\text{В 1-й лагерь: } 424 : 53 &= 8 \text{ автобусов} \\
		\text{Во 2-й лагерь: } 477 : 53 &= 9 \text{ автобусов} \\
		\text{Всего: } 8 + 9 &= 17 \text{ автобусов}
	\end{align*}
	
	\textbf{Ответ:} 17~автобусов, по 53~места в~каждом.
	
	\subsection*{Задача 5}
	\textbf{Условие:} Один спортсмен круг пробегает за 90~сек, второй за 106~сек. Через сколько времени они встретятся на финише?
	
	\textbf{Решение:}
	Ищем НОК(90; 106).
	
	\begin{align*}
		90 &= 2 \cdot 3^2 \cdot 5 \\
		106 &= 2 \cdot 53 \\
		\text{НОК}(90; 106) &= 2 \cdot 3^2 \cdot 5 \cdot 53 = 90 \cdot 53 = 4770 \text{ сек}
	\end{align*}
	
	Преобразуем в~минуты и~секунды:
	\begin{align*}
		4770 \text{ сек} &= 4770 : 60 = 79 \text{ мин } 30 \text{ сек} \\
		&= 1 \text{ ч } 19 \text{ мин } 30 \text{ сек}
	\end{align*}
	
	\textbf{Ответ:} Через 4770~секунд или 1~час 19~минут 30~секунд.
	
	\newpage
	
	\section*{Самостоятельная работа -- Вариант 2}
	
	\subsection*{Задача 1}
	\textbf{Условие:} Стол 195~см на 156~см украшают квадратными плитками. Какой наибольший размер плитки? Сколько плиток?
	
	\textbf{Решение:}
	Ищем НОД(195; 156).
	
	\begin{align*}
		195 &= 3 \cdot 5 \cdot 13 \\
		156 &= 2^2 \cdot 3 \cdot 13 \\
		\text{НОД}(195; 156) &= 3 \cdot 13 = 39 \text{ см}
	\end{align*}
	
	Количество плиток:
	\begin{align*}
		\text{По длине: } 195 : 39 &= 5 \text{ плиток} \\
		\text{По ширине: } 156 : 39 &= 4 \text{ плитки} \\
		\text{Всего: } 5 \times 4 &= 20 \text{ плиток}
	\end{align*}
	
	\textbf{Ответ:} Размер плитки 39~см × 39~см, всего 20~плиток.
	
	\subsection*{Задача 2}
	\textbf{Условие:} Во флористическую мастерскую привезли 156~ромашек, 234~василька и~390~травинок. Сколько букетов больше 50? Какой состав?
	
	\textbf{Решение:}
	Ищем НОД(156; 234; 390).
	
	\begin{align*}
		156 &= 2^2 \cdot 3 \cdot 13 \\
		234 &= 2 \cdot 3^2 \cdot 13 \\
		390 &= 2 \cdot 3 \cdot 5 \cdot 13 \\
		\text{НОД}(156; 234; 390) &= 2 \cdot 3 \cdot 13 = 78
	\end{align*}
	
	По условию требуется больше 50~букетов одинакового состава. Возьмём делитель 13 (меньше 78, даёт большее число букетов):
	
	\begin{align*}
		156 : 13 &= 12 \\
		234 : 13 &= 18 \\
		390 : 13 &= 30
	\end{align*}
	
	Получается 13~букетов одинакового состава по 12, 18 и~30 цветов соответственно.
	
	\textbf{Ответ:} 13~букетов, состав: 12~ромашек, 18~васильков, 30~травинок.
	
	\subsection*{Задача 3}
	\textbf{Условие:} Три беспилотника: первый 8~мин, второй 12~мин, третий 18~мин. Через какое время вернутся одновременно?
	
	\textbf{Решение:}
	Ищем НОК(8; 12; 18).
	
	\begin{align*}
		8 &= 2^3 \\
		12 &= 2^2 \cdot 3 \\
		18 &= 2 \cdot 3^2 \\
		\text{НОК}(8; 12; 18) &= 2^3 \cdot 3^2 = 8 \cdot 9 = 72 \text{ мин}
	\end{align*}
	
	Проверка: $72 : 8 = 9$ ✓, $72 : 12 = 6$ ✓, $72 : 18 = 4$ ✓
	
	\textbf{Ответ:} Через 72~минуты (1~час 12~минут).
	
	\subsection*{Задача 4}
	\textbf{Условие:} В~наборе 185~бусин лилового цвета и~111~бусин бирюзового. Сколько браслетов? Сколько бусин каждого цвета в~браслете?
	
	\textbf{Решение:}
	Ищем НОД(185; 111).
	
	Алгоритм Евклида:
	\begin{align*}
		185 : 111 &= 1 \text{ (остаток } 74\text{)} \\
		111 : 74 &= 1 \text{ (остаток } 37\text{)} \\
		74 : 37 &= 2 \text{ (остаток } 0\text{)} \\
		\text{НОД}(185; 111) &= 37
	\end{align*}
	
	Состав браслета:
	\begin{align*}
		\text{Лиловые: } 185 : 37 &= 5 \text{ бусин} \\
		\text{Бирюзовые: } 111 : 37 &= 3 \text{ бусины}
	\end{align*}
	
	\textbf{Ответ:} 37~браслетов, в~каждом 5~лиловых и~3~бирюзовых бусины.
	
	\subsection*{Задача 5}
	\textbf{Условие:} На полке музея 100~экспонатов. Они расставлены по~3, по~4, по~5, по~6~штук. Сколько всего?
	
	\textbf{Решение:}
	Экспонаты расставлены так, что делятся нацело на 3, 4, 5, 6. Нужно найти НОК(3; 4; 5; 6), но это число должно быть в~диапазоне около 100.
	
	\begin{align*}
		3 &= 3 \\
		4 &= 2^2 \\
		5 &= 5 \\
		6 &= 2 \cdot 3 \\
		\text{НОК}(3; 4; 5; 6) &= 2^2 \cdot 3 \cdot 5 = 4 \cdot 3 \cdot 5 = 60
	\end{align*}
	
	Так как 60 < 100 < 120, следующее кратное 60~это 120, но оно больше 100. Значит, ответ~60.
	
	\textbf{Ответ:} 60~экспонатов.
	
	\newpage
	
	\section*{Самостоятельная работа -- Вариант 3}
	
	\subsection*{Задача 1}
	\textbf{Условие:} Для поделок использовали 36~желудей, 48~орехов и~72~веточки. Какое наибольшее число разных поделок из одинакового количества каждого материала?
	
	\textbf{Решение:}
	Ищем НОД(36; 48; 72).
	
	\begin{align*}
		36 &= 2^2 \cdot 3^2 \\
		48 &= 2^4 \cdot 3 \\
		72 &= 2^3 \cdot 3^2 \\
		\text{НОД}(36; 48; 72) &= 2^2 \cdot 3 = 12
	\end{align*}
	
	\textbf{Ответ:} 12~поделок.
	
	\subsection*{Задача 2}
	\textbf{Условие:} На новогодний смотр театров приехало 145~мальчиков и~87~девочек. В~каждой группе одинаковое число м. и~д. Сколько групп? Состав группы?
	
	\textbf{Решение:}
	Ищем НОД(145; 87).
	
	Алгоритм Евклида:
	\begin{align*}
		145 : 87 &= 1 \text{ (остаток } 58\text{)} \\
		87 : 58 &= 1 \text{ (остаток } 29\text{)} \\
		58 : 29 &= 2 \text{ (остаток } 0\text{)} \\
		\text{НОД}(145; 87) &= 29
	\end{align*}
	
	Состав группы:
	\begin{align*}
		\text{Мальчиков: } 145 : 29 &= 5 \\
		\text{Девочек: } 87 : 29 &= 3
	\end{align*}
	
	\textbf{Ответ:} 29~групп, в~каждой 5~мальчиков и~3~девочки.
	
	\subsection*{Задача 3}
	\textbf{Условие:} Три друга в~компьютерном клубе: один ходит 1~раз в~5~дней, второй раз в~12~дней, третий раз в~10~дней. Через какое время встретятся?
	
	\textbf{Решение:}
	Ищем НОК(5; 12; 10).
	
	\begin{align*}
		5 &= 5 \\
		12 &= 2^2 \cdot 3 \\
		10 &= 2 \cdot 5 \\
		\text{НОК}(5; 12; 10) &= 2^2 \cdot 3 \cdot 5 = 60 \text{ дней}
	\end{align*}
	
	\textbf{Ответ:} Через 60~дней.
	
	\subsection*{Задача 4}
	\textbf{Условие:} На участок 54~м на 48~м ставят столбы для забора через равные расстояния. Сколько столбов? На каком максимальном расстоянии?
	
	\textbf{Решение:}
	Ищем НОД(54; 48)~-- это максимальное расстояние между столбами.
	
	\begin{align*}
		54 &= 2 \cdot 3^3 \\
		48 &= 2^4 \cdot 3 \\
		\text{НОД}(54; 48) &= 2 \cdot 3 = 6 \text{ м}
	\end{align*}
	
	Периметр участка: $2 \cdot (54 + 48) = 2 \cdot 102 = 204$~м
	
	Количество столбов: $204 : 6 = 34$~столба
	
	\textbf{Ответ:} 34~столба, расстояние 6~м.
	
	\subsection*{Задача 5}
	\textbf{Условие:} Утка шагает 60~мм, гусь 75~мм. На каком наименьшем расстоянии они сделают целое число шагов?
	
	\textbf{Решение:}
	Ищем НОК(60; 75).
	
	\begin{align*}
		60 &= 2^2 \cdot 3 \cdot 5 \\
		75 &= 3 \cdot 5^2 \\
		\text{НОК}(60; 75) &= 2^2 \cdot 3 \cdot 5^2 = 4 \cdot 3 \cdot 25 = 300 \text{ мм}
	\end{align*}
	
	Проверка: $300 : 60 = 5$~шагов утки ✓, $300 : 75 = 4$~шага гуся ✓
	
	\textbf{Ответ:} 300~мм = 30~см.
	
	\newpage
	
	\section*{Самостоятельная работа -- Вариант 4}
	
	\subsection*{Задача 1}
	\textbf{Условие:} Бусинок 185~лиловых и~111~бирюзовых. Сколько браслетов поровну? Сколько бусин каждого вида в~браслете?
	
	\textbf{Решение:} [см. Вариант~2, Задача~4]
	
	\textbf{Ответ:} 37~браслетов, в~каждом 5~лиловых и~3~бирюзовых бусины.
	
	\subsection*{Задача 2}
	\textbf{Условие:} Ромашек 156, васильков 234, травинок 390. Больше 50~букетов одинакового состава. Сколько букетов и~какой состав?
	
	\textbf{Решение:} [см. Вариант~2, Задача~2]
	
	\textbf{Ответ:} 13~букетов, состав: 12~ромашек, 18~васильков, 30~травинок.
	
	\subsection*{Задача 3}
	\textbf{Условие:} На морском побережье салют: каждые 2~сек жёлтые, каждые 3~сек красные, каждые 4~сек белые. Когда вспыхнут вместе?
	
	\textbf{Решение:} [см. Вариант~1, Задача~3]
	
	\textbf{Ответ:} Через 12~секунд.
	
	\subsection*{Задача 4}
	\textbf{Условие:} Выпускники подарили первоклассникам: 69~карандашей и~46~ластиков поровну каждому. Сколько учеников в~классе?
	
	\textbf{Решение:}
	Ищем НОД(69; 46).
	
	Алгоритм Евклида:
	\begin{align*}
		69 : 46 &= 1 \text{ (остаток } 23\text{)} \\
		46 : 23 &= 2 \text{ (остаток } 0\text{)} \\
		\text{НОД}(69; 46) &= 23
	\end{align*}
	
	\textbf{Ответ:} 23~ученика.
	
	\subsection*{Задача 5}
	\textbf{Условие:} Лента разрезается на куски 35~см и~50~см без обрезков. Какой наименьший размер ленты? (в~метрах)
	
	\textbf{Решение:}
	Ищем НОК(35; 50).
	
	\begin{align*}
		35 &= 5 \cdot 7 \\
		50 &= 2 \cdot 5^2 \\
		\text{НОК}(35; 50) &= 2 \cdot 5^2 \cdot 7 = 2 \cdot 25 \cdot 7 = 350 \text{ см}
	\end{align*}
	
	В~метрах: $350$~см~$= 3{,}5$~м
	
	Проверка: $350 : 35 = 10$~кусков ✓, $350 : 50 = 7$~кусков ✓
	
	\textbf{Ответ:} 3,5~метра.
	
	\newpage
	
	\section*{Проверочная работа -- Вариант 1}
	
	\begin{enumerate}
		\item \textbf{Задача:} В~сувенирном магазине 36~ракушек одного вида, 48~другого и~72~третьего. Какое наибольшее число одинаковых наборов?
		
		\textbf{Решение:} НОД(36; 48; 72)~= 12
		
		\textbf{Ответ:} \boxed{12}
		
		\item \textbf{Задача:} В~конкурсе положили в~волшебные сундучки 18~жемчужин и~24~кристалла поровну. Сколько сундучков?
		
		\textbf{Решение:} НОД(18; 24)~= 6
		
		\textbf{Ответ:} \boxed{6}
		
		\item \textbf{Задача:} Марш солдат: 12~человек в~шеренге и~18~человек. Какое минимальное число солдат?
		
		\textbf{Решение:} НОК(12; 18)~= 36
		
		\textbf{Ответ:} \boxed{36}
		
		\item \textbf{Задача:} Лента разрезается на куски 35~см и~50~см. Какой наименьший размер? (в~см)
		
		\textbf{Решение:} НОК(35; 50)~= 350~см
		
		\textbf{Ответ:} \boxed{350}
		
		\item \textbf{Задача:} На участок 54~м на 48~м ставят столбы забора. Максимальное расстояние между столбами?
		
		\textbf{Решение:} НОД(54; 48)~= 6~м
		
		\textbf{Ответ:} \boxed{6}
	\end{enumerate}
	
	\newpage
	
	\section*{Проверочная работа -- Вариант 2}
	
	\begin{enumerate}
		\item \textbf{Ответ:} \boxed{6}
		
		\item \textbf{Ответ:} \boxed{13}
		
		\item \textbf{Ответ:} \boxed{12}
		
		\item \textbf{Ответ:} \boxed{90}
		
		\item \textbf{Ответ:} \boxed{12}
	\end{enumerate}
	
	\section*{Проверочная работа -- Вариант 3}
	
	\begin{enumerate}
		\item \textbf{Ответ:} \boxed{37}
		
		\item \textbf{Ответ:} \boxed{300}
		
		\item \textbf{Ответ:} \boxed{60}
		
		\item \textbf{Ответ:} \boxed{6}
		
		\item \textbf{Ответ:} \boxed{42}
	\end{enumerate}
	
	\section*{Проверочная работа -- Вариант 4}
	
	\begin{enumerate}
		\item \textbf{Ответ:} \boxed{8}
		
		\item \textbf{Ответ:} \boxed{29}
		
		\item \textbf{Ответ:} \boxed{72}
		
		\item \textbf{Ответ:} \boxed{24}
		
		\item \textbf{Ответ:} \boxed{37}
	\end{enumerate}
	
\end{document}
