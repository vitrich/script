% Красивая презентация о шкалах в Beamer (Петерсон)
\documentclass[10pt,utf8,russian]{beamer}

\usepackage[T2A]{fontenc}
\usepackage[utf8]{inputenc}
\usepackage[russian]{babel}
\usepackage{amssymb}
\usepackage{amsmath}
\usepackage{mathtools}
\usepackage{tikz}
\usetikzlibrary{shapes,arrows,positioning,calc}
\usepackage{graphicx}
\usepackage{fancybox}

% Цветовая схема (современная и контрастная)
\definecolor{darkblue}{RGB}{25, 70, 130}
\definecolor{lightblue}{RGB}{100, 180, 230}
\definecolor{accent}{RGB}{230, 80, 100}
\definecolor{accent2}{RGB}{80, 180, 100}
\definecolor{lightgray}{RGB}{240, 240, 245}

% Тема Beamer с кастомизацией
\usetheme{Madrid}
\usecolortheme{default}
\setbeamercolor{frametitle}{bg=darkblue, fg=white}
\setbeamercolor{structure}{fg=darkblue}
\setbeamercolor{normal text}{fg=darkblue}
\setbeamercolor{alerted text}{fg=accent}
\setbeamerfont{frametitle}{size=\Large, series=\bfseries}
\setbeamerfont{block title}{size=\large, series=\bfseries}

% Шапка с полоской цвета
\setbeamertemplate{headline}{
  \leavevmode%
  \hbox{%
    \begin{beamercolorbox}[wd=\paperwidth,ht=2.5ex,dp=1.125ex]{structure}%
      \rule{0pt}{1pt}%
    \end{beamercolorbox}}%
}

% Нижняя часть с номером слайда
\setbeamertemplate{footline}{
  \leavevmode%
  \hbox{%
    \begin{beamercolorbox}[wd=.5\paperwidth,ht=2.25ex,dp=1ex,left]{section in head/foot}%
      \hspace{2mm}\insertshortinstitute%
    \end{beamercolorbox}%
    \begin{beamercolorbox}[wd=.5\paperwidth,ht=2.25ex,dp=1ex,right]{section in head/foot}%
      \insertframenumber{} / \inserttotalframenumber\hspace{2mm}%
    \end{beamercolorbox}}%
  \vskip0pt%
}

% Убрать навигационные символы
\beamertemplatenavigationsymbolsempty

% Название презентации
\title{\textbf{Шкалы}}
\subtitle{Как измерять и читать шкалы \\ Программа \textit{Петерсон}}
\author{Учебная презентация}
\institute{Начальная школа}
\date{\today}

\begin{document}

% Титульный слайд
\begin{frame}[plain]
  \begin{tikzpicture}[remember picture, overlay]
    \fill[color=darkblue] (0,0) rectangle (\paperwidth,\paperheight);
    \fill[color=lightblue] (0,0) rectangle (\paperwidth,0.8);
    \fill[color=accent2] (0,\paperheight-1.2) rectangle (\paperwidth,\paperheight);
  \end{tikzpicture}
  
  \vspace{2cm}
  {\Large\color{white}\bfseries\inserttitle}
  
  \vspace{0.5cm}
  {\color{white}\insertsubtitle}
  
  \vspace{1.5cm}
  {\small\color{white}\insertauthor}
\end{frame}

% Слайд 1: Что такое шкала?
\begin{frame}{Что такое шкала?}
  \begin{block}{Определение}
    \textbf{Шкала} — это система делений на приборе, предназначенная для измерения величин.
  \end{block}
  
  \vspace{0.5cm}
  \textbf{Примеры шкал в жизни:}
  \begin{itemize}
    \item \alert{Линейка} — шкала для измерения длины
    \item \alert{Термометр} — шкала для измерения температуры
    \item \alert{Часы} — шкала для измерения времени
    \item \alert{Спидометр} — шкала для измерения скорости
    \item \alert{Весы} — шкала для измерения массы
  \end{itemize}
  
  \vspace{0.5cm}
  \begin{center}
    \tikz{
      \draw[line width=2pt, color=darkblue] (0,0) -- (6,0);
      \foreach \x in {0,1,2,3,4,5,6}
        \draw[line width=1pt, color=darkblue] (\x,-0.2) -- (\x,0.2);
      \node at (-0.5,-0.5) {0};
      \node at (6.5,-0.5) {6};
    }
  \end{center}
\end{frame}

% Слайд 2: Части шкалы
\begin{frame}{Части шкалы и их названия}
  \begin{columns}[T]
    \column{0.5\textwidth}
    \begin{block}{Элементы шкалы:}
      \begin{enumerate}
        \item \textbf{Штрихи} — разметки
        \item \textbf{Числа} — подписи
        \item \textbf{Деления} — промежутки между штрихами
      \end{enumerate}
    \end{block}
    
    \column{0.5\textwidth}
    \begin{center}
      \tikz[scale=1.2]{
        % Штрихи
        \draw[line width=2pt, color=darkblue] (0,0) -- (5,0);
        \foreach \x in {0,1,2,3,4,5}
          \draw[line width=2pt, color=darkblue] (\x,-0.3) -- (\x,0.3);
        
        % Числа
        \node[below] at (0,-0.6) {0};
        \node[below] at (1,-0.6) {1};
        \node[below] at (2,-0.6) {2};
        \node[below] at (3,-0.6) {3};
        \node[below] at (4,-0.6) {4};
        \node[below] at (5,-0.6) {5};
        
        % Стрелки с подписями
        \draw[->, line width=1pt, color=accent2] (0.5,1) -- (0.5,0.5);
        \node[above, color=accent2] at (0.5,1.1) {\small деление};
        
        \draw[->, line width=1pt, color=accent] (0,-0.2) -- (-0.3,-0.4);
        \node[left, color=accent] at (-0.3,-0.4) {\small штрих};
      }
    \end{center}
  \end{columns}
\end{frame}

% Слайд 3: Цена деления
\begin{frame}{Цена деления шкалы}
  \begin{block}{Цена деления}
    \textbf{Цена деления} — это значение, соответствующее одному делению шкалы.
  \end{block}
  
  \vspace{0.5cm}
  \textbf{Как найти цену деления:}
  \begin{itemize}
    \item Выбрать два соседних числа на шкале
    \item Найти разность между ними
    \item Разделить на количество делений между ними
  \end{itemize}
  
  \vspace{0.8cm}
  \textbf{Пример:}
  \begin{center}
    \tikz{
      \draw[line width=2pt, color=darkblue] (0,0) -- (6,0);
      \foreach \x in {0,1,2,3,4,5,6}
        \draw[line width=1.5pt, color=darkblue] (\x,-0.25) -- (\x,0.25);
      
      \node[below] at (0,-0.7) {0};
      \node[below] at (6,-0.7) {6};
      
      \draw[<->, line width=1pt, color=accent] (0,0.7) -- (6,0.7);
      \node[above, color=accent] at (3,0.8) {6 делений};
      
      \draw[<->, line width=1pt, color=accent2] (0,-1.2) -- (1,-1.2);
      \node[below, color=accent2] at (0.5,-1.4) {1 деление = 1};
    }
    
    \vspace{0.5cm}
    \alert{Цена деления = (6 -- 0) ÷ 6 = 1}
  \end{center}
\end{frame}

% Слайд 4: Пример с линейкой
\begin{frame}{Примеры: Линейка}
  \textbf{Линейка со сложной шкалой:}
  
  \begin{center}
    \tikz[scale=0.8]{
      % Основная шкала сантиметры
      \draw[line width=2pt, color=darkblue] (0,0) -- (10,0);
      \foreach \x in {0,1,2,3,4,5,6,7,8,9,10}
        \draw[line width=2pt, color=darkblue] (\x,-0.3) -- (\x,0.4);
      
      % Подшкала миллиметры
      \foreach \x in {0,0.5,1,1.5,2,2.5,3,3.5,4,4.5,5,5.5,6,6.5,7,7.5,8,8.5,9,9.5,10}
        \draw[line width=1pt, color=lightblue] (\x,-0.15) -- (\x,0.2);
      
      % Числа сантиметров
      \node[below] at (0,-0.7) {0};
      \node[below] at (2,-0.7) {2};
      \node[below] at (4,-0.7) {4};
      \node[below] at (6,-0.7) {6};
      \node[below] at (8,-0.7) {8};
      \node[below] at (10,-0.7) {10};
      
      % Подписи
      \node at (5,-1.5) {\textbf{сантиметры}};
      \node[color=lightblue] at (5,1) {\textbf{\small миллиметры}};
    }
  \end{center}
  
  \vspace{0.8cm}
  \textbf{Цена деления:}
  \begin{itemize}
    \item Между числами: от 0 до 2 см = 2 деления, цена = \alert{1 см}
    \item Малые деления: 10 малых делений на 1 см, цена = \alert{1 мм}
  \end{itemize}
\end{frame}

% Слайд 5: Пример с координатным лучом
\begin{frame}{Пример: Координатный луч}
  \textbf{Координатный луч с единичным отрезком:}
  
  \begin{center}
    \tikz[scale=0.9]{
      \draw[->, line width=2pt, color=darkblue] (0,0) -- (10,0);
      
      \foreach \x in {0,2,4,6,8,10}
        \draw[line width=2pt, color=darkblue] (\x,-0.25) -- (\x,0.25);
      
      \foreach \x in {1,3,5,7,9}
        \draw[line width=1pt, color=lightblue] (\x,-0.15) -- (\x,0.15);
      
      \node[below] at (0,-0.7) {0};
      \node[below] at (2,-0.7) {1};
      \node[below] at (4,-0.7) {2};
      \node[below] at (6,-0.7) {3};
      \node[below] at (8,-0.7) {4};
      \node[below] at (10,-0.7) {5};
      
      \draw[<->, line width=1.5pt, color=accent2] (0,-1.2) -- (2,-1.2);
      \node[below, color=accent2] at (1,-1.5) {\textbf{единичный отрезок}};
    }
  \end{center}
  
  \vspace{0.8cm}
  \textbf{Как работать:}
  \begin{itemize}
    \item \alert{Единичный отрезок} = расстояние между 0 и 1
    \item Все деления имеют одинаковый размер
    \item Точка с координатой 3 находится на расстоянии 3 единичных отрезков от 0
  \end{itemize}
\end{frame}

% Слайд 6: Алгоритм чтения шкалы
\begin{frame}{Алгоритм чтения показаний шкалы}
  \begin{enumerate}
    \item \textbf{Найти} два соседних числа на шкале
    \item \textbf{Вычислить} разность чисел
    \item \textbf{Посчитать} количество делений между числами
    \item \textbf{Разделить} разность на количество делений — получим цену деления
    \item \textbf{Определить} положение нужной точки
    \item \textbf{Умножить} цену деления на количество делений от нулевого штриха
  \end{enumerate}
  
  \vspace{1cm}
  \begin{block}{Формула}
    $$\text{Показание} = \text{число} + \text{цена деления} \times \text{количество доп. делений}$$
  \end{block}
\end{frame}

% Слайд 7: Практический пример
\begin{frame}{Практический пример}
  \textbf{Найдите показание приборов:}
  
  \vspace{0.5cm}
  \textbf{Задача 1: Термометр}
  \begin{center}
    \tikz[scale=0.8]{
      % Шкала термометра
      \draw[line width=2pt, color=darkblue] (0,-3) -- (0,3);
      \foreach \y in {-3,-2,-1,0,1,2,3}
        \draw[line width=1.5pt, color=darkblue] (-0.3,\y) -- (0.3,\y);
      
      % Числа
      \node[left] at (-0.5,-3) {$-30$};
      \node[left] at (-0.5,-1) {$-10$};
      \node[left] at (-0.5,1) {$10$};
      \node[left] at (-0.5,3) {$30$};
      
      % Столбик
      \fill[color=red, opacity=0.5] (-0.15,-3) rectangle (0.15,0.5);
      
      % Точка показания
      \draw[circle, draw=accent, line width=2pt, fill=accent] (0,0.5) circle (0.15);
    }
  \end{center}
  
  \vspace{0.5cm}
  \textbf{Решение:}
  \begin{itemize}
    \item Цена деления = (10 -- (-10)) ÷ 4 = 5°
    \item Показание = 10 + 5 = \alert{15°}
  \end{itemize}
\end{frame}

% Слайд 8: Задание для закрепления
\begin{frame}{Задание для вас}
  \begin{block}{Задача}
    На шкале линейки найти показание точки А:
  \end{block}
  
  \begin{center}
    \tikz[scale=1.1]{
      \draw[line width=2pt, color=darkblue] (0,0) -- (8,0);
      \foreach \x in {0,1,2,3,4,5,6,7,8}
        \draw[line width=1.5pt, color=darkblue] (\x,-0.2) -- (\x,0.3);
      
      \node[below] at (0,-0.6) {0};
      \node[below] at (4,-0.6) {4};
      \node[below] at (8,-0.6) {8};
      
      % Точка А
      \draw[circle, draw=accent2, line width=3pt, fill=accent2] (5.5,0) circle (0.15);
      \node[above, color=accent2] at (5.5,0.7) {\textbf{А}};
    }
  \end{center}
  
  \vspace{1cm}
  \textbf{Вопросы:}
  \begin{itemize}
    \item Какова цена деления на этой шкале?
    \item Какое показание у точки А?
    \item На сколько делений точка А удалена от нуля?
  \end{itemize}
\end{frame}

% Слайд 9: Применение в жизни
\begin{frame}{Где мы используем шкалы?}
  \begin{columns}[T]
    \column{0.5\textwidth}
    \textbf{Дома:}
    \begin{itemize}
      \item Часы
      \item Весы
      \item Термометр
      \item Микроволновка
    \end{itemize}
    
    \column{0.5\textwidth}
    \textbf{На улице:}
    \begin{itemize}
      \item Спидометр
      \item Дорожные знаки расстояния
      \item Столб с метками высоты
      \item Барометр
    \end{itemize}
  \end{columns}
  
  \vspace{1cm}
  \begin{center}
    \Large\alert{Шкалы везде, где нужно что-то измерять!}
  \end{center}
\end{frame}

% Слайд 10: Итоги
\begin{frame}{Что мы узнали?}
  \begin{block}{Ключевые понятия}
    \begin{itemize}
      \item \textbf{Шкала} — система делений для измерения
      \item \textbf{Цена деления} — значение одного деления
      \item \textbf{Штрихи и числа} — элементы шкалы
      \item \textbf{Координатный луч} — шкала с единичными отрезками
    \end{itemize}
  \end{block}
  
  \vspace{0.8cm}
  \begin{block}{Алгоритм}
    Чтобы прочитать показание:
    \begin{enumerate}
      \item Найти цену деления
      \item Посчитать количество делений от нуля
      \item Умножить и добавить к ближайшему числу
    \end{enumerate}
  \end{block}
\end{frame}

% Итоговый слайд
\begin{frame}[plain]
  \begin{tikzpicture}[remember picture, overlay]
    \fill[color=darkblue] (0,0) rectangle (\paperwidth,\paperheight);
    \fill[color=accent2] (0,0) rectangle (\paperwidth,0.8);
  \end{tikzpicture}
  
  \vspace{3cm}
  \begin{center}
    {\Large\color{white}\bfseries Спасибо за внимание!}
    
    \vspace{1.5cm}
    {\color{accent}\Large Все ясно?}
    
    \vspace{1cm}
    {\small\color{white} Вопросы и задания помогут закрепить материал}
  \end{center}
\end{frame}

\end{document}