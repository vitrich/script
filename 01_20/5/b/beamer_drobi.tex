% Красивая презентация о дробях в Beamer (Петерсон)
\documentclass[10pt,utf8,russian]{beamer}

\usepackage[T2A]{fontenc}
\usepackage[utf8]{inputenc}
\usepackage[russian]{babel}
\usepackage{amssymb}
\usepackage{amsmath}
\usepackage{mathtools}
\usepackage{tikz}
\usetikzlibrary{shapes,arrows,positioning,calc}
\usepackage{graphicx}
\usepackage{fancybox}

% Цветовая схема (современная и контрастная)
\definecolor{darkblue}{RGB}{25, 70, 130}
\definecolor{lightblue}{RGB}{100, 180, 230}
\definecolor{accent}{RGB}{230, 80, 100}
\definecolor{accent2}{RGB}{80, 180, 100}
\definecolor{lightgray}{RGB}{240, 240, 245}

% Тема Beamer с кастомизацией
\usetheme{Madrid}
\usecolortheme{default}
\setbeamercolor{frametitle}{bg=darkblue, fg=white}
\setbeamercolor{structure}{fg=darkblue}
\setbeamercolor{normal text}{fg=darkblue}
\setbeamercolor{alerted text}{fg=accent}
\setbeamerfont{frametitle}{size=\Large, series=\bfseries}
\setbeamerfont{block title}{size=\large, series=\bfseries}

% Шапка с полоской цвета
\setbeamertemplate{headline}{
  \leavevmode%
  \hbox{%
    \begin{beamercolorbox}[wd=\paperwidth,ht=2.5ex,dp=1.125ex]{structure}%
      \rule{0pt}{1pt}%
    \end{beamercolorbox}}%
}

% Нижняя часть с номером слайда
\setbeamertemplate{footline}{
  \leavevmode%
  \hbox{%
    \begin{beamercolorbox}[wd=.5\paperwidth,ht=2.25ex,dp=1ex,left]{section in head/foot}%
      \hspace{2mm}\insertshortinstitute%
    \end{beamercolorbox}%
    \begin{beamercolorbox}[wd=.5\paperwidth,ht=2.25ex,dp=1ex,right]{section in head/foot}%
      \insertframenumber{} / \inserttotalframenumber\hspace{2mm}%
    \end{beamercolorbox}}%
  \vskip0pt%
}

% Убрать навигационные символы
\beamertemplatenavigationsymbolsempty

% Название презентации
\title{\textbf{Дроби: Часть от целого}}
\subtitle{Как найти часть от целого числа \\ Программа \textit{Петерсон}}
\author{Учебная презентация}
\institute{Начальная школа, 4-5 класс}
\date{\today}

\begin{document}

% Титульный слайд
\begin{frame}[plain]
  \begin{tikzpicture}[remember picture, overlay]
    \fill[color=darkblue] (0,0) rectangle (\paperwidth,\paperheight);
    \fill[color=lightblue] (0,0) rectangle (\paperwidth,0.8);
    \fill[color=accent2] (0,\paperheight-1.2) rectangle (\paperwidth,\paperheight);
  \end{tikzpicture}
  
  \vspace{2cm}
  {\Large\color{white}\bfseries\inserttitle}
  
  \vspace{0.5cm}
  {\color{white}\insertsubtitle}
  
  \vspace{1.5cm}
  {\small\color{white}\insertauthor}
\end{frame}

% Слайд 1: Что такое дробь?
\begin{frame}{Что такое дробь?}
  \begin{block}{Определение}
    \textbf{Дробь} — это одна или несколько равных частей целого.
  \end{block}
  
  \vspace{0.5cm}
  \textbf{Примеры дробей в жизни:}
  \begin{itemize}
    \item Половина пирога = $\frac{1}{2}$
    \item Четверть часа = $\frac{1}{4}$ часа (15 минут)
    \item Треть яблока = $\frac{1}{3}$
    \item Пять восьмых пути
  \end{itemize}
  
  \vspace{0.5cm}
  \begin{center}
    \tikz{
      % Целый круг
      \draw[line width=2pt, color=darkblue] (0,0) circle (0.8);
      \fill[color=lightblue, opacity=0.3] (0,0) circle (0.8);
      \node at (0,0) {\Large $1$};
      
      % Половина круга
      \draw[line width=2pt, color=darkblue] (2.5,0) circle (0.8);
      \fill[color=accent2, opacity=0.6] (2.5,0) circle (0.8);
      \draw[line width=2pt, color=darkblue] (2.5,-0.8) -- (2.5,0.8);
      \node at (2.5,0) {\Large $\frac{1}{2}$};
      
      % Четверть круга
      \draw[line width=2pt, color=darkblue] (5,0) circle (0.8);
      \fill[color=accent, opacity=0.6] (5,0.8) -- (5,0) -- (5.8,0) -- (5,0.8);
      \draw[line width=2pt, color=darkblue] (5,-0.8) -- (5,0.8);
      \draw[line width=2pt, color=darkblue] (4.2,0) -- (5.8,0);
      \node at (5,0) {\Large $\frac{1}{4}$};
    }
  \end{center}
\end{frame}

% Слайд 2: Части дроби
\begin{frame}{Части дроби: числитель и знаменатель}
  \begin{center}
    {\huge $\dfrac{3}{4}$}
  \end{center}
  
  \vspace{0.5cm}
  \begin{columns}[T]
    \column{0.5\textwidth}
    \textbf{Знаменатель} (внизу):
    \begin{itemize}
      \item На сколько частей разделили целое
      \item Показывает размер одной части
      \item В примере: \alert{4} части
    \end{itemize}
    
    \column{0.5\textwidth}
    \textbf{Числитель} (вверху):
    \begin{itemize}
      \item Сколько частей взяли
      \item Показывает количество
      \item В примере: взяли \alert{3} части
    \end{itemize}
  \end{columns}
  
  \vspace{1cm}
  \begin{block}{Прочтение}
    $\dfrac{3}{4}$ читается: \textbf{три четвёртых}
  \end{block}
\end{frame}

% Слайд 3: Понимание дробей на примерах
\begin{frame}{Примеры: Как разделить целое на части}
  \textbf{Пример 1: Шоколад}
  \begin{center}
    \tikz{
      % Целый шоколад
      \draw[line width=2pt, color=darkblue] (0,0) rectangle (2,1);
      \foreach \x in {0.4,0.8,1.2,1.6} {\draw[line width=1pt, color=darkblue] (\x,0) -- (\x,1);}
      \foreach \y in {0.5} {\draw[line width=1pt, color=darkblue] (0,\y) -- (2,\y);}
      \fill[color=accent2, opacity=0.5] (0,0) rectangle (0.4,0.5);
      \fill[color=accent2, opacity=0.5] (0.4,0) rectangle (0.8,0.5);
      \node at (1,-0.5) {$\dfrac{2}{8}$ шоколада};
    }
  \end{center}
  
  \vspace{0.5cm}
  \textbf{Пример 2: Яблоки}
  \begin{center}
    \tikz[scale=0.8]{
      \draw[circle, line width=2pt, color=darkblue, fill=lightblue, opacity=0.5] (0,0) circle (0.5);
      \draw[circle, line width=2pt, color=darkblue, fill=accent2, opacity=0.7] (1,0) circle (0.5);
      \draw[circle, line width=2pt, color=darkblue, fill=accent2, opacity=0.7] (2,0) circle (0.5);
      \draw[circle, line width=2pt, color=darkblue, fill=lightblue, opacity=0.5] (3,0) circle (0.5);
      \draw[circle, line width=2pt, color=darkblue, fill=lightblue, opacity=0.5] (4,0) circle (0.5);
      \node at (2,-1.5) {$\dfrac{2}{5}$ яблок};
    }
  \end{center}
\end{frame}

% Слайд 4: Алгоритм 1 — Нахождение части от целого
\begin{frame}{Алгоритм 1: Найти часть от целого}
  \textbf{Задача:} Найти $\dfrac{3}{4}$ от 20 конфет
  
  \vspace{0.8cm}
  \begin{enumerate}
    \item \textbf{Разделить} целое число на знаменатель (узнаём одну часть)
    $$20 \div 4 = 5 \text{ конфет в одной части}$$
    
    \item \textbf{Умножить} результат на числитель (берём нужное количество частей)
    $$5 \times 3 = 15 \text{ конфет}$$
  \end{enumerate}
  
  \vspace{0.8cm}
  \begin{block}{Ответ}
    $\dfrac{3}{4}$ от 20 = \alert{15 конфет}
  \end{block}
  
  \vspace{0.5cm}
  \textbf{Формула:}
  $$\boxed{\text{Часть} = \text{Целое} \div \text{знаменатель} \times \text{числитель}}$$
\end{frame}

% Слайд 5: Алгоритм 2 — Альтернативный способ
\begin{frame}{Алгоритм 2: Второй способ (умножение)}
  \textbf{Задача:} Найти $\dfrac{3}{4}$ от 20 конфет
  
  \vspace{0.8cm}
  \begin{enumerate}
    \item \textbf{Умножить} целое число на числитель
    $$20 \times 3 = 60$$
    
    \item \textbf{Разделить} результат на знаменатель
    $$60 \div 4 = 15 \text{ конфет}$$
  \end{enumerate}
  
  \vspace{0.8cm}
  \begin{block}{Ответ}
    $\dfrac{3}{4}$ от 20 = \alert{15 конфет}
  \end{block}
  
  \vspace{0.5cm}
  \textbf{Формула:}
  $$\boxed{\text{Часть} = \text{Целое} \times \text{числитель} \div \text{знаменатель}}$$
\end{frame}

% Слайд 6: Практический пример 1
\begin{frame}{Практический пример: Деревья в саду}
  \begin{block}{Задача}
    В саду 60 деревьев. $\dfrac{1}{3}$ из них яблони. Сколько яблонь?
  \end{block}
  
  \vspace{0.8cm}
  \textbf{Решение (способ 1):}
  \begin{itemize}
    \item $60 \div 3 = 20$ (одна третья часть)
    \item $20 \times 1 = 20$ яблонь
  \end{itemize}
  
  \vspace{0.5cm}
  \textbf{Решение (способ 2):}
  \begin{itemize}
    \item $60 \times 1 = 60$
    \item $60 \div 3 = 20$ яблонь
  \end{itemize}
  
  \vspace{0.8cm}
  \begin{center}
    \tikz[scale=0.6]{
      \foreach \x in {0,1,2} {
        \foreach \y in {0,1,2,3,4} {
          \draw[circle, line width=1pt, color=darkblue, fill=lightblue, opacity=0.3] (\x*2,\y*1.5) circle (0.4);
        }
      }
      \foreach \x in {0} {
        \foreach \y in {0,1,2,3,4} {
          \draw[circle, line width=2pt, color=darkblue, fill=accent2, opacity=0.7] (\x*2,\y*1.5) circle (0.4);
        }
      }
      \node at (3,-1.5) {\Large \alert{20 яблонь}};
    }
  \end{center}
\end{frame}

% Слайд 7: Практический пример 2
\begin{frame}{Практический пример: Длина пути}
  \begin{block}{Задача}
    Длина дороги 84 км. Путник прошёл $\dfrac{5}{7}$ пути. Сколько км он прошёл?
  \end{block}
  
  \vspace{0.8cm}
  \textbf{Решение:}
  \begin{enumerate}
    \item Одна седьмая часть: $84 \div 7 = 12$ км
    \item Пять седьмых части: $12 \times 5 = 60$ км
  \end{enumerate}
  
  \vspace{0.8cm}
  \begin{center}
    \tikz{
      \draw[line width=2pt, color=darkblue] (0,0) -- (7,0);
      \foreach \x in {0,1,2,3,4,5,6,7}
        \draw[line width=1.5pt, color=darkblue] (\x,-0.2) -- (\x,0.2);
      
      \fill[color=accent2, opacity=0.6] (0,-0.3) rectangle (5,0.3);
      
      \node[below] at (0,-0.7) {0};
      \node[below] at (7,-0.7) {84 км};
      \node[above, color=accent2] at (2.5,0.7) {\textbf{60 км}};
    }
  \end{center}
  
  \vspace{0.5cm}
  \alert{Ответ: путник прошёл 60 км}
\end{frame}

% Слайд 8: Проверка результата
\begin{frame}{Как проверить правильность ответа?}
  \begin{block}{Способ проверки}
    Если мы нашли $\dfrac{3}{4}$ от 20 и получили 15, то:
    $$15 \times 4 = 60 \quad \Rightarrow \quad 60 \div 3 = 20 \text{ ✓}$$
  \end{block}
  
  \vspace{0.8cm}
  \textbf{Схема проверки:}
  \begin{center}
    \tikz{
      % Коробка с числом
      \draw[rectangle, line width=2pt, color=darkblue] (-1,-0.5) rectangle (1,0.5);
      \node at (0,0) {\Large 20};
      
      % Стрелка вниз
      \draw[->, line width=2pt, color=accent2] (0,-0.5) -- (0,-1.2);
      \node[right, color=accent2] at (0.2,-0.9) {$\div 4 \times 3$};
      
      % Результат
      \draw[rectangle, line width=2pt, color=accent2] (-1,-1.7) rectangle (1,-0.7);
      \node at (0,-1.2) {\Large 15};
      
      % Стрелка вверх для проверки
      \draw[->, line width=2pt, color=accent] (0,-0.7) -- (0,0);
      \node[left, color=accent] at (-0.2,-0.35) {$\times 4 \div 3$};
    }
  \end{center}
  
  \vspace{0.8cm}
  \textbf{Правило:} Обратные операции должны дать исходное число!
\end{frame}

% Слайд 9: Особый случай — Половина
\begin{frame}{Особые случаи: Половина и треть}
  \textbf{Половина = $\dfrac{1}{2}$}
  \begin{itemize}
    \item От 10 конфет: $10 \div 2 = \alert{5}$ конфет
    \item От 84 км: $84 \div 2 = \alert{42}$ км
    \item \textbf{Запомни:} просто делим на 2!
  \end{itemize}
  
  \vspace{0.8cm}
  \textbf{Треть = $\dfrac{1}{3}$}
  \begin{itemize}
    \item От 30 яблок: $30 \div 3 = \alert{10}$ яблок
    \item От 90 деревьев: $90 \div 3 = \alert{30}$ деревьев
    \item \textbf{Запомни:} просто делим на 3!
  \end{itemize}
  
  \vspace{0.8cm}
  \textbf{Четверть = $\dfrac{1}{4}$}
  \begin{itemize}
    \item От 40 конфет: $40 \div 4 = \alert{10}$ конфет
    \item \textbf{Запомни:} просто делим на 4!
  \end{itemize}
\end{frame}

% Слайд 10: Связь с процентами
\begin{frame}{Дроби и проценты}
  \textbf{Как дроби связаны с процентами?}
  
  \vspace{0.5cm}
  \begin{center}
    \begin{tabular}{|c|c|c|}
      \hline
      \textbf{Дробь} & \textbf{Процент} & \textbf{Пример} \\
      \hline
      $\dfrac{1}{2}$ & 50\% & половина \\
      \hline
      $\dfrac{1}{4}$ & 25\% & четверть \\
      \hline
      $\dfrac{1}{5}$ & 20\% & одна пятая \\
      \hline
      $\dfrac{1}{10}$ & 10\% & одна десятая \\
      \hline
      $\dfrac{3}{4}$ & 75\% & три четвёртых \\
      \hline
    \end{tabular}
  \end{center}
  
  \vspace{0.8cm}
  \textbf{Пример:} 50\% от 200 = $\dfrac{1}{2}$ от 200 = 100
  
  \vspace{0.3cm}
  \textbf{Пример:} 25\% от 80 = $\dfrac{1}{4}$ от 80 = 20
\end{frame}

% Слайд 11: Итоги
\begin{frame}{Что мы узнали?}
  \begin{block}{Главные формулы}
    $$\text{Часть} = \text{Целое} \div \text{знаменатель} \times \text{числитель}$$
    \vspace{0.3cm}
    или
    \vspace{0.3cm}
    $$\text{Часть} = \text{Целое} \times \text{числитель} \div \text{знаменатель}$$
  \end{block}
  
  \vspace{0.8cm}
  \textbf{Помни:}
  \begin{itemize}
    \item \alert{Знаменатель} показывает на сколько частей делим
    \item \alert{Числитель} показывает сколько частей берём
    \item \alert{Всегда проверяй} обратной операцией
    \item \alert{Половина} = просто деление на 2
  \end{itemize}
\end{frame}

% Итоговый слайд
\begin{frame}[plain]
  \begin{tikzpicture}[remember picture, overlay]
    \fill[color=darkblue] (0,0) rectangle (\paperwidth,\paperheight);
    \fill[color=accent2] (0,0) rectangle (\paperwidth,0.8);
  \end{tikzpicture}
  
  \vspace{3cm}
  \begin{center}
    {\Large\color{white}\bfseries Теперь ты знаешь как найти\\часть от целого!}
    
    \vspace{1.5cm}
    {\color{accent}\Large Попробуй сам!}
    
    \vspace{1cm}
    {\small\color{white} Практикуйся на заданиях из учебника}
  \end{center}
\end{frame}

\end{document}