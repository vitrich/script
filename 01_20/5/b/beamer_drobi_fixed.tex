% Красивая презентация о дробях в Beamer (Петерсон)
\documentclass[10pt]{beamer}

\usepackage[T2A]{fontenc}
\usepackage[utf8]{inputenc}
\usepackage[russian]{babel}
\usepackage{amssymb}
\usepackage{amsmath}
\usepackage{mathtools}
\usepackage{tikz}
\usetikzlibrary{shapes,arrows,positioning,calc}
\usepackage{graphicx}
\usepackage{fancybox}

% Цветовая схема (современная и контрастная)
\definecolor{darkblue}{RGB}{25, 70, 130}
\definecolor{lightblue}{RGB}{100, 180, 230}
\definecolor{accent}{RGB}{230, 80, 100}
\definecolor{accent2}{RGB}{80, 180, 100}
\definecolor{lightgray}{RGB}{240, 240, 245}

% Тема Beamer с кастомизацией
\usetheme{Madrid}
\usecolortheme{default}
\setbeamercolor{frametitle}{bg=darkblue, fg=white}
\setbeamercolor{structure}{fg=darkblue}
\setbeamercolor{normal text}{fg=darkblue}
\setbeamercolor{alerted text}{fg=accent}
\setbeamerfont{frametitle}{size=\Large, series=\bfseries}
\setbeamerfont{block title}{size=\large, series=\bfseries}

% Шапка с полоской цвета
\setbeamertemplate{headline}{
  \leavevmode%
  \hbox{%
    \begin{beamercolorbox}[wd=\paperwidth,ht=2.5ex,dp=1.125ex]{structure}%
      \rule{0pt}{1pt}%
    \end{beamercolorbox}}%
}

% Нижняя часть с номером слайда
\setbeamertemplate{footline}{
  \leavevmode%
  \hbox{%
    \begin{beamercolorbox}[wd=.5\paperwidth,ht=2.25ex,dp=1ex,left]{section in head/foot}%
      \hspace{2mm}\insertshortinstitute%
    \end{beamercolorbox}%
    \begin{beamercolorbox}[wd=.5\paperwidth,ht=2.25ex,dp=1ex,right]{section in head/foot}%
      \insertframenumber{} / \inserttotalframenumber\hspace{2mm}%
    \end{beamercolorbox}}%
  \vskip0pt%
}

% Убрать навигационные символы
\beamertemplatenavigationsymbolsempty

% Название презентации
\title{\textbf{Дроби: Часть от целого}}
\subtitle{Как найти часть от целого числа \\ Программа \textit{Петерсон}}
\author{Учебная презентация}
\institute{Начальная школа, 4-5 класс}
\date{\today}

\begin{document}

% Титульный слайд
\begin{frame}[plain]
	\begin{tikzpicture}[remember picture, overlay]
		\fill[color=darkblue] (0,0) rectangle (\paperwidth,\paperheight);
		\fill[color=lightblue] (0,0) rectangle (\paperwidth,0.8);
		\fill[color=accent2] (0,\paperheight-1.2) rectangle (\paperwidth,\paperheight);
	\end{tikzpicture}

	\vspace{2cm}
	{\Large\color{white}\bfseries\inserttitle}

	\vspace{0.5cm}
	{\color{white}\insertsubtitle}

	\vspace{1.5cm}
	{\small\color{white}\insertauthor}
\end{frame}

% Слайд 1: Что такое дробь?
\begin{frame}{Что такое дробь?}
	\begin{block}{Определение}
		\textbf{Дробь} — это одна или несколько равных частей целого.
	\end{block}

	\vspace{0.3cm}
	\textbf{Примеры дробей в жизни:}
	\begin{itemize}
		\item Половина пирога = $\frac{1}{2}$
		\item Четверть часа = $\frac{1}{4}$ часа
		\item Треть яблока = $\frac{1}{3}$
	\end{itemize}

	\vspace{0.3cm}
	\begin{center}
		\tikz{
			% Целый круг
			\draw[line width=2pt, color=darkblue] (0,0) circle (0.8);
			\fill[color=lightblue, opacity=0.3] (0,0) circle (0.8);
			\node at (0,0) {\Large $1$};

			% Половина круга
			\draw[line width=2pt, color=darkblue] (2.5,0) circle (0.8);
			\fill[color=accent2, opacity=0.6] (2.5,0) circle (0.8);
			\draw[line width=2pt, color=darkblue] (2.5,-0.8) -- (2.5,0.8);
			\node at (2.5,0) {\Large $\frac{1}{2}$};

			% Четверть круга
			\draw[line width=2pt, color=darkblue] (5,0) circle (0.8);
			\fill[color=accent, opacity=0.6] (5,0.8) -- (5,0) -- (5.8,0) -- (5,0.8);
			\draw[line width=2pt, color=darkblue] (5,-0.8) -- (5,0.8);
			\draw[line width=2pt, color=darkblue] (4.2,0) -- (5.8,0);
			\node at (5,0) {\Large $\frac{1}{4}$};
		}
	\end{center}
\end{frame}

% Слайд 2: Части дроби
\begin{frame}{Части дроби: числитель и знаменатель}
	\begin{center}
		{\huge $\dfrac{3}{4}$}
	\end{center}

	\vspace{0.3cm}
	\begin{columns}[T]
		\column{0.5\textwidth}
		\textbf{Знаменатель} (внизу):
		\begin{itemize}
			\item На сколько частей разделили
			\item В примере: \alert{4} части
		\end{itemize}

		\column{0.5\textwidth}
		\textbf{Числитель} (вверху):
		\begin{itemize}
			\item Сколько частей взяли
			\item В примере: взяли \alert{3} части
		\end{itemize}
	\end{columns}

	\vspace{0.5cm}
	\begin{block}{Прочтение}
		$\dfrac{3}{4}$ читается: \textbf{три четвёртых}
	\end{block}
\end{frame}

% Слайд 3: Понимание дробей на примерах
\begin{frame}{Примеры: Как разделить целое}
	\textbf{Пример 1: Шоколад}
	\begin{center}
		\tikz{
			% Целый шоколад
			\draw[line width=2pt, color=darkblue] (0,0) rectangle (2,1);
			\foreach \x in {0.4,0.8,1.2,1.6} {\draw[line width=1pt, color=darkblue] (\x,0) -- (\x,1);}
			\foreach \y in {0.5} {\draw[line width=1pt, color=darkblue] (0,\y) -- (2,\y);}
			\fill[color=accent2, opacity=0.5] (0,0) rectangle (0.4,0.5);
			\fill[color=accent2, opacity=0.5] (0.4,0) rectangle (0.8,0.5);
			\node at (1,-0.5) {$\dfrac{2}{8}$ шоколада};
		}
	\end{center}

	\vspace{0.3cm}
	\textbf{Пример 2: Яблоки}
	\begin{center}
		\tikz[scale=0.8]{
			\draw[circle, line width=2pt, color=darkblue, fill=lightblue, opacity=0.5] (0,0) circle (0.5);
			\draw[circle, line width=2pt, color=darkblue, fill=accent2, opacity=0.7] (1,0) circle (0.5);
			\draw[circle, line width=2pt, color=darkblue, fill=accent2, opacity=0.7] (2,0) circle (0.5);
			\draw[circle, line width=2pt, color=darkblue, fill=lightblue, opacity=0.5] (3,0) circle (0.5);
			\draw[circle, line width=2pt, color=darkblue, fill=lightblue, opacity=0.5] (4,0) circle (0.5);
			\node at (2,-1.2) {$\dfrac{2}{5}$ яблок};
		}
	\end{center}
\end{frame}

% Слайд 4: Алгоритм 1 — Нахождение части от целого
\begin{frame}{Алгоритм 1: Найти часть от целого}
	\textbf{Задача:} Найти $\dfrac{3}{4}$ от 20 конфет

	\vspace{0.5cm}
	\begin{enumerate}
		\item \textbf{Разделить} целое на знаменатель
		      $$20 \div 4 = 5 \text{ конфет}$$

		\item \textbf{Умножить} на числитель
		      $$5 \times 3 = 15 \text{ конфет}$$
	\end{enumerate}

	\vspace{0.5cm}
	\begin{block}{Ответ}
		$\dfrac{3}{4}$ от 20 = \alert{15 конфет}
	\end{block}

	\vspace{0.3cm}
	\textbf{Формула:}
	$$\boxed{\text{Часть} = \text{Целое} \div \text{знам.} \times \text{числ.}}$$
\end{frame}

% Слайд 5: Алгоритм 2 — Альтернативный способ
\begin{frame}{Алгоритм 2: Второй способ}
	\textbf{Задача:} Найти $\dfrac{3}{4}$ от 20 конфет

	\vspace{0.5cm}
	\begin{enumerate}
		\item \textbf{Умножить} целое на числитель
		      $$20 \times 3 = 60$$

		\item \textbf{Разделить} на знаменатель
		      $$60 \div 4 = 15 \text{ конфет}$$
	\end{enumerate}

	\vspace{0.5cm}
	\begin{block}{Ответ}
		$\dfrac{3}{4}$ от 20 = \alert{15 конфет}
	\end{block}

	\vspace{0.3cm}
	\textbf{Формула:}
	$$\boxed{\text{Часть} = \text{Целое} \times \text{числ.} \div \text{знам.}}$$
\end{frame}

% Слайд 6: Практический пример 1
\begin{frame}{Пример: Деревья в саду}
	\begin{block}{Задача}
		В саду 60 деревьев. $\dfrac{1}{3}$ из них яблони. Сколько яблонь?
	\end{block}

	\vspace{0.5cm}
	\textbf{Решение:}
	\begin{itemize}
		\item $60 \div 3 = 20$ (одна треть)
		\item $20 \times 1 = 20$ яблонь
	\end{itemize}

	\vspace{0.5cm}
	\begin{center}
		\tikz[scale=0.5]{
			\foreach \x in {0,1,2} {
					\foreach \y in {0,1,2,3,4} {
							\draw[circle, line width=1pt, color=darkblue, fill=lightblue, opacity=0.3] (\x*2,\y*1.5) circle (0.4);
						}
				}
			\foreach \x in {0} {
					\foreach \y in {0,1,2,3,4} {
							\draw[circle, line width=2pt, color=darkblue, fill=accent2, opacity=0.7] (\x*2,\y*1.5) circle (0.4);
						}
				}
			\node at (3,-1.5) {\Large \alert{20 яблонь}};
		}
	\end{center}
\end{frame}

% Слайд 7: Практический пример 2
\begin{frame}{Пример: Длина пути}
	\begin{block}{Задача}
		Длина дороги 84 км. Путник прошёл $\dfrac{5}{7}$ пути. Сколько км?
	\end{block}

	\vspace{0.5cm}
	\textbf{Решение:}
	\begin{enumerate}
		\item Одна седьмая: $84 \div 7 = 12$ км
		\item Пять седьмых: $12 \times 5 = 60$ км
	\end{enumerate}

	\vspace{0.5cm}
	\begin{center}
		\tikz{
			\draw[line width=2pt, color=darkblue] (0,0) -- (7,0);
			\foreach \x in {0,1,2,3,4,5,6,7}
			\draw[line width=1.5pt, color=darkblue] (\x,-0.2) -- (\x,0.2);

			\fill[color=accent2, opacity=0.6] (0,-0.3) rectangle (5,0.3);

			\node[below] at (0,-0.7) {0};
			\node[below] at (7,-0.7) {84 км};
			\node[above, color=accent2] at (2.5,0.7) {\textbf{60 км}};
		}
	\end{center}

	\alert{Ответ: 60 км}
\end{frame}

% Слайд 8: Проверка результата
\begin{frame}{Как проверить ответ?}
	\begin{block}{Способ проверки}
		Если нашли $\dfrac{3}{4}$ от 20 = 15, то проверка:
		$$15 \times 4 = 60 \quad \Rightarrow \quad 60 \div 3 = 20$$
		Верно!
	\end{block}

	\vspace{0.5cm}
	\textbf{Схема проверки:}
	\begin{center}
		\tikz{
			% Коробка с числом
			\draw[rectangle, line width=2pt, color=darkblue] (-1,-0.5) rectangle (1,0.5);
			\node at (0,0) {\Large 20};

			% Стрелка вниз
			\draw[->, line width=2pt, color=accent2] (0,-0.5) -- (0,-1.2);
			\node[right, color=accent2] at (0.2,-0.9) {$\div 4 \times 3$};

			% Результат
			\draw[rectangle, line width=2pt, color=accent2] (-1,-1.7) rectangle (1,-0.7);
			\node at (0,-1.2) {\Large 15};

			% Стрелка вверх
			\draw[->, line width=2pt, color=accent] (1.5,-1.2) -- (1.5,-0.5);
			\node[right, color=accent] at (1.7,-0.9) {$\times 4 \div 3$};
		}
	\end{center}
\end{frame}

% Слайд 9: Особые случаи
\begin{frame}{Особые случаи}
	\textbf{Половина = $\dfrac{1}{2}$}
	\begin{itemize}
		\item От 10: $10 \div 2 = \alert{5}$
		\item От 84: $84 \div 2 = \alert{42}$
	\end{itemize}

	\vspace{0.3cm}
	\textbf{Треть = $\dfrac{1}{3}$}
	\begin{itemize}
		\item От 30: $30 \div 3 = \alert{10}$
		\item От 90: $90 \div 3 = \alert{30}$
	\end{itemize}

	\vspace{0.3cm}
	\textbf{Четверть = $\dfrac{1}{4}$}
	\begin{itemize}
		\item От 40: $40 \div 4 = \alert{10}$
	\end{itemize}
\end{frame}

% Слайд 10: Связь с процентами
\begin{frame}{Дроби и проценты}
	\begin{center}
		\begin{tabular}{|c|c|c|}
			\hline
			\textbf{Дробь}  & \textbf{Процент} & \textbf{Название} \\
			\hline
			$\dfrac{1}{2}$  & 50\%             & половина          \\
			\hline
			$\dfrac{1}{4}$  & 25\%             & четверть          \\
			\hline
			$\dfrac{1}{5}$  & 20\%             & одна пятая        \\
			\hline
			$\dfrac{1}{10}$ & 10\%             & одна десятая      \\
			\hline
		\end{tabular}
	\end{center}

	\vspace{0.5cm}
	\textbf{Примеры:}
	\begin{itemize}
		\item 50\% от 200 = $\dfrac{1}{2}$ от 200 = 100
		\item 25\% от 80 = $\dfrac{1}{4}$ от 80 = 20
	\end{itemize}
\end{frame}

% Слайд 11: Итоги
\begin{frame}{Что мы узнали?}
	\begin{block}{Главные формулы}
		$$\text{Часть} = \text{Целое} \div \text{знам.} \times \text{числ.}$$
		или
		$$\text{Часть} = \text{Целое} \times \text{числ.} \div \text{знам.}$$
	\end{block}

	\vspace{0.5cm}
	\textbf{Помни:}
	\begin{itemize}
		\item Знаменатель — на сколько частей делим
		\item Числитель — сколько частей берём
		\item Всегда проверяй обратной операцией
	\end{itemize}
\end{frame}

% Итоговый слайд
\begin{frame}[plain]
	\begin{tikzpicture}[remember picture, overlay]
		\fill[color=darkblue] (0,0) rectangle (\paperwidth,\paperheight);
		\fill[color=accent2] (0,0) rectangle (\paperwidth,0.8);
	\end{tikzpicture}

	\vspace{3cm}
	\begin{center}
		{\Large\color{white}\bfseries Теперь ты знаешь дроби!}

		\vspace{1.5cm}
		{\color{accent}\Large Попробуй сам!}

		\vspace{1cm}
		{\small\color{white} Практикуйся на задачах}
	\end{center}
\end{frame}

\end{document}
