\documentclass[12pt,a4paper,landscape]{article}
\usepackage[utf8]{inputenc}
\usepackage[russian]{babel}
\usepackage{amsmath}
\usepackage{geometry}
\usepackage{enumitem}
\usepackage{multicol}
\geometry{
	left=1cm,
	right=1cm,
	top=1cm,
	bottom=1cm,
	heightrounded
}

% Команда для линии разреза
\newcommand{\cutline}{%
	\vspace{0.3cm}%
	\noindent\makebox[\linewidth]{\rule{0.45\paperwidth}{0.4pt}}%
	\vspace{0.3cm}%
}

\pagestyle{empty}

\begin{document}
	
	% ==================== ОБЪЯСНЕНИЕ - ЛИСТ 1 ====================
	\begin{minipage}[t]{0.48\textwidth}
		\section*{Объяснение: Нахождение части и целого}
		
		\subsection*{Тип 1: Нахождение части от числа}
		
		\textbf{Идея:} Дробь показывает, на сколько равных частей разделено целое (знаменатель) и сколько таких частей берём (числитель).
		
		\textbf{Правило:} Чтобы найти часть от числа, выраженную дробью,
		нужно:
		\begin{enumerate}[leftmargin=*]
			\item разделить число на знаменатель дроби (найти одну часть);
			\item умножить результат на числитель (взять нужное количество частей).
		\end{enumerate}
		
		\textit{Пример 1:} В автобусе 51 место. Две трети мест заняты. Сколько мест занято?
		
		Одна треть мест:
		\[
		51 : 3 = 17
		\]
		
		Две трети — это две такие части:
		\[
		17 \cdot 2 = 34 \text{ (места)}
		\]
		
		\vspace{0.3cm}
		
		\textit{Пример 2:} Длина дороги 20 км. Заасфальтировали $\frac{3}{4}$ дороги. Сколько км заасфальтировали?
		
		Одна четверть дороги:
		\[
		20 : 4 = 5 \text{ км}
		\]
		
		Три четверти — три такие части:
		\[
		5 \cdot 3 = 15 \text{ км}
		\]
		
		\subsection*{Тип 2: Нахождение числа по его части}
		
		\textbf{Идея:} Известно, какая часть (дробь) и чему она равна. Надо восстановить всё целое.
		
		\textbf{Правило:} Чтобы найти число по его части, выраженной дробью,
		нужно:
		\begin{enumerate}[leftmargin=*]
			\item разделить известное число на числитель (найти одну часть);
			\item умножить результат на знаменатель (узнать всё целое).
		\end{enumerate}
		
		\textit{Пример 3:} Число 12 составляет $\frac{3}{4}$ от загаданного числа. Найти число.
		
		Одна часть (одна четверть):
		\[
		12 : 3 = 4
		\]
		
		Всё число — это 4 таких части:
		\[
		4 \cdot 4 = 16
		\]
		
	\end{minipage}%
	\hfill
	\begin{minipage}[t]{0.48\textwidth}
		\subsection*{Тип 3: Какую часть составляет число}
		
		\textbf{Идея:} Сравниваем два числа: во сколько раз одно меньше другого и записываем это в виде дроби.
		
		\textbf{Правило:} Чтобы узнать, какую часть одно число составляет от другого,
		нужно первое число разделить на второе и результат записать в виде дроби или десятичной дроби.
		
		\textit{Пример 4:} В гараже 30 зелёных машин из 120. Какую часть составляют зелёные?
		
		\[
		120 : 30 = 4
		\]
		
		Значит, зелёные — это одна часть из четырёх:
		\[
		\frac{1}{4} = 0{,}25
		\]
		
		\vspace{0.5cm}
		
		\subsection*{Как определить тип задачи?}
		
		\begin{enumerate}[leftmargin=*]
			\item \textbf{Тип 1:} Дано целое и дробь $\to$ найти часть.  
			Действия: разделить целое на знаменатель дроби, затем умножить на числитель.
			\item \textbf{Тип 2:} Дана часть и какая доля она составляет $\to$ найти целое.  
			Действия: разделить часть на числитель, затем умножить на знаменатель.
			\item \textbf{Тип 3:} Даны два числа $\to$ узнать, какую часть одно составляет от другого.  
			Действия: разделить одно число на другое и записать результат дробью или десятичной дробью.
		\end{enumerate}
		
		\vspace{0.3cm}
		
		\textbf{Важно помнить:} Дробь показывает: знаменатель — на сколько частей разделили, числитель — сколько таких частей взяли.
		
	\end{minipage}
	
	\cutline
	
	% ==================== ОБЪЯСНЕНИЕ - ЛИСТ 2 ====================
	\begin{minipage}[t]{0.48\textwidth}
		\section*{Объяснение (продолжение)}
		
		\subsection*{Комбинированные задачи}
		
		\textit{Пример 5:} Турист прошёл за первый день 18 км, что составляет $\frac{2}{3}$ пути второго дня. Сколько км он прошёл за оба дня?
		
		\begin{enumerate}[leftmargin=*]
			\item Одна часть пути второго дня (одна треть):
			\[
			18 : 2 = 9 \text{ км}
			\]
			\item Путь второго дня — три такие части:
			\[
			9 \cdot 3 = 27 \text{ км}
			\]
			\item Весь путь за два дня:
			\[
			18 + 27 = 45 \text{ км}
			\]
		\end{enumerate}
		
		\vspace{0.3cm}
		
		\textit{Пример 6:} В классе 25 учеников. $\frac{3}{5}$ --- мальчики. Сколько девочек?
		
		\begin{enumerate}[leftmargin=*]
			\item Одна часть (одна пятая) класса:
			\[
			25 : 5 = 5 \text{ учеников}
			\]
			\item Мальчики — три такие части:
			\[
			5 \cdot 3 = 15 \text{ мальчиков}
			\]
			\item Девочки:
			\[
			25 - 15 = 10 \text{ девочек}
			\]
		\end{enumerate}
		
	\end{minipage}%
	\hfill
	\begin{minipage}[t]{0.48\textwidth}
		\subsection*{Практические советы}
		
		\begin{enumerate}[leftmargin=*]
			\item Внимательно читай условие.
			\item Определи, что дано и что нужно найти.
			\item Определи тип задачи (1, 2 или 3).
			\item Подумай, что обозначает дробь в задаче: на сколько частей разделили и сколько взяли.
			\item Запиши действия: сначала деление на знаменатель или числитель, потом умножение.
			\item Проверь ответ на разумность.
		\end{enumerate}
		
		\vspace{0.5cm}
		
		\subsection*{Полезные схемы}
		
		\begin{align*}
			\text{Часть} &= \frac{\text{Целое}}{\text{знаменатель}} \times \text{числитель}\\[0.2cm]
			\text{Целое} &= \frac{\text{Часть}}{\text{числитель}} \times \text{знаменатель}\\[0.2cm]
			\text{Дробь} &= \frac{\text{Часть}}{\text{Целое}}
		\end{align*}
		
	\end{minipage}
	
	\newpage
	
	% ==================== ДОМАШНЯЯ РАБОТА - ЛИСТ 1 ====================
	\begin{minipage}[t]{0.48\textwidth}
		\section*{Домашняя работа}
		
		\textbf{Фамилия, Имя:} \underline{\hspace{5cm}}
		
		\subsection*{Часть А: Нахождение части от числа}
		
		\begin{enumerate}[leftmargin=*]
			\item В автобусе 51 место. $\frac{2}{3}$ мест заняты. Сколько свободных мест?  
			
			(Сначала найди одну треть, потом две трети, затем вычти из 51.)
			
			\vspace{0.8cm}
			
			\item От дыни 2 кг 400 г Ване отрезали $\frac{1}{3}$, Маше $\frac{1}{4}$. Сколько г осталось?  
			
			(Найди отдельно одну треть и одну четверть массы дыни.)
			
			\vspace{0.8cm}
			
			\item Петя готовил уроки 1 ч 40 мин. На математику $\frac{3}{5}$ времени. Сколько минут на географию?
			
			\vspace{0.8cm}
			
			\item Собрали 42 кг огурцов, $\frac{2}{3}$ засолили. Сколько кг свежих?
			
			\vspace{0.8cm}
			
			\item В классе 25 учеников, $\frac{3}{5}$ --- мальчики. Сколько девочек?
			
			\vspace{0.8cm}
		\end{enumerate}
		
	\end{minipage}%
	\hfill
	\begin{minipage}[t]{0.48\textwidth}
		\subsection*{Часть Б: Нахождение числа по части}
		
		\begin{enumerate}[leftmargin=*,resume]
			\item Число 12 составляет $\frac{3}{4}$ загаданного числа. Найти число.  
			
			(Раздели 12 на 3, затем умножь результат на 4.)
			
			\vspace{0.8cm}
			
			\item Костюм стоит 120 р, это $\frac{3}{5}$ всех денег. Сколько было денег?
			
			\vspace{0.8cm}
			
			\item Выгрузили 42 т, это $\frac{7}{15}$ зерна в вагоне. Сколько тонн было?
			
			\vspace{0.8cm}
			
			\item В классе 10 мальчиков, это $\frac{2}{5}$ класса. Сколько всего учащихся?
			
			\vspace{0.8cm}
			
			\item В баке 18 л, это $\frac{1}{4}$ объёма. Какой объём бака?
			
			\vspace{0.8cm}
		\end{enumerate}
		
	\end{minipage}
	
	\cutline
	
	% ==================== ДОМАШНЯЯ РАБОТА - ЛИСТ 2 ====================
	\begin{minipage}[t]{0.48\textwidth}
		\section*{Домашняя работа (продолжение)}
		
		\textbf{Фамилия, Имя:} \underline{\hspace{5cm}}
		
		\subsection*{Часть В: Какую часть составляет}
		
		\begin{enumerate}[leftmargin=*,start=11]
			\item 30 зелёных машин из 120. Какую часть составляют зелёные? (десятичной дробью)
			
			\vspace{0.8cm}
			
			\item Урок 45 мин, задача решалась 9 мин. Какая часть урока? (десятичной дробью)
			
			\vspace{0.8cm}
			
			\item 8 машин, 2 серые, остальные синие. Какую часть синие? (десятичной дробью)
			
			\vspace{0.8cm}
		\end{enumerate}
		
		\subsection*{Часть Г: Комбинированные задачи}
		
		\begin{enumerate}[leftmargin=*,start=14]
			\item Турист прошёл 18 км в первый день, это $\frac{2}{3}$ пути второго дня. Сколько км за оба дня?
			
			\vspace{0.8cm}
		\end{enumerate}
		
	\end{minipage}%
	\hfill
	\begin{minipage}[t]{0.48\textwidth}
		
		\begin{enumerate}[leftmargin=*,start=15]
			\item Картофель посадили: $\frac{2}{7}$ в первый день, $\frac{3}{14}$ во второй. Участок 14 м$^2$. Сколько м$^2$ засажено?
			
			\vspace{0.8cm}
			
			\item Ели на $\frac{8}{15}$ участка, сосны на $\frac{4}{15}$. Участок 300 га. Сколько га занято елью и сосной?
			
			\vspace{0.8cm}
			
			\item Турист прошёл $\frac{5}{12}$ пути в 1-й день, $\frac{1}{3}$ во 2-й. За два дня 36 км. Весь путь?
			
			\vspace{0.8cm}
		\end{enumerate}
		
		\vspace{1cm}
		
		\textbf{Ответы:} 1) 17 \quad 2) 1000 г \quad 3) 40 мин \quad 4) 14 кг\\
		5) 10 \quad 6) 16 \quad 7) 200 р \quad 8) 90 т \quad 9) 25\\
		10) 72 л \quad 11) 0,25 \quad 12) 0,2 \quad 13) 0,75\\
		14) 45 км \quad 15) 7 м$^2$ \quad 16) 240 га \quad 17) 48 км
		
	\end{minipage}
	
	\newpage
	
	% ==================== ПРОВЕРОЧНАЯ РАБОТА - ВАРИАНТ 1 ====================
	\begin{minipage}[t]{0.48\textwidth}
		\section*{Проверочная работа. Вариант 1}
		
		\textbf{Фамилия, Имя:} \underline{\hspace{5cm}}
		
		\textbf{Класс:} \underline{\hspace{2cm}} \textbf{Дата:} \underline{\hspace{2cm}}
		
		\begin{enumerate}[leftmargin=*]
			\item Отряд решил собрать 12 т металлолома, собрал $\frac{7}{12}$. Сколько тонн собрал?
			
			\vspace{1.2cm}
			
			\item На базу доставили 22 собаки. $\frac{9}{11}$ в упряжке. Сколько не вошло?
			
			\vspace{1.2cm}
			
			\item Купили 5 кг 600 г сахара, израсходовали $\frac{5}{7}$. Сколько г осталось?
			
			\vspace{1.2cm}
			
			\item Из сливок получили 18 кг масла, это $\frac{3}{5}$ массы сливок. Сколько кг сливок?
			
			\vspace{1.2cm}
		\end{enumerate}
		
	\end{minipage}%
	\hfill
	\begin{minipage}[t]{0.48\textwidth}
		
		\begin{enumerate}[leftmargin=*,start=5]
			\item 27 мастеров, $\frac{2}{3}$ --- гончары. Сколько художников?
			
			\vspace{1.2cm}
			
			\item $\frac{4}{9}$ класса --- девочки, мальчиков 15. Сколько всего учащихся?
			
			\vspace{1.2cm}
			
			\item Овощи: $\frac{2}{7}$ --- помидоры, $\frac{3}{7}$ --- огурцы. Огурцов 105 кг. Сколько кг помидоров?
			
			\vspace{1.2cm}
			
			\item Прошли 75 км: $\frac{4}{15}$ в 1-й день, $\frac{7}{15}$ во 2-й. Сколько км за два дня?
			
			\vspace{1.2cm}
		\end{enumerate}
		
		\vspace{0.5cm}
		
		\textbf{Ответы:} \underline{\hspace{10cm}}
		
	\end{minipage}
	
	\cutline
	
	% ==================== ПРОВЕРОЧНАЯ РАБОТА - ВАРИАНТ 2 ====================
	\begin{minipage}[t]{0.48\textwidth}
		\section*{Проверочная работа. Вариант 2}
		
		\textbf{Фамилия, Имя:} \underline{\hspace{5cm}}
		
		\textbf{Класс:} \underline{\hspace{2cm}} \textbf{Дата:} \underline{\hspace{2cm}}
		
		\begin{enumerate}[leftmargin=*]
			\item Длина дороги 36 км. Заасфальтировали $\frac{5}{9}$. Сколько км осталось?
			
			\vspace{1.2cm}
			
			\item Команда набрала 112 очков. Лучший игрок $\frac{1}{4}$ очков. Сколько очков остальные?
			
			\vspace{1.2cm}
			
			\item В книге 87 страниц. Прочитал $\frac{2}{3}$. Сколько страниц осталось?
			
			\vspace{1.2cm}
			
			\item Съели $\frac{5}{8}$ пирожков, доели 12. Сколько испекла бабушка?
			
			\vspace{1.2cm}
		\end{enumerate}
		
	\end{minipage}%
	\hfill
	\begin{minipage}[t]{0.48\textwidth}
		
		\begin{enumerate}[leftmargin=*,start=5]
			\item 80 пятиклассников, $\frac{3}{5}$ в музей. Сколько в театр?
			
			\vspace{1.2cm}
			
			\item 12 мальчиков, это $\frac{3}{7}$ класса. Сколько девочек?
			
			\vspace{1.2cm}
			
			\item Овощи: $\frac{3}{7}$ --- помидоры, $\frac{2}{7}$ --- огурцы. Огурцов 84 кг. Сколько кг помидоров?
			
			\vspace{1.2cm}
			
			\item В зале 90 мест. Продано $\frac{2}{3}$ билетов. Сколько ещё можно продать?
			
			\vspace{1.2cm}
		\end{enumerate}
		
		\vspace{0.5cm}
		
		\textbf{Ответы:} \underline{\hspace{10cm}}
		
	\end{minipage}
	
	\newpage
	
	% ==================== ОТВЕТЫ К ПРОВЕРОЧНОЙ ====================
	\begin{minipage}[t]{0.48\textwidth}
		\section*{Ответы к проверочной работе}
		
		\subsection*{Вариант 1}
		
		\begin{enumerate}[leftmargin=*]
			\item 7 тонн
			
			\textit{Решение:} одна двенадцатая плана:
			\[
			12 : 12 = 1 \text{ т}
			\]
			семь двенадцатых:
			\[
			1 \cdot 7 = 7 \text{ т}
			\]
			
			\item 2 собаки
			
			\textit{Решение:} одна одиннадцатая всех собак:
			\[
			22 : 11 = 2
			\]
			девять одиннадцатых:
			\[
			2 \cdot 9 = 18
			\]
			не вошло:
			\[
			22 - 18 = 2
			\]
			
			\item 1600 г
			
			\textit{Решение:} масса сахара:
			\[
			5{,}6 \text{ кг} = 5600 \text{ г}
			\]
			одна седьмая:
			\[
			5600 : 7 = 800 \text{ г}
			\]
			пять седьмых:
			\[
			800 \cdot 5 = 4000 \text{ г}
			\]
			осталось:
			\[
			5600 - 4000 = 1600 \text{ г}
			\]
			
			\item 30 кг
			
			\textit{Решение:} одна часть (одна треть) сливок:
			\[
			18 : 3 = 6 \text{ кг}
			\]
			всё количество (пять частей):
			\[
			6 \cdot 5 = 30 \text{ кг}
			\]
			
			\item 9 художников
			
			\textit{Решение:} одна треть:
			\[
			27 : 3 = 9
			\]
			две трети — гончары:
			\[
			9 \cdot 2 = 18
			\]
			художники:
			\[
			27 - 18 = 9
			\]
			
			\item 27 учащихся
			
			\textit{Решение:} девять частей класса (девять девятых):
			\[
			15 : 5 = 3 \text{ (одна девятая)}
			\]
			\[
			3 \cdot 9 = 27 \text{ учащихся}
			\]
			
			\item 70 кг
			
			\textit{Решение:} одна часть (одна седьмая) овощей:
			\[
			105 : 3 = 35 \text{ кг (одна седьмая всех овощей)}
			\]
			две седьмых помидоров:
			\[
			35 \cdot 2 = 70 \text{ кг}
			\]
			
			\item 55 км
			
			\textit{Решение:} одна пятнадцатая пути:
			\[
			75 : 15 = 5 \text{ км}
			\]
			доля за два дня:
			\[
			4 + 7 = 11 \text{ частей}
			\]
			\[
			5 \cdot 11 = 55 \text{ км}
			\]
		\end{enumerate}
		
	\end{minipage}%
	\hfill
	\begin{minipage}[t]{0.48\textwidth}
		
		\subsection*{Вариант 2}
		
		\begin{enumerate}[leftmargin=*]
			\item 16 км
			
			\textit{Решение:} одна девятая дороги:
			\[
			36 : 9 = 4 \text{ км}
			\]
			пять девятых:
			\[
			4 \cdot 5 = 20 \text{ км}
			\]
			осталось:
			\[
			36 - 20 = 16 \text{ км}
			\]
			
			\item 84 очка
			
			\textit{Решение:} четверть очков:
			\[
			112 : 4 = 28
			\]
			остальные:
			\[
			112 - 28 = 84
			\]
			
			\item 29 страниц
			
			\textit{Решение:} одна треть книги:
			\[
			87 : 3 = 29 \text{ страниц}
			\]
			две трети прочитано:
			\[
			29 \cdot 2 = 58 \text{ страниц}
			\]
			осталось:
			\[
			87 - 58 = 29 \text{ страниц}
			\]
			
			\item 32 пирожка
			
			\textit{Решение:} три восьмых пирожков — это 12:
			\[
			12 : 3 = 4 \text{ (одна восьмая)}
			\]
			все пирожки (восемь восьмых):
			\[
			4 \cdot 8 = 32
			\]
			
			\item 32 пятиклассника
			
			\textit{Решение:} одна пятая от 80:
			\[
			80 : 5 = 16
			\]
			три пятых:
			\[
			16 \cdot 3 = 48
			\]
			в театр:
			\[
			80 - 48 = 32
			\]
			
			\item 16 девочек
			
			\textit{Решение:} одна часть (одна седьмая) класса:
			\[
			12 : 3 = 4 \text{ (одна седьмая)}
			\]
			весь класс:
			\[
			4 \cdot 7 = 28
			\]
			девочки:
			\[
			28 - 12 = 16
			\]
			
			\item 126 кг
			
			\textit{Решение:} одна часть (одна седьмая) овощей:
			\[
			84 : 2 = 42 \text{ кг (одна седьмая)}
			\]
			помидоры — три седьмых:
			\[
			42 \cdot 3 = 126 \text{ кг}
			\]
			
			\item 30 билетов
			
			\textit{Решение:} одна треть мест:
			\[
			90 : 3 = 30
			\]
			две трети продано:
			\[
			30 \cdot 2 = 60
			\]
			осталось:
			\[
			90 - 60 = 30
			\]
		\end{enumerate}
		
	\end{minipage}
	
	\cutline
	
	% ==================== РЕЗЕРВНЫЕ ЗАДАЧИ ====================
	\begin{minipage}[t]{0.48\textwidth}
		\section*{Дополнительные задачи}
		
		\textbf{Для тех, кто быстро справился}
		
		\begin{enumerate}[leftmargin=*]
			\item Десятую часть миллиона уменьшили на 10\,000 и результат уменьшили в 1000 раз. Сколько получили?
			
			\vspace{1cm}
			
			\item Банка вмещает $\frac{3}{4}$ кг мёда. Сколько банок нужно для $\frac{15}{2}$ кг мёда?
			
			\vspace{1cm}
			
			\item Когда прочитали 35 страниц, осталось $\frac{2}{7}$ книги. Сколько страниц в книге?
			
			\vspace{1cm}
			
			\item Мама израсходовала половину денег и $\frac{1}{3}$ остатка. Осталось 6000 руб. Сколько было?
			
			\vspace{1cm}
		\end{enumerate}
		
	\end{minipage}%
	\hfill
	\begin{minipage}[t]{0.48\textwidth}
		
		\begin{enumerate}[leftmargin=*,start=5]
			\item Сыну 8 лет, его возраст $\frac{2}{9}$ возраста отца. Возраст отца $\frac{3}{5}$ возраста дедушки. Сколько лет дедушке?
			
			\vspace{1cm}
			
			\item Уменьшите 90 руб. на $\frac{1}{10}$ этой суммы.
			
			\vspace{1cm}
			
			\item Увеличьте 80 рублей на $\frac{2}{5}$ этой суммы.
			
			\vspace{1cm}
		\end{enumerate}
		
		\vspace{1cm}
		
		\textbf{Ответы:} 1) 90 \quad 2) 10 банок \quad 3) 49 страниц\\
		4) 18\,000 руб. \quad 5) 60 лет \quad 6) 81 руб. \quad 7) 112 руб.
		
	\end{minipage}
	
\end{document}
