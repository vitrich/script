\documentclass[12pt,a4paper,landscape]{article}
\usepackage[utf8]{inputenc}
\usepackage[russian]{babel}
\usepackage{amsmath}
\usepackage{geometry}
\usepackage{enumitem}
\usepackage{multicol}
\geometry{left=1cm,right=1cm,top=1cm,bottom=1cm}

% Команда для линии разреза
\newcommand{\cutline}{%
  \vspace{0.3cm}
  \noindent\makebox[\linewidth]{\rule{0.45\paperwidth}{0.4pt}}
  \vspace{0.3cm}
}

\pagestyle{empty}

\begin{document}

% ==================== ОБЪЯСНЕНИЕ - ЛИСТ 1 ====================
\begin{minipage}[t][0.48\textheight][t]{0.48\textwidth}
\section*{Объяснение: Нахождение части и целого}

\subsection*{Тип 1: Нахождение части от числа}

\textbf{Правило:} Чтобы найти часть от числа, нужно это число \textbf{умножить} на дробь.

\textit{Пример 1:} В автобусе 51 место. Две трети мест заняты. Сколько мест занято?

\textit{Решение:} $51 \cdot \frac{2}{3} = \frac{102}{3} = 34$ (места)

\vspace{0.3cm}

\textit{Пример 2:} Длина дороги 20 км. Заасфальтировали $\frac{3}{4}$ дороги. Сколько км заасфальтировали?

\textit{Решение:} $20 \cdot \frac{3}{4} = \frac{60}{4} = 15$ (км)

\subsection*{Тип 2: Нахождение числа по его части}

\textbf{Правило:} Чтобы найти число по его части, нужно известную величину \textbf{разделить} на дробь.

\textit{Пример 3:} Число 12 составляет $\frac{3}{4}$ от загаданного числа. Найти число.

\textit{Решение:} $12 : \frac{3}{4} = 12 \cdot \frac{4}{3} = 16$

\end{minipage}%
\hfill
\begin{minipage}[t][0.48\textheight][t]{0.48\textwidth}
\subsection*{Тип 3: Какую часть составляет число}

\textbf{Правило:} Чтобы узнать, какую часть одно число составляет от другого, нужно первое число \textbf{разделить} на второе.

\textit{Пример 4:} В гараже 30 зелёных машин из 120. Какую часть составляют зелёные?

\textit{Решение:} $30 : 120 = \frac{1}{4} = 0{,}25$

\vspace{0.5cm}

\subsection*{Как определить тип задачи?}

\begin{enumerate}[leftmargin=*]
\item \textbf{Тип 1:} Дано целое и дробь $\to$ найти часть $\to$ \textbf{умножить}

\item \textbf{Тип 2:} Дана часть и какую долю она составляет $\to$ найти целое $\to$ \textbf{разделить на дробь}

\item \textbf{Тип 3:} Даны два числа $\to$ найти какую часть одно от другого $\to$ \textbf{разделить}
\end{enumerate}

\vspace{0.3cm}

\textbf{Важно помнить:} Деление на дробь = умножение на обратную дробь!

\end{minipage}

\cutline

% ==================== ОБЪЯСНЕНИЕ - ЛИСТ 2 ====================
\begin{minipage}[t][0.48\textheight][t]{0.48\textwidth}
\section*{Объяснение (продолжение)}

\subsection*{Комбинированные задачи}

\textit{Пример 5:} Турист прошёл за первый день 18 км, что составляет $\frac{2}{3}$ пути второго дня. Сколько км за оба дня?

\textit{Решение:}
\begin{enumerate}[leftmargin=*]
\item Путь второго дня: $18 : \frac{2}{3} = 27$ км
\item Весь путь: $18 + 27 = 45$ км
\end{enumerate}

\vspace{0.3cm}

\textit{Пример 6:} В классе 25 учеников. $\frac{3}{5}$ --- мальчики. Сколько девочек?

\textit{Решение:}
\begin{enumerate}[leftmargin=*]
\item Мальчики: $25 \cdot \frac{3}{5} = 15$
\item Девочки: $25 - 15 = 10$
\end{enumerate}

\end{minipage}%
\hfill
\begin{minipage}[t][0.48\textheight][t]{0.48\textwidth}
\subsection*{Практические советы}

\begin{enumerate}[leftmargin=*]
\item Внимательно читай условие
\item Определи, что дано и что найти
\item Выбери тип задачи
\item Запиши формулу
\item Выполни вычисления
\item Проверь ответ на логичность
\end{enumerate}

\vspace{0.5cm}

\subsection*{Полезные формулы}

\begin{align*}
\text{Часть} &= \text{Целое} \times \text{Дробь}\\[0.2cm]
\text{Целое} &= \text{Часть} : \text{Дробь}\\[0.2cm]
\text{Дробь} &= \text{Часть} : \text{Целое}
\end{align*}

\end{minipage}

\newpage

% ==================== ДОМАШНЯЯ РАБОТА - ЛИСТ 1 ====================
\begin{minipage}[t][0.48\textheight][t]{0.48\textwidth}
\section*{Домашняя работа}

\textbf{Фамилия, Имя:} \underline{\hspace{5cm}}

\subsection*{Часть А: Нахождение части от числа}

\begin{enumerate}[leftmargin=*]
\item В автобусе 51 место. $\frac{2}{3}$ мест заняты. Сколько свободных мест?

\vspace{0.8cm}

\item От дыни 2 кг 400 г Ване отрезали $\frac{1}{3}$, Маше $\frac{1}{4}$. Сколько г осталось?

\vspace{0.8cm}

\item Петя готовил уроки 1 ч 40 мин. На математику $\frac{3}{5}$ времени. Сколько минут на географию?

\vspace{0.8cm}

\item Собрали 42 кг огурцов, $\frac{2}{3}$ засолили. Сколько кг свежих?

\vspace{0.8cm}

\item В классе 25 учеников, $\frac{3}{5}$ --- мальчики. Сколько девочек?

\vspace{0.8cm}
\end{enumerate}

\end{minipage}%
\hfill
\begin{minipage}[t][0.48\textheight][t]{0.48\textwidth}
\subsection*{Часть Б: Нахождение числа по части}

\begin{enumerate}[leftmargin=*,resume]
\item Число 12 составляет $\frac{3}{4}$ загаданного числа. Найти число.

\vspace{0.8cm}

\item Костюм стоит 120 р, это $\frac{3}{5}$ всех денег. Сколько было денег?

\vspace{0.8cm}

\item Выгрузили 42 т, это $\frac{7}{15}$ зерна в вагоне. Сколько тонн было?

\vspace{0.8cm}

\item В классе 10 мальчиков, это $\frac{2}{5}$ класса. Сколько всего учащихся?

\vspace{0.8cm}

\item В баке 18 л, это $\frac{1}{4}$ объёма. Какой объём бака?

\vspace{0.8cm}
\end{enumerate}

\end{minipage}

\cutline

% ==================== ДОМАШНЯЯ РАБОТА - ЛИСТ 2 ====================
\begin{minipage}[t][0.48\textheight][t]{0.48\textwidth}
\section*{Домашняя работа (продолжение)}

\textbf{Фамилия, Имя:} \underline{\hspace{5cm}}

\subsection*{Часть В: Какую часть составляет}

\begin{enumerate}[leftmargin=*,start=11]
\item 30 зелёных машин из 120. Какую часть составляют зелёные? (десятичной дробью)

\vspace{0.8cm}

\item Урок 45 мин, задача решалась 9 мин. Какая часть урока? (десятичной дробью)

\vspace{0.8cm}

\item 8 машин, 2 серые, остальные синие. Какую часть синие? (десятичной дробью)

\vspace{0.8cm}
\end{enumerate}

\subsection*{Часть Г: Комбинированные задачи}

\begin{enumerate}[leftmargin=*,start=14]
\item Турист прошёл 18 км в первый день, это $\frac{2}{3}$ пути второго дня. Сколько км за оба дня?

\vspace{0.8cm}
\end{enumerate}

\end{minipage}%
\hfill
\begin{minipage}[t][0.48\textheight][t]{0.48\textwidth}

\begin{enumerate}[leftmargin=*,start=15]
\item Картофель посадили: $\frac{2}{7}$ в первый день, $\frac{3}{14}$ во второй. Участок 14 м$^2$. Сколько м$^2$ засажено?

\vspace{0.8cm}

\item Ели на $\frac{8}{15}$ участка, сосны на $\frac{4}{15}$. Участок 300 га. Сколько га занято елью и сосной?

\vspace{0.8cm}

\item Турист прошёл $\frac{5}{12}$ пути в 1-й день, $\frac{1}{3}$ во 2-й. За два дня 36 км. Весь путь?

\vspace{0.8cm}
\end{enumerate}

\vspace{1cm}

\textbf{Ответы:} 1) 17 \quad 2) 1000 г \quad 3) 40 мин \quad 4) 14 кг\\
5) 10 \quad 6) 16 \quad 7) 200 р \quad 8) 90 т \quad 9) 25\\
10) 72 л \quad 11) 0,25 \quad 12) 0,2 \quad 13) 0,75\\
14) 45 км \quad 15) 7 м$^2$ \quad 16) 240 га \quad 17) 48 км

\end{minipage}

\newpage

% ==================== ПРОВЕРОЧНАЯ РАБОТА - ВАРИАНТ 1 ====================
\begin{minipage}[t][0.48\textheight][t]{0.48\textwidth}
\section*{Проверочная работа. Вариант 1}

\textbf{Фамилия, Имя:} \underline{\hspace{5cm}}

\textbf{Класс:} \underline{\hspace{2cm}} \textbf{Дата:} \underline{\hspace{2cm}}

\begin{enumerate}[leftmargin=*]
\item Отряд решил собрать 12 т металлолома, собрал $\frac{7}{12}$. Сколько тонн собрал?

\vspace{1.2cm}

\item На базу доставили 22 собаки. $\frac{9}{11}$ в упряжке. Сколько не вошло?

\vspace{1.2cm}

\item Купили 5 кг 600 г сахара, израсходовали $\frac{5}{7}$. Сколько г осталось?

\vspace{1.2cm}

\item Из сливок получили 18 кг масла, это $\frac{3}{5}$ массы сливок. Сколько кг сливок?

\vspace{1.2cm}
\end{enumerate}

\end{minipage}%
\hfill
\begin{minipage}[t][0.48\textheight][t]{0.48\textwidth}

\begin{enumerate}[leftmargin=*,start=5]
\item 27 мастеров, $\frac{2}{3}$ --- гончары. Сколько художников?

\vspace{1.2cm}

\item $\frac{4}{9}$ класса --- девочки, мальчиков 15. Сколько всего учащихся?

\vspace{1.2cm}

\item Овощи: $\frac{2}{7}$ --- помидоры, $\frac{3}{7}$ --- огурцы. Огурцов 105 кг. Сколько кг помидоров?

\vspace{1.2cm}

\item Прошли 75 км: $\frac{4}{15}$ в 1-й день, $\frac{7}{15}$ во 2-й. Сколько км за два дня?

\vspace{1.2cm}
\end{enumerate}

\vspace{0.5cm}

\textbf{Ответы:} \underline{\hspace{10cm}}

\end{minipage}

\cutline

% ==================== ПРОВЕРОЧНАЯ РАБОТА - ВАРИАНТ 2 ====================
\begin{minipage}[t][0.48\textheight][t]{0.48\textwidth}
\section*{Проверочная работа. Вариант 2}

\textbf{Фамилия, Имя:} \underline{\hspace{5cm}}

\textbf{Класс:} \underline{\hspace{2cm}} \textbf{Дата:} \underline{\hspace{2cm}}

\begin{enumerate}[leftmargin=*]
\item Длина дороги 36 км. Заасфальтировали $\frac{5}{9}$. Сколько км осталось?

\vspace{1.2cm}

\item Команда набрала 112 очков. Лучший игрок $\frac{1}{4}$ очков. Сколько очков остальные?

\vspace{1.2cm}

\item В книге 87 страниц. Прочитал $\frac{2}{3}$. Сколько страниц осталось?

\vspace{1.2cm}

\item Съели $\frac{5}{8}$ пирожков, доели 12. Сколько испекла бабушка?

\vspace{1.2cm}
\end{enumerate}

\end{minipage}%
\hfill
\begin{minipage}[t][0.48\textheight][t]{0.48\textwidth}

\begin{enumerate}[leftmargin=*,start=5]
\item 80 пятиклассников, $\frac{3}{5}$ в музей. Сколько в театр?

\vspace{1.2cm}

\item 12 мальчиков, это $\frac{3}{7}$ класса. Сколько девочек?

\vspace{1.2cm}

\item Овощи: $\frac{3}{7}$ --- помидоры, $\frac{2}{7}$ --- огурцы. Огурцов 84 кг. Сколько кг помидоров?

\vspace{1.2cm}

\item В зале 90 мест. Продано $\frac{2}{3}$ билетов. Сколько ещё можно продать?

\vspace{1.2cm}
\end{enumerate}

\vspace{0.5cm}

\textbf{Ответы:} \underline{\hspace{10cm}}

\end{minipage}

\newpage

% ==================== ОТВЕТЫ К ПРОВЕРОЧНОЙ ====================
\begin{minipage}[t][0.48\textheight][t]{0.48\textwidth}
\section*{Ответы к проверочной работе}

\subsection*{Вариант 1}

\begin{enumerate}[leftmargin=*]
\item 7 тонн

\textit{Решение:} $12 \cdot \frac{7}{12} = 7$ т

\item 2 собаки

\textit{Решение:} $22 - 22 \cdot \frac{9}{11} = 22 - 18 = 2$

\item 1600 г

\textit{Решение:} $5600 - 5600 \cdot \frac{5}{7} = 5600 - 4000 = 1600$ г

\item 30 кг

\textit{Решение:} $18 : \frac{3}{5} = 18 \cdot \frac{5}{3} = 30$ кг

\item 9 художников

\textit{Решение:} $27 - 27 \cdot \frac{2}{3} = 27 - 18 = 9$

\item 27 учащихся

\textit{Решение:} $15 : \frac{5}{9} = 15 \cdot \frac{9}{5} = 27$

\item 70 кг

\textit{Решение:} $105 : \frac{3}{7} \cdot \frac{2}{7} = 245 \cdot \frac{2}{7} = 70$ кг

\item 55 км

\textit{Решение:} $75 \cdot (\frac{4}{15} + \frac{7}{15}) = 75 \cdot \frac{11}{15} = 55$ км
\end{enumerate}

\end{minipage}%
\hfill
\begin{minipage}[t][0.48\textheight][t]{0.48\textwidth}

\subsection*{Вариант 2}

\begin{enumerate}[leftmargin=*]
\item 16 км

\textit{Решение:} $36 - 36 \cdot \frac{5}{9} = 36 - 20 = 16$ км

\item 84 очка

\textit{Решение:} $112 - 112 \cdot \frac{1}{4} = 112 - 28 = 84$

\item 29 страниц

\textit{Решение:} $87 - 87 \cdot \frac{2}{3} = 87 - 58 = 29$

\item 32 пирожка

\textit{Решение:} $12 : \frac{3}{8} = 12 \cdot \frac{8}{3} = 32$

\item 32 пятиклассника

\textit{Решение:} $80 - 80 \cdot \frac{3}{5} = 80 - 48 = 32$

\item 16 девочек

\textit{Решение:} $12 : \frac{3}{7} - 12 = 28 - 12 = 16$

\item 126 кг

\textit{Решение:} $84 : \frac{2}{7} \cdot \frac{3}{7} = 294 \cdot \frac{3}{7} = 126$ кг

\item 30 билетов

\textit{Решение:} $90 - 90 \cdot \frac{2}{3} = 90 - 60 = 30$
\end{enumerate}

\end{minipage}

\cutline

% ==================== РЕЗЕРВНЫЕ ЗАДАЧИ ====================
\begin{minipage}[t][0.48\textheight][t]{0.48\textwidth}
\section*{Дополнительные задачи}

\textbf{Для тех, кто быстро справился}

\begin{enumerate}[leftmargin=*]
\item Десятую часть миллиона уменьшили на 10\,000 и результат уменьшили в 1000 раз. Сколько получили?

\vspace{1cm}

\item Банка вмещает $\frac{3}{4}$ кг мёда. Сколько банок нужно для $\frac{15}{2}$ кг мёда?

\vspace{1cm}

\item Когда прочитали 35 страниц, осталось $\frac{2}{7}$ книги. Сколько страниц в книге?

\vspace{1cm}

\item Мама израсходовала половину денег и $\frac{1}{3}$ остатка. Осталось 6000 руб. Сколько было?

\vspace{1cm}
\end{enumerate}

\end{minipage}%
\hfill
\begin{minipage}[t][0.48\textheight][t]{0.48\textwidth}

\begin{enumerate}[leftmargin=*,start=5]
\item Сыну 8 лет, его возраст $\frac{2}{9}$ возраста отца. Возраст отца $\frac{3}{5}$ возраста дедушки. Сколько лет дедушке?

\vspace{1cm}

\item Уменьшите 90 руб. на $\frac{1}{10}$ этой суммы.

\vspace{1cm}

\item Увеличьте 80 рублей на $\frac{2}{5}$ этой суммы.

\vspace{1cm}
\end{enumerate}

\vspace{1cm}

\textbf{Ответы:} 1) 90 \quad 2) 10 банок \quad 3) 49 страниц\\
4) 18\,000 руб. \quad 5) 60 лет \quad 6) 81 руб. \quad 7) 112 руб.

\end{minipage}

\end{document}