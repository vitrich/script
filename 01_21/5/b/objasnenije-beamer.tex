\documentclass[aspectratio=169,12pt]{beamer}
\usepackage[utf8]{inputenc}
\usepackage[russian]{babel}
\usepackage{amsmath}
\usepackage{tikz}

\usetheme{Madrid}
\usecolortheme{default}

\title{Нахождение части числа и числа по его части}
\subtitle{Математика, 5 класс}
\author{}
\date{}

\begin{document}

\frame{\titlepage}

\begin{frame}
\frametitle{Содержание}
\tableofcontents
\end{frame}

\section{Введение}

\begin{frame}
\frametitle{Три основных типа задач}

В этой теме мы изучаем три типа задач:

\begin{enumerate}
\item \textbf{Нахождение части от числа}\\
{\small Сколько составляет $\frac{2}{3}$ от 60?}

\item \textbf{Нахождение числа по его части}\\
{\small Найти число, если $\frac{3}{4}$ от него равны 12}

\item \textbf{Нахождение какую часть составляет число от другого}\\
{\small Какую часть от 120 составляет число 30?}
\end{enumerate}

\end{frame}

\section{Тип 1: Нахождение части от числа}

\begin{frame}
\frametitle{Тип 1: Нахождение части от числа}

\begin{block}{Правило}
Чтобы найти \textbf{часть от числа}, нужно это число \textbf{умножить} на дробь.
\end{block}

\vspace{0.5cm}

\begin{exampleblock}{Формула}
$$\text{Часть} = \text{Целое} \times \text{Дробь}$$
\end{exampleblock}

\end{frame}

\begin{frame}
\frametitle{Пример 1}

\textbf{Задача:} В автобусе 51 место для пассажиров. Две трети этих мест уже заняты. Сколько мест занято?

\pause

\vspace{0.5cm}

\textbf{Решение:}

\begin{itemize}
\item Всего мест: 51
\item Занято: $\frac{2}{3}$ от всех мест
\item Нужно найти: $51 \times \frac{2}{3}$
\end{itemize}

\pause

$$51 \times \frac{2}{3} = \frac{51 \times 2}{3} = \frac{102}{3} = 34 \text{ (места)}$$

\pause

\textbf{Ответ:} 34 места занято

\end{frame}

\begin{frame}
\frametitle{Пример 2}

\textbf{Задача:} Длина дороги 20 км. Заасфальтировали $\frac{3}{4}$ дороги. Сколько километров заасфальтировали?

\pause

\vspace{0.5cm}

\textbf{Решение:}

$$20 \times \frac{3}{4} = \frac{20 \times 3}{4} = \frac{60}{4} = 15 \text{ (км)}$$

\pause

\textbf{Ответ:} 15 км заасфальтировали

\end{frame}

\begin{frame}
\frametitle{Задача с остатком}

\textbf{Задача:} В классе 25 учеников. Из них три пятых --- мальчики. Сколько девочек учится в классе?

\pause

\vspace{0.5cm}

\textbf{Решение:}

\begin{enumerate}
\item Найдём количество мальчиков: $25 \times \frac{3}{5} = \frac{75}{5} = 15$ (мальчиков)
\pause
\item Найдём количество девочек: $25 - 15 = 10$ (девочек)
\end{enumerate}

\pause

\textbf{Ответ:} 10 девочек

\end{frame}

\section{Тип 2: Нахождение числа по его части}

\begin{frame}
\frametitle{Тип 2: Нахождение числа по его части}

\begin{block}{Правило}
Чтобы найти \textbf{число по его части}, нужно известную величину \textbf{разделить} на дробь (или умножить на обратную дробь).
\end{block}

\vspace{0.5cm}

\begin{exampleblock}{Формула}
$$\text{Целое} = \text{Часть} : \text{Дробь} = \text{Часть} \times \frac{1}{\text{Дробь}}$$
\end{exampleblock}

\end{frame}

\begin{frame}
\frametitle{Пример 3}

\textbf{Задача:} Вася загадал число. Известно, что число 12 составляет $\frac{3}{4}$ от загаданного Васей числа. Какое число загадал Вася?

\pause

\vspace{0.5cm}

\textbf{Решение:}

\begin{itemize}
\item Известная часть: 12
\item Это составляет: $\frac{3}{4}$ от целого
\item Нужно найти: $12 : \frac{3}{4}$
\end{itemize}

\pause

$$12 : \frac{3}{4} = 12 \times \frac{4}{3} = \frac{12 \times 4}{3} = \frac{48}{3} = 16$$

\pause

\textbf{Ответ:} Вася загадал число 16

\end{frame}

\begin{frame}
\frametitle{Пример 4}

\textbf{Задача:} До обеда выгрузили $\frac{7}{15}$ зерна, находившегося в товарном вагоне. Выгрузили 42 т. Сколько тонн зерна было в вагоне?

\pause

\vspace{0.5cm}

\textbf{Решение:}

$$42 : \frac{7}{15} = 42 \times \frac{15}{7} = \frac{42 \times 15}{7} = \frac{630}{7} = 90 \text{ (т)}$$

\pause

\textbf{Ответ:} 90 тонн зерна было в вагоне

\end{frame}

\begin{frame}
\frametitle{Пример 5}

\textbf{Задача:} В баке осталось ровно 18 л бензина, при этом бак заполнен на четверть. Сколько всего литров бензина помещается в бак?

\pause

\vspace{0.5cm}

\textbf{Решение:}

\begin{itemize}
\item В баке сейчас: 18 л
\item Это составляет: $\frac{1}{4}$ от всего объёма
\end{itemize}

\pause

$$18 : \frac{1}{4} = 18 \times 4 = 72 \text{ (л)}$$

\pause

\textbf{Ответ:} В бак помещается 72 литра бензина

\end{frame}

\section{Тип 3: Какую часть составляет одно число от другого}

\begin{frame}
\frametitle{Тип 3: Какую часть составляет одно число от другого}

\begin{block}{Правило}
Чтобы узнать, \textbf{какую часть} одно число составляет от другого, нужно первое число \textbf{разделить} на второе.
\end{block}

\vspace{0.5cm}

\begin{exampleblock}{Формула}
$$\text{Часть} = \frac{\text{Первое число}}{\text{Второе число}}$$
\end{exampleblock}

\end{frame}

\begin{frame}
\frametitle{Пример 6}

\textbf{Задача:} В гараже 30 зелёных машин, всего машин --- 120. Какую часть составляют зелёные машины?

\pause

\vspace{0.5cm}

\textbf{Решение:}

$$\frac{30}{120} = \frac{1}{4} = 0{,}25$$

\pause

\textbf{Ответ:} Зелёные машины составляют $\frac{1}{4}$ или 0,25 от всех машин

\end{frame}

\begin{frame}
\frametitle{Пример 7}

\textbf{Задача:} Продолжительность урока 45 минут. На решение задачи ушло 9 мин. Какая часть урока ушла на решение задачи?

\pause

\vspace{0.5cm}

\textbf{Решение:}

$$\frac{9}{45} = \frac{1}{5} = 0{,}2$$

\pause

\textbf{Ответ:} На решение задачи ушло $\frac{1}{5}$ или 0,2 урока

\end{frame}

\begin{frame}
\frametitle{Пример 8}

\textbf{Задача:} Около дома стояло 8 машин. Из них 2 были серыми, а остальные синими. Какую часть всех машин составляли синие машины?

\pause

\vspace{0.5cm}

\textbf{Решение:}

\begin{enumerate}
\item Найдём количество синих машин: $8 - 2 = 6$ (машин)
\pause
\item Найдём какую часть составляют синие машины: $\frac{6}{8} = \frac{3}{4} = 0{,}75$
\end{enumerate}

\pause

\textbf{Ответ:} Синие машины составляли 0,75 от всех машин

\end{frame}

\section{Комбинированные задачи}

\begin{frame}
\frametitle{Комбинированная задача}

\textbf{Задача:} Турист прошёл за первый день 18 км, что составляет $\frac{2}{3}$ пути, который он должен пройти во второй день. Сколько километров должен пройти турист за оба дня вместе?

\pause

\vspace{0.3cm}

\textbf{Решение:}

\begin{enumerate}
\item Найдём путь второго дня: $18 : \frac{2}{3} = 18 \times \frac{3}{2} = 27$ (км)
\pause
\item Найдём весь путь: $18 + 27 = 45$ (км)
\end{enumerate}

\pause

\textbf{Ответ:} 45 км

\end{frame}

\begin{frame}
\frametitle{Комбинированная задача 2}

\textbf{Задача:} В первый день картофель посадили на $\frac{2}{7}$ участка, а во второй день --- на $\frac{3}{14}$ участка. Какая площадь была засажена картофелем за эти два дня, если площадь участка 14 м$^2$?

\pause

\vspace{0.3cm}

\textbf{Решение:}

\begin{enumerate}
\item Первый день: $14 \times \frac{2}{7} = 4$ (м$^2$)
\pause
\item Второй день: $14 \times \frac{3}{14} = 3$ (м$^2$)
\pause
\item Всего: $4 + 3 = 7$ (м$^2$)
\end{enumerate}

\pause

\textbf{Ответ:} 7 м$^2$

\end{frame}

\section{Алгоритм решения}

\begin{frame}
\frametitle{Как определить тип задачи?}

\begin{enumerate}
\item \textbf{Нахождение части от числа}
\begin{itemize}
\item Известно: целое число и дробь
\item Найти: часть от этого числа
\item Действие: умножение
\end{itemize}

\vspace{0.3cm}

\item \textbf{Нахождение числа по его части}
\begin{itemize}
\item Известно: часть числа и дробь (какую часть она составляет)
\item Найти: целое число
\item Действие: деление на дробь
\end{itemize}

\vspace{0.3cm}

\item \textbf{Какую часть составляет одно число от другого}
\begin{itemize}
\item Известно: два числа
\item Найти: какую часть одно составляет от другого
\item Действие: деление одного числа на другое
\end{itemize}
\end{enumerate}

\end{frame}

\begin{frame}
\frametitle{Практические советы}

\begin{block}{Совет 1}
Внимательно читайте условие задачи и определяйте, что дано и что нужно найти
\end{block}

\begin{block}{Совет 2}
Помните обратные дроби: $\frac{a}{b} \times \frac{b}{a} = 1$
\end{block}

\begin{block}{Совет 3}
Делить на дробь --- то же самое, что умножить на обратную дробь
\end{block}

\begin{block}{Совет 4}
Проверяйте свои ответы: результат должен быть логичным
\end{block}

\end{frame}

\begin{frame}
\frametitle{Закрепление}

\begin{center}
\Large
Спасибо за внимание!\\[1cm]

Теперь переходим к решению задач
\end{center}

\end{frame}

\end{document}