\documentclass[aspectratio=169,10pt]{beamer}

\usepackage[utf8]{inputenc}
\usepackage[russian]{babel}
\usepackage{amsmath}
\usepackage{tikz}

\usetheme{Madrid}
\usecolortheme{default}

\title{Нахождение части числа и числа по его части}
\subtitle{Математика, 5 класс}
\author{}
\date{}

\begin{document}
	
	\frame{\titlepage}
	
	\begin{frame}
		\frametitle{Содержание}
		\tableofcontents
	\end{frame}
	
	\section{Введение}
	
	\begin{frame}
		\frametitle{Три основных типа задач}
		
		В этой теме мы изучаем три типа задач:
		
		\begin{enumerate}
			\item \textbf{Нахождение части от числа}\\
			{\small Сколько составляет $\frac{2}{3}$ от 60?}
			
			\item \textbf{Нахождение числа по его части}\\
			{\small Найти число, если $\frac{3}{4}$ от него равны 12}
			
			\item \textbf{Нахождение, какую часть составляет число от другого}\\
			{\small Какую часть от 120 составляет число 30?}
		\end{enumerate}
		
	\end{frame}
	
	\section{Тип 1: Нахождение части от числа}
	
	\begin{frame}
		\frametitle{Тип 1: Нахождение части от числа}
		
		\begin{block}{Идея}
			Дробь показывает, на сколько равных частей разделено целое (знаменатель) и сколько таких частей берут (числитель).
		\end{block}
		
		\vspace{0.5cm}
		
		\begin{block}{Правило}
			Чтобы найти \textbf{часть от числа}, нужно:
			\begin{itemize}
				\item разделить число на знаменатель дроби (найти одну часть);
				\item умножить результат на числитель (взять нужное количество частей).
			\end{itemize}
		\end{block}
		
	\end{frame}
	
	\begin{frame}
		\frametitle{Пример 1}
		
		\textbf{Задача:} В автобусе 51 место для пассажиров. Две трети этих мест уже заняты. Сколько мест занято?
		
		\pause
		\vspace{0.5cm}
		
		\textbf{Решение:}
		
		\begin{itemize}
			\item Всего мест: 51
			\item Две трети — это 2 части из 3 равных
		\end{itemize}
		
		\pause
		
		Сначала найдём одну треть всех мест:
		\[
		51 : 3 = 17 \text{ (мест)}
		\]
		
		\pause
		
		Две трети — это две такие части:
		\[
		17 \cdot 2 = 34 \text{ (места)}
		\]
		
		\pause
		
		\textbf{Ответ:} 34 места занято
		
	\end{frame}
	
	\begin{frame}
		\frametitle{Пример 2}
		
		\textbf{Задача:} Длина дороги 20 км. Заасфальтировали $\frac{3}{4}$ дороги. Сколько километров заасфальтировали?
		
		\pause
		\vspace{0.5cm}
		
		\textbf{Решение:}
		
		\begin{itemize}
			\item Вся дорога: 20 км
			\item Три четверти — это 3 части из 4 равных
		\end{itemize}
		
		\pause
		
		Сначала найдём одну четверть дороги:
		\[
		20 : 4 = 5 \text{ км}
		\]
		
		\pause
		
		Три четверти — это три такие части:
		\[
		5 \cdot 3 = 15 \text{ км}
		\]
		
		\pause
		
		\textbf{Ответ:} 15 км заасфальтировали
		
	\end{frame}
	
	\begin{frame}
		\frametitle{Задача с остатком}
		
		\textbf{Задача:} В классе 25 учеников. Из них три пятых --- мальчики. Сколько девочек учится в классе?
		
		\pause
		\vspace{0.5cm}
		
		\textbf{Решение:}
		
		\begin{enumerate}
			\item Одна пятая класса:
			\[
			25 : 5 = 5 \text{ учеников}
			\]
			
			\pause
			
			\item Три пятых — это три такие части:
			\[
			5 \cdot 3 = 15 \text{ мальчиков}
			\]
			
			\pause
			
			\item Девочки:
			\[
			25 - 15 = 10 \text{ девочек}
			\]
		\end{enumerate}
		
		\pause
		
		\textbf{Ответ:} 10 девочек
		
	\end{frame}
	
	\section{Тип 2: Нахождение числа по его части}
	
	\begin{frame}
		\frametitle{Тип 2: Нахождение числа по его части}
		
		\begin{block}{Идея}
			Известно, что некоторая дробь от числа равна данной величине. Сначала находим одну часть, затем всё целое.
		\end{block}
		
		\vspace{0.5cm}
		
		\begin{block}{Правило}
			Чтобы найти \textbf{число по его части}, нужно:
			\begin{itemize}
				\item разделить известную часть на числитель дроби (найти одну долю);
				\item умножить полученное число на знаменатель (восстановить целое).
			\end{itemize}
		\end{block}
		
	\end{frame}
	
	\begin{frame}
		\frametitle{Пример 3}
		
		\textbf{Задача:} Вася загадал число. Известно, что число 12 составляет $\frac{3}{4}$ от загаданного Васей числа. Какое число загадал Вася?
		
		\pause
		\vspace{0.5cm}
		
		\textbf{Решение:}
		
		\begin{itemize}
			\item Известная часть: 12
			\item Это три части из четырёх (три четверти)
		\end{itemize}
		
		\pause
		
		Сначала найдём одну четверть:
		\[
		12 : 3 = 4
		\]
		
		\pause
		
		Теперь найдём всё число (четыре четверти):
		\[
		4 \cdot 4 = 16
		\]
		
		\pause
		
		\textbf{Ответ:} Вася загадал число 16
		
	\end{frame}
	
	\begin{frame}
		\frametitle{Пример 4}
		
		\textbf{Задача:} До обеда выгрузили $\frac{7}{15}$ зерна, находившегося в товарном вагоне. Выгрузили 42 т. Сколько тонн зерна было в вагоне?
		
		\pause
		\vspace{0.5cm}
		
		\textbf{Решение:}
		
		\begin{itemize}
			\item 42 т — это 7 частей из 15 (семь пятнадцатых)
		\end{itemize}
		
		\pause
		
		Сначала найдём одну пятнадцатую:
		\[
		42 : 7 = 6 \text{ т}
		\]
		
		\pause
		
		Теперь найдём все 15 частей:
		\[
		6 \cdot 15 = 90 \text{ т}
		\]
		
		\pause
		
		\textbf{Ответ:} 90 тонн зерна было в вагоне
		
	\end{frame}
	
	\begin{frame}
		\frametitle{Пример 5}
		
		\textbf{Задача:} В баке осталось ровно 18 л бензина, при этом бак заполнен на четверть. Сколько всего литров бензина помещается в бак?
		
		\pause
		\vspace{0.5cm}
		
		\textbf{Решение:}
		
		\begin{itemize}
			\item 18 л — это одна четверть объёма бака
		\end{itemize}
		
		\pause
		
		Чтобы найти весь объём (четыре четверти), делаем так:
		\[
		18 \cdot 4 = 72 \text{ л}
		\]
		
		\pause
		
		\textbf{Ответ:} В бак помещается 72 литра бензина
		
	\end{frame}
	
	\section{Тип 3: Какую часть составляет одно число от другого}
	
	\begin{frame}
		\frametitle{Тип 3: Какую часть составляет одно число от другого}
		
		\begin{block}{Правило}
			Чтобы узнать, \textbf{какую часть} одно число составляет от другого, нужно первое число разделить на второе и записать результат в виде дроби или десятичной дроби.
		\end{block}
		
		\vspace{0.5cm}
		
		\begin{exampleblock}{Запись}
			\[
			\text{Часть} = \frac{\text{Первое число}}{\text{Второе число}}
			\]
		\end{exampleblock}
		
	\end{frame}
	
	\begin{frame}
		\frametitle{Пример 6}
		
		\textbf{Задача:} В гараже 30 зелёных машин, всего машин --- 120. Какую часть составляют зелёные машины?
		
		\pause
		\vspace{0.5cm}
		
		\textbf{Решение:}
		
		\[
		\frac{30}{120}
		\]
		
		\pause
		
		Сократим дробь, деля числитель и знаменатель на 30:
		\[
		\frac{30 : 30}{120 : 30} = \frac{1}{4} = 0{,}25
		\]
		
		\pause
		
		\textbf{Ответ:} Зелёные машины составляют $\frac{1}{4}$ или 0,25 от всех машин
		
	\end{frame}
	
	\begin{frame}
		\frametitle{Пример 7}
		
		\textbf{Задача:} Продолжительность урока 45 минут. На решение задачи ушло 9 мин. Какая часть урока ушла на решение задачи?
		
		\pause
		\vspace{0.5cm}
		
		\textbf{Решение:}
		
		\[
		\frac{9}{45}
		\]
		
		\pause
		
		Делим числитель и знаменатель на 9:
		\[
		\frac{9 : 9}{45 : 9} = \frac{1}{5} = 0{,}2
		\]
		
		\pause
		
		\textbf{Ответ:} На решение задачи ушло $\frac{1}{5}$ или 0,2 урока
		
	\end{frame}
	
	\begin{frame}
		\frametitle{Пример 8}
		
		\textbf{Задача:} Около дома стояло 8 машин. Из них 2 были серыми, а остальные синими. Какую часть всех машин составляли синие машины?
		
		\pause
		\vspace{0.5cm}
		
		\textbf{Решение:}
		
		\begin{enumerate}
			\item Найдём количество синих машин:
			\[
			8 - 2 = 6 \text{ (машин)}
			\]
			
			\pause
			
			\item Найдём, какую часть составляют синие машины:
			\[
			\frac{6}{8}
			\]
			Делим числитель и знаменатель на 2:
			\[
			\frac{6 : 2}{8 : 2} = \frac{3}{4} = 0{,}75
			\]
		\end{enumerate}
		
		\pause
		
		\textbf{Ответ:} Синие машины составляли 0,75 от всех машин
		
	\end{frame}
	
	\section{Комбинированные задачи}
	
	\begin{frame}
		\frametitle{Комбинированная задача}
		
		\textbf{Задача:} Турист прошёл за первый день 18 км, что составляет $\frac{2}{3}$ пути, который он должен пройти во второй день. Сколько километров должен пройти турист за оба дня вместе?
		
		\pause
		\vspace{0.3cm}
		
		\textbf{Решение:}
		
		\begin{enumerate}
			\item 18 км — это две части из трёх пути второго дня.
			
			\pause
			
			Сначала найдём одну такую часть:
			\[
			18 : 2 = 9 \text{ км}
			\]
			
			\pause
			
			Тогда путь второго дня — три части:
			\[
			9 \cdot 3 = 27 \text{ км}
			\]
			
			\pause
			
			\item Весь путь за два дня:
			\[
			18 + 27 = 45 \text{ км}
			\]
		\end{enumerate}
		
		\pause
		
		\textbf{Ответ:} 45 км
		
	\end{frame}
	
	\begin{frame}
		\frametitle{Комбинированная задача 2}
		
		\textbf{Задача:} В первый день картофель посадили на $\frac{2}{7}$ участка, а во второй день --- на $\frac{3}{14}$ участка. Какая площадь была засажена картофелем за эти два дня, если площадь участка 14 м$^2$?
		
		\pause
		\vspace{0.3cm}
		
		\textbf{Решение:}
		
		\begin{enumerate}
			\item Первый день: две части из семи.
			
			Сначала найдём одну седьмую:
			\[
			14 : 7 = 2 \text{ м}^2
			\]
			
			Две седьмых:
			\[
			2 \cdot 2 = 4 \text{ м}^2
			\]
			
			\pause
			
			\item Второй день: три части из четырнадцати.
			
			Одна четырнадцатая:
			\[
			14 : 14 = 1 \text{ м}^2
			\]
			
			Три четырнадцатых:
			\[
			1 \cdot 3 = 3 \text{ м}^2
			\]
			
			\pause
			
			\item Всего засажено:
			\[
			4 + 3 = 7 \text{ м}^2
			\]
		\end{enumerate}
		
		\pause
		
		\textbf{Ответ:} 7 м$^2$
		
	\end{frame}
	
	\section{Алгоритм решения}
	
	\begin{frame}
		\frametitle{Как определить тип задачи?}
		
		\begin{enumerate}
			\item \textbf{Нахождение части от числа}
			\begin{itemize}
				\item Известно: целое число и дробь
				\item Найти: часть от этого числа
				\item Действия: разделить целое на знаменатель дроби, затем умножить на числитель
			\end{itemize}
			
			\vspace{0.3cm}
			
			\item \textbf{Нахождение числа по его части}
			\begin{itemize}
				\item Известно: часть числа и дробь (какую долю она составляет)
				\item Найти: целое число
				\item Действия: разделить часть на числитель дроби, затем умножить на знаменатель
			\end{itemize}
			
			\vspace{0.3cm}
			
			\item \textbf{Какую часть составляет одно число от другого}
			\begin{itemize}
				\item Известно: два числа
				\item Найти: какую часть одно составляет от другого
				\item Действия: разделить одно число на другое и записать результат в виде дроби или десятичной дроби
			\end{itemize}
		\end{enumerate}
		
	\end{frame}
	
	\begin{frame}
		\frametitle{Практические советы}
		
		\begin{block}{Совет 1}
			Внимательно читайте условие задачи и определяйте, что дано и что нужно найти.
		\end{block}
		
		\begin{block}{Совет 2}
			Помните смысл дроби: знаменатель показывает, на сколько равных частей разделили, числитель — сколько таких частей взяли.
		\end{block}
		
		\begin{block}{Совет 3}
			Когда видите дробь от числа, сначала ищите одну часть (делением), потом нужное количество частей (умножением на числитель).
		\end{block}
		
		\begin{block}{Совет 4}
			Проверяйте свои ответы: результат должен быть логичным.
		\end{block}
		
	\end{frame}
	
	\begin{frame}
		\frametitle{Закрепление}
		
		\begin{center}
			\Large
			Спасибо за внимание!\\[1cm]
			Теперь переходим к решению задач
		\end{center}
		
	\end{frame}
	
\end{document}
