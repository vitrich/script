\documentclass[12pt,a4paper]{article}
\usepackage[utf8]{inputenc}
\usepackage[russian]{babel}
\usepackage{amsmath}
\usepackage{geometry}
\usepackage{enumitem}
\geometry{left=2cm,right=2cm,top=2cm,bottom=2cm}

\title{Задачи по теме «Нахождение части числа и числа по его части»}
\author{5 класс}
\date{}

\begin{document}

\maketitle

\section{Объяснение}

\subsection{Типы задач}

При решении задач на части числа и числа по его части встречаются три основных типа:

\textbf{Тип 1: Нахождение части от числа}

Чтобы найти часть от числа, нужно это число умножить на дробь.

\textit{Пример 1:} В автобусе 51 место для пассажиров. Две трети этих мест уже заняты. Сколько мест занято?

\textit{Решение:} $51 \cdot \frac{2}{3} = \frac{51 \cdot 2}{3} = \frac{102}{3} = 34$ (места)

\textit{Пример 2:} Длина дороги 20 км. Заасфальтировали $\frac{3}{4}$ дороги. Сколько километров заасфальтировали?

\textit{Решение:} $20 \cdot \frac{3}{4} = \frac{20 \cdot 3}{4} = \frac{60}{4} = 15$ (км)

\vspace{0.5cm}

\textbf{Тип 2: Нахождение числа по его части}

Чтобы найти число по его части, нужно известную величину разделить на дробь (или умножить на обратную дробь).

\textit{Пример 3:} Вася загадал число. Известно, что число 12 составляет $\frac{3}{4}$ от загаданного Васей числа. Какое число загадал Вася?

\textit{Решение:} $12 : \frac{3}{4} = 12 \cdot \frac{4}{3} = \frac{12 \cdot 4}{3} = \frac{48}{3} = 16$

\textit{Пример 4:} До обеда выгрузили $\frac{7}{15}$ зерна, находившегося в товарном вагоне. Выгрузили 42 т. Сколько тонн зерна было в вагоне?

\textit{Решение:} $42 : \frac{7}{15} = 42 \cdot \frac{15}{7} = \frac{42 \cdot 15}{7} = \frac{630}{7} = 90$ (т)

\vspace{0.5cm}

\textbf{Тип 3: Нахождение какую часть одно число составляет от другого}

Чтобы узнать, какую часть одно число составляет от другого, нужно первое число разделить на второе.

\textit{Пример 5:} В гараже 30 зелёных машин, всего машин --- 120. Какую часть составляют зелёные машины?

\textit{Решение:} $30 : 120 = \frac{30}{120} = \frac{1}{4} = 0{,}25$

\textit{Пример 6:} Продолжительность урока 45 минут. На решение задачи ушло 9 мин. Какая часть урока ушла на решение задачи?

\textit{Решение:} $9 : 45 = \frac{9}{45} = \frac{1}{5} = 0{,}2$

\vspace{1cm}

\section{Домашняя работа}

\subsection{Задачи на нахождение части от числа}

\begin{enumerate}[label=\arabic*.]
\item В автобусе 51 место для пассажиров. Две трети этих мест уже заняты. Сколько еще пассажиров может сесть в автобус на оставшиеся места?

\item От дыни массой 2 кг 400 г Ване отрезали $\frac{1}{3}$ дыни, а Маше $\frac{1}{4}$ дыни. Сколько граммов дыни осталось?

\item Петя готовил уроки 1 ч 40 мин. На математику он потратил $\frac{3}{5}$ этого времени, а оставшееся время потратил на географию. Сколько минут Петя готовил географию?

\item На огороде собрали 42 кг огурцов и $\frac{2}{3}$ всех огурцов засолили. Сколько килограммов огурцов остались свежими?

\item Мастерская получила 700 м шёлка. Из $\frac{2}{7}$ полученной ткани сшили халаты, а из $\frac{3}{4}$ полученной ткани сшили платья. Сколько метров шёлка осталось?

\item У ученика было 50 к. На завтрак он истратил $\frac{3}{5}$ этих денег. Сколько копеек у него осталось?

\item В классе 25 учеников. Из них три пятых --- мальчики. Сколько девочек учится в классе?

\item В классе 30 учеников, из них две пятых --- девочки. Сколько мальчиков учится в классе?
\end{enumerate}

\subsection{Задачи на нахождение числа по его части}

\begin{enumerate}[resume,label=\arabic*.]
\item Вася загадал число. Известно, что число 12 составляет $\frac{3}{4}$ от загаданного Васей числа. Какое число загадал Вася?

\item На приобретение костюма покупатель израсходовал $\frac{3}{5}$ своих денег. Сколько рублей было у покупателя, если костюм стоил 120 р?

\item До обеда выгрузили $\frac{7}{15}$ зерна, находившегося в товарном вагоне. Сколько тонн зерна было в вагоне, если выгрузили 42 т?

\item Три пятых всех учащихся класса составляют девочки. Сколько всего учащихся в этом классе, если в этом классе 10 мальчиков?

\item Две пятых всех учащихся класса составляют девочки. Сколько всего учащихся в этом классе, если в этом классе 18 мальчиков?

\item В пятом классе 12 девочек, что составляет две пятых учащихся класса. Сколько мальчиков в этом классе?

\item В пятом классе 15 девочек, что составляет три пятых учащихся класса. Сколько мальчиков в этом классе?

\item В баке осталось ровно 18 л бензина, при этом бак заполнен на четверть. Сколько всего литров бензина помещается в бак?
\end{enumerate}

\subsection{Задачи на нахождение какую часть составляет число от другого}

\begin{enumerate}[resume,label=\arabic*.]
\item В гараже 30 зелёных машин, всего машин --- 120. Какую часть составляют зелёные машины? Ответ выразите десятичной дробью.

\item Иван Владимирович работает на предприятии. В апреле он не работал 15 дней. Какую часть апреля работал Иван Владимирович? Ответ выразите десятичной дробью.

\item Продолжительность урока 45 минут. На решение задачи ушло 9 мин. Какая часть урока ушла на решение задачи? Ответ выразите десятичной дробью.

\item Около дома стояло 8 машин. Из них 2 были серыми, а остальные синими. Какую часть всех машин составляли синие машины? Ответ выразите десятичной дробью.

\item В классе 40 человек. Из них 10 человек ещё не сдали нормы ГТО. Какая часть учащихся сдала нормы ГТО? Ответ выразите десятичной дробью.
\end{enumerate}

\subsection{Комбинированные задачи}

\begin{enumerate}[resume,label=\arabic*.]
\item Турист прошёл за первый день 18 км, что составляет $\frac{2}{3}$ пути, который он должен пройти во второй день. Сколько километров должен пройти турист за оба дня вместе?

\item В первый день картофель посадили на $\frac{2}{7}$ участка, а во второй день --- на $\frac{3}{14}$ участка. Какая площадь (в м$^2$) была засажена картофелем за эти два дня, если площадь участка 14 м$^2$?

\item Для посадки леса выделили участок площадью 300 га. Ели высадили на $\frac{8}{15}$ участка, а сосну --- на $\frac{4}{15}$ участка. Сколько гектаров занято елью и сосной вместе?

\item В первый день турист прошёл $\frac{5}{12}$ всего пути, а во второй день $\frac{1}{3}$ всего пути. Известно, что за эти два дня турист прошёл 36 км. Сколько всего километров составляет путь туриста?
\end{enumerate}

\newpage

\section{Проверочная работа}

\subsection{Вариант 1}

\begin{enumerate}[label=\arabic*.]
\item Отряд решил собрать 12 т металлолома, а собрал $\frac{7}{12}$ этого количества. Сколько тонн металлолома собрал отряд?

\item На базу в Антарктиду доставили 22 собаки. Из $\frac{9}{11}$ всех собак составили упряжку, на которой отправились в поход. Сколько собак не вошло в упряжку?

\item Купили 5 кг 600 г сахара и израсходовали на варенье $\frac{5}{7}$ всего сахара. Сколько граммов сахара осталось?

\item Из сливок получили 18 кг масла, что составляет $\frac{3}{5}$ массы сливок. Сколько кг сливок было взято?

\item В художественной мастерской работает 27 мастеров. Из них две трети --- гончары, а остальные --- художники. Сколько художников работает в мастерской?

\item Четыре девятых всех учащихся класса составляют девочки. Сколько всего учащихся в этом классе, если в этом классе 15 мальчиков?

\item В магазин завезли овощи. Две седьмых всех овощей --- помидоры, а три седьмых всех овощей --- огурцы. Сколько килограммов помидоров завезли в магазин, если огурцов завезли 105 кг?

\item Пионеры прошли 75 км по местам боевой славы. В первый день они прошли $\frac{4}{15}$ всего расстояния, а во второй $\frac{7}{15}$ всего расстояния. Сколько километров они прошли за эти два дня?
\end{enumerate}

\subsection{Вариант 2}

\begin{enumerate}[label=\arabic*.]
\item Длина дороги 36 км. Заасфальтировали $\frac{5}{9}$ дороги. Сколько километров осталось заасфальтировать?

\item В матче баскетбольная команда набрала 112 очков. Лучший игрок этой команды заработал четверть всех очков. Сколько очков заработали все остальные игроки команды вместе?

\item В книге 87 страниц. Стас уже прочитал две трети всех страниц. Сколько страниц осталось прочитать Стасу?

\item Бабушка напекла пирожков. За завтраком члены семьи съели $\frac{5}{8}$ всех пирожков. В обед доели оставшиеся 12 пирожков. Сколько пирожков испекла бабушка?

\item В школе 80 пятиклассников. Три пятых всех пятиклассников поехали на экскурсию в музей, а остальные пошли в театр. Сколько пятиклассников пошло в театр?

\item В пятом классе 12 мальчиков, что составляет три седьмых учащихся класса. Сколько девочек в этом классе?

\item В магазин завезли овощи. Три седьмых всех овощей --- помидоры, а две седьмых всех овощей --- огурцы. Сколько килограммов помидоров завезли в магазин, если огурцов завезли 84 кг?

\item В кинозале 90 мест. На сеанс уже продано две трети всех билетов. Сколько ещё билетов можно продать на этот сеанс?
\end{enumerate}

\newpage

\section{Ответы}

\subsection{Домашняя работа}

\begin{enumerate}[label=\arabic*.]
\item 17 пассажиров
\item 1000 г (или 1 кг)
\item 40 минут
\item 14 кг
\item 175 м
\item 20 копеек
\item 10 девочек
\item 18 мальчиков
\item 16
\item 200 рублей
\item 90 тонн
\item 25 учащихся
\item 30 учащихся
\item 18 мальчиков
\item 10 мальчиков
\item 72 литра
\item 0,25
\item 0,5
\item 0,2
\item 0,75
\item 0,75
\item 45 км
\item 7 м$^2$
\item 240 га
\item 48 км
\end{enumerate}

\subsection{Проверочная работа. Вариант 1}

\begin{enumerate}[label=\arabic*.]
\item 7 тонн
\item 2 собаки
\item 1600 г
\item 30 кг
\item 9 художников
\item 27 учащихся
\item 70 кг
\item 55 км
\end{enumerate}

\subsection{Проверочная работа. Вариант 2}

\begin{enumerate}[label=\arabic*.]
\item 16 км
\item 84 очка
\item 29 страниц
\item 32 пирожка
\item 32 пятиклассника
\item 16 девочек
\item 126 кг
\item 30 билетов
\end{enumerate}

\end{document}