\documentclass{beamer}
\usepackage[utf8]{inputenc}
\usepackage[russian]{babel}
\usepackage{amsmath}
\usepackage{amssymb}
\usepackage{geometry}
\usepackage{xcolor}
\usepackage{listings}

% Тема и цвета
\usetheme{Madrid}
\usecolortheme{default}
\setbeamertemplate{navigation symbols}{}

% Кастомные цвета
\definecolor{darkblue}{RGB}{15, 23, 42}
\definecolor{lightblue}{RGB}{56, 189, 248}
\definecolor{darkgray}{RGB}{30, 41, 59}
\definecolor{lightgray}{RGB}{203, 213, 225}
\definecolor{mutedgray}{RGB}{148, 163, 184}

% Применяем цвета
\setbeamercolor{background canvas}{bg=darkblue}
\setbeamercolor{frametitle}{bg=darkgray, fg=lightblue}
\setbeamercolor{title}{fg=lightblue}
\setbeamercolor{subtitle}{fg=lightgray}
\setbeamercolor{normal text}{fg=lightgray}
\setbeamercolor{structure}{fg=lightblue}

% Переопределяем шрифты
\setbeamerfont{title}{size=\Large, series=\bfseries}
\setbeamerfont{frametitle}{size=\large, series=\bfseries}
\setbeamerfont{normal text}{size=\normalsize}

% Убираем навигацию снизу
\beamertemplatenavigationsymbolsempty

\title{Шкалы}
\subtitle{Измерение величин и работа со шкалами}
\author{Математика • 4 класс}
\date{Учебник: Л.Г. Петерсон}

\begin{document}

% Титульный слайд
\frame{\titlepage}

% Слайд 1: Определение
\begin{frame}
\frametitle{Что такое шкала?}

Шкала — это линия или поверхность с делениями (отметками) и числами для измерения величин.

\vspace{1cm}

\begin{beamercolorbox}[rounded=true, shadow=true, sep=0.5cm]{structure}
    Шкала используется для измерения:
    \begin{itemize}
        \item Длины и расстояния
        \item Температуры
        \item Времени и скорости
    \end{itemize}
\end{beamercolorbox}

\end{frame}

% Слайд 2: Примеры шкал
\begin{frame}
\frametitle{Примеры шкал}

\vspace{0.5cm}

{\bfseries Линейка:} \\
\texttt{0 — 1 — 2 — 3 — 4 — 5 — 6 — 7 — 8 — 9 — 10 см}

\vspace{0.8cm}

{\bfseries Часы:} \\
\texttt{12, 1, 2, 3, 4, 5, 6, 7, 8, 9, 10, 11}

\vspace{0.8cm}

{\bfseries Градусник:} \\
\texttt{0 — 10 — 20 — 30 — 35 — 36,6 — 40°C}

\end{frame}

% Слайд 3: Единица на шкале
\begin{frame}
\frametitle{Единица на шкале}

Единица на шкале — это расстояние между соседними делениями.

\vspace{1cm}

\begin{beamercolorbox}[rounded=true, shadow=true, sep=0.5cm]{structure}
    На линейке:\\
    \bfseries 1 единица = 1 см
\end{beamercolorbox}

\vspace{0.5cm}

\begin{beamercolorbox}[rounded=true, shadow=true, sep=0.5cm]{structure}
    На весах:\\
    \bfseries 1 единица = 100 г (или 1 кг)
\end{beamercolorbox}

\end{frame}

% Слайд 4: Как читать шкалу
\begin{frame}
\frametitle{Как читать шкалу?}

\begin{enumerate}
    \item Найти \textcolor{lightblue}{нулевую отметку} (начало отсчёта)
    
    \item Определить \textcolor{lightblue}{цену деления} — размер одного деления
    
    \item Посчитать \textcolor{lightblue}{количество делений} от нуля до нужной точки
    
    \item Умножить количество делений на цену деления
\end{enumerate}

\end{frame}

% Слайд 5: Практический пример 1 (Линейка)
\begin{frame}
\frametitle{Пример: Линейка}

\vspace{0.5cm}

\begin{center}
\begin{beamercolorbox}[rounded=true, shadow=true, sep=0.5cm, wd=\textwidth]{darkgray}
    \centering \texttt{\bfseries 0 | 1 | 2 | 3 | 4 | 5 | 6 | см}
\end{beamercolorbox}
\end{center}

\vspace{1cm}

\begin{beamercolorbox}[rounded=true, shadow=true, sep=0.5cm]{structure}
    \textbf{\textcolor{lightblue}{Цена деления:}} 1 см
    
    \vspace{0.3cm}
    
    \textbf{\textcolor{lightblue}{Длина объекта:}} если конец находится на отметке 5, то длина = 5 см
\end{beamercolorbox}

\end{frame}

% Слайд 6: Практический пример 2 (Спидометр)
\begin{frame}
\frametitle{Пример: Шкала с крупными делениями}

\vspace{0.5cm}

\begin{center}
\begin{beamercolorbox}[rounded=true, shadow=true, sep=0.5cm, wd=\textwidth]{darkgray}
    \centering \texttt{\bfseries 0 — 20 — 40 — 60 — 80 — 100 км/ч}
\end{beamercolorbox}
\end{center}

\vspace{0.8cm}

\begin{itemize}
    \item \textbf{\textcolor{lightblue}{Отмечено:}} 5 делений
    
    \item \textbf{\textcolor{lightblue}{Диапазон:}} от 0 до 100 км/ч
    
    \item \textbf{\textcolor{lightblue}{Цена деления:}} $100 \div 5 = 20$ км/ч
\end{itemize}

\end{frame}

% Слайд 7: Где используются шкалы
\begin{frame}
\frametitle{Где используются шкалы?}

\begin{columns}[T]

\column{0.5\textwidth}
\begin{beamercolorbox}[rounded=true, shadow=true, sep=0.5cm]{structure}
    \textbf{Приборы}
    
    Линейка, весы, часы, градусник
\end{beamercolorbox}

\vspace{0.5cm}

\begin{beamercolorbox}[rounded=true, shadow=true, sep=0.5cm]{structure}
    \textbf{Карты}
    
    Масштабные шкалы расстояний
\end{beamercolorbox}

\column{0.5\textwidth}
\begin{beamercolorbox}[rounded=true, shadow=true, sep=0.5cm]{structure}
    \textbf{Графики}
    
    Оси координат с числовыми шкалами
\end{beamercolorbox}

\vspace{0.5cm}

\begin{beamercolorbox}[rounded=true, shadow=true, sep=0.5cm]{structure}
    \textbf{Таблицы}
    
    Организация данных по шкале
\end{beamercolorbox}

\end{columns}

\end{frame}

% Слайд 8: Задание
\begin{frame}
\frametitle{Задание: Определите показания}

\vspace{0.5cm}

\begin{beamercolorbox}[rounded=true, shadow=true, sep=0.5cm]{darkgray}
    \textbf{На шкале 5 делений от 0 до 50:}
    
    \textcolor{lightblue}{\bfseries Цена одного деления = ?}
\end{beamercolorbox}

\vspace{0.8cm}

\begin{beamercolorbox}[rounded=true, shadow=true, sep=0.5cm]{darkgray}
    \textbf{Стрелка указывает на 3-е деление:}
    
    \textcolor{lightblue}{\bfseries Показание прибора = ?}
\end{beamercolorbox}

\vspace{1cm}

\small \textcolor{mutedgray}{Ответы: цена деления = 10; показание = 30}

\end{frame}

% Слайд 9: Итоги
\begin{frame}
\frametitle{Запомните!}

\vspace{1cm}

\begin{center}

\begin{beamercolorbox}[rounded=true, shadow=true, sep=0.7cm, wd=0.9\textwidth]{darkgray}
    \large \textbf{\textcolor{lightblue}{Шкала}} — система делений для измерения
\end{beamercolorbox}

\vspace{0.5cm}

\begin{beamercolorbox}[rounded=true, shadow=true, sep=0.7cm, wd=0.9\textwidth]{darkgray}
    \large \textbf{\textcolor{lightblue}{Единица шкалы}} — расстояние между делениями
\end{beamercolorbox}

\vspace{0.5cm}

\begin{beamercolorbox}[rounded=true, shadow=true, sep=0.7cm, wd=0.9\textwidth]{darkgray}
    \large \textbf{\textcolor{lightblue}{Цена деления}} = Диапазон $\div$ Количество делений
\end{beamercolorbox}

\end{center}

\end{frame}

% Финальный слайд
\begin{frame}
\begin{center}
    \Huge \textcolor{lightblue}{\bfseries Спасибо за внимание!}
    
    \vspace{2cm}
    
    \Large \textcolor{lightgray}{Вопросы?}
\end{center}
\end{frame}

\end{document}
