\documentclass[10pt]{article}
\usepackage[centertags]{amsmath} %centertags указывает, что ном. многостр. ф. посредине
\usepackage{amsfonts}
\usepackage{amssymb}
\usepackage{amsthm}
\usepackage{amscd}
\usepackage{newlfont}
%\usepackage[T2A]{fontenc}
%\usepackage[cp866]{inputenc}
\usepackage[russian]{babel}
\usepackage[utf8]{inputenc} 
\usepackage{verbatim}
\usepackage{floatflt}
\usepackage{fancybox}

\setcounter{secnumdepth}{4}             %нумерация заголовка вплоть до уровня \paragraph
\setcounter{section}{0}                 %назначение счетчика section
\setcounter{tocdepth}{4}           %в оглавление входит все вплоть до уровня \paragraph
%\numberwithin{equation}{section}        %независимая нумер. внутри кажд. раздела section
%\usepackage{showkeys} % печать меток (удобно при наборе)

\oddsidemargin=-10mm  %вел. отступа от лев. кр. л.=1 дюйм(25.4мм)+(то, что стоит после =)
\textwidth=185mm    %ширина текста
\topmargin=-30mm   %вел. отступа от верхн. кр. л.=1 дюйм(25.4мм)+(то, что стоит после =)
\textheight=258mm %высота текста
\newcommand{\D}{\displaystyle}   %Позволяет более элегантно записать сложные дроби

  \renewcommand{\baselinestretch}{1.1}   %расстояние между строками

% \renewcommand{\contentsname}{Оглавление}
% \renewcommand{\bibname}{Литература}
% \renewcommand{\chaptername}{Лекция}
\def\intyk{\int\limits_{-1/2}^{1/2}}
\def\intyy{\int\limits_{-1/2}^{x_2}}
\def\intxk{\int\limits_{-l/2}^{l/2}}
\def\intxx{\int\limits_{-l/2}^{x_1}}
\def\intigr{\int\limits_{-1/2}^{y_1}}
\def\intt{\int\limits_{0}^{t}}
\def\intV{\int\limits_{V}}
\def\up#1,#2{\stackrel{#1}{#2}}
\def\und#1{\underset{\widetilde{}}{#1}}

\usepackage{hyperref}
\hypersetup{
	colorlinks,
	citecolor=black,
	filecolor=black,
	linkcolor=black,
	urlcolor=black
}

\renewcommand{\rmdefault}{Tempora-TLF}
\renewcommand{\sfdefault}{antt}
\renewcommand{\ttdefault}{antt}

% \usepackage[landscape]{geometry}

\usepackage{enumitem}
\makeatletter
\AddEnumerateCounter{\asbuk}{\russian@alph}{щ}
\makeatother
\usepackage{tikz}
\usepackage{multicol}
\usepackage{datetime}
\usepackage{siunitx}
\pagestyle{empty}

%\title{Распределительное свойство умножения}

\begin{document}


%\tableofcontents
\begin{large}

	\newpage
	
	\subsection*{Вариант 1 для  Каниной Агнии. Дроби \\ \today   \currenttime} 
  \begin{enumerate}
\item Найди число, если его  19 \%  равна 11400. Нарисуй схему.  \\ \nopagebreak \begin{tikzpicture}
 \draw[step=0.7cm ,gray,very thin] (0,0) grid (18,6);
 % Обводка внешнего контура  \draw[black, thick] (0,0) rectangle (18,6);
 \end{tikzpicture}
\item Найди число, если его $\frac{1}{6}$  часть равна 60. Нарисуй схему.  \\ \nopagebreak \begin{tikzpicture}
 \draw[step=0.7cm ,gray,very thin] (0,0) grid (18,6);
 % Обводка внешнего контура  \draw[black, thick] (0,0) rectangle (18,6);
 \end{tikzpicture}
\item Найди число, если его  18 \%  равна 28800. Нарисуй схему.  \\ \nopagebreak \begin{tikzpicture}
 \draw[step=0.7cm ,gray,very thin] (0,0) grid (18,6);
 % Обводка внешнего контура  \draw[black, thick] (0,0) rectangle (18,6);
 \end{tikzpicture}
\item Найди число, если его $\frac{4}{15}$  часть равна 60. Нарисуй схему.  \\ \nopagebreak \begin{tikzpicture}
 \draw[step=0.7cm ,gray,very thin] (0,0) grid (18,6);
 % Обводка внешнего контура  \draw[black, thick] (0,0) rectangle (18,6);
 \end{tikzpicture}
\item Найди число, если его $\frac{9}{14}$  часть равна 126. Нарисуй схему.  \\ \nopagebreak \begin{tikzpicture}
 \draw[step=0.7cm ,gray,very thin] (0,0) grid (18,6);
 % Обводка внешнего контура  \draw[black, thick] (0,0) rectangle (18,6);
 \end{tikzpicture}
\item Найди число, если его $\frac{1}{8}$  часть равна 32. Нарисуй схему.  \\ \nopagebreak \begin{tikzpicture}
 \draw[step=0.7cm ,gray,very thin] (0,0) grid (18,6);
 % Обводка внешнего контура  \draw[black, thick] (0,0) rectangle (18,6);
 \end{tikzpicture}
\end{enumerate}  
\newpage
\subsection*{Вариант 2 для  Спицына Марка. Дроби \\ \today   \currenttime} 
  \begin{enumerate}
\item Найди число, если его $\frac{1}{3}$  часть равна 15. Нарисуй схему.  \\ \nopagebreak \begin{tikzpicture}
 \draw[step=0.7cm ,gray,very thin] (0,0) grid (18,6);
 % Обводка внешнего контура  \draw[black, thick] (0,0) rectangle (18,6);
 \end{tikzpicture}
\item Найди число, если его  7 \%  равна 7700. Нарисуй схему.  \\ \nopagebreak \begin{tikzpicture}
 \draw[step=0.7cm ,gray,very thin] (0,0) grid (18,6);
 % Обводка внешнего контура  \draw[black, thick] (0,0) rectangle (18,6);
 \end{tikzpicture}
\item Найди число, если его $\frac{3}{10}$  часть равна 30. Нарисуй схему.  \\ \nopagebreak \begin{tikzpicture}
 \draw[step=0.7cm ,gray,very thin] (0,0) grid (18,6);
 % Обводка внешнего контура  \draw[black, thick] (0,0) rectangle (18,6);
 \end{tikzpicture}
\item Найди число, если его  15 \%  равна 6000. Нарисуй схему.  \\ \nopagebreak \begin{tikzpicture}
 \draw[step=0.7cm ,gray,very thin] (0,0) grid (18,6);
 % Обводка внешнего контура  \draw[black, thick] (0,0) rectangle (18,6);
 \end{tikzpicture}
\item Найди число, если его $\frac{1}{4}$  часть равна 24. Нарисуй схему.  \\ \nopagebreak \begin{tikzpicture}
 \draw[step=0.7cm ,gray,very thin] (0,0) grid (18,6);
 % Обводка внешнего контура  \draw[black, thick] (0,0) rectangle (18,6);
 \end{tikzpicture}
\item Найди число, если его $\frac{2}{10}$  часть равна 20. Нарисуй схему.  \\ \nopagebreak \begin{tikzpicture}
 \draw[step=0.7cm ,gray,very thin] (0,0) grid (18,6);
 % Обводка внешнего контура  \draw[black, thick] (0,0) rectangle (18,6);
 \end{tikzpicture}
\end{enumerate}  
\newpage
\subsection*{Вариант 3 для  Александровой Ксении Андреевны. Дроби \\ \today   \currenttime} 
  \begin{enumerate}
\item Найди число, если его $\frac{1}{4}$  часть равна 32. Нарисуй схему.  \\ \nopagebreak \begin{tikzpicture}
 \draw[step=0.7cm ,gray,very thin] (0,0) grid (18,6);
 % Обводка внешнего контура  \draw[black, thick] (0,0) rectangle (18,6);
 \end{tikzpicture}
\item Найди число, если его $\frac{1}{4}$  часть равна 36. Нарисуй схему.  \\ \nopagebreak \begin{tikzpicture}
 \draw[step=0.7cm ,gray,very thin] (0,0) grid (18,6);
 % Обводка внешнего контура  \draw[black, thick] (0,0) rectangle (18,6);
 \end{tikzpicture}
\item Найди число, если его $\frac{5}{9}$  часть равна 45. Нарисуй схему.  \\ \nopagebreak \begin{tikzpicture}
 \draw[step=0.7cm ,gray,very thin] (0,0) grid (18,6);
 % Обводка внешнего контура  \draw[black, thick] (0,0) rectangle (18,6);
 \end{tikzpicture}
\item Найди число, если его $\frac{7}{17}$  часть равна 119. Нарисуй схему.  \\ \nopagebreak \begin{tikzpicture}
 \draw[step=0.7cm ,gray,very thin] (0,0) grid (18,6);
 % Обводка внешнего контура  \draw[black, thick] (0,0) rectangle (18,6);
 \end{tikzpicture}
\item Найди число, если его  16 \%  равна 6400. Нарисуй схему.  \\ \nopagebreak \begin{tikzpicture}
 \draw[step=0.7cm ,gray,very thin] (0,0) grid (18,6);
 % Обводка внешнего контура  \draw[black, thick] (0,0) rectangle (18,6);
 \end{tikzpicture}
\item Найди число, если его  7 \%  равна 6300. Нарисуй схему.  \\ \nopagebreak \begin{tikzpicture}
 \draw[step=0.7cm ,gray,very thin] (0,0) grid (18,6);
 % Обводка внешнего контура  \draw[black, thick] (0,0) rectangle (18,6);
 \end{tikzpicture}
\end{enumerate}  
\newpage
\subsection*{Вариант 4 для  Баумгертнер Елизаветы Владиславовны. Дроби \\ \today   \currenttime} 
  \begin{enumerate}
\item Найди число, если его  13 \%  равна 18200. Нарисуй схему.  \\ \nopagebreak \begin{tikzpicture}
 \draw[step=0.7cm ,gray,very thin] (0,0) grid (18,6);
 % Обводка внешнего контура  \draw[black, thick] (0,0) rectangle (18,6);
 \end{tikzpicture}
\item Найди число, если его  12 \%  равна 16800. Нарисуй схему.  \\ \nopagebreak \begin{tikzpicture}
 \draw[step=0.7cm ,gray,very thin] (0,0) grid (18,6);
 % Обводка внешнего контура  \draw[black, thick] (0,0) rectangle (18,6);
 \end{tikzpicture}
\item Найди число, если его $\frac{4}{11}$  часть равна 44. Нарисуй схему.  \\ \nopagebreak \begin{tikzpicture}
 \draw[step=0.7cm ,gray,very thin] (0,0) grid (18,6);
 % Обводка внешнего контура  \draw[black, thick] (0,0) rectangle (18,6);
 \end{tikzpicture}
\item Найди число, если его $\frac{1}{4}$  часть равна 8. Нарисуй схему.  \\ \nopagebreak \begin{tikzpicture}
 \draw[step=0.7cm ,gray,very thin] (0,0) grid (18,6);
 % Обводка внешнего контура  \draw[black, thick] (0,0) rectangle (18,6);
 \end{tikzpicture}
\item Найди число, если его $\frac{7}{12}$  часть равна 84. Нарисуй схему.  \\ \nopagebreak \begin{tikzpicture}
 \draw[step=0.7cm ,gray,very thin] (0,0) grid (18,6);
 % Обводка внешнего контура  \draw[black, thick] (0,0) rectangle (18,6);
 \end{tikzpicture}
\item Найди число, если его $\frac{1}{4}$  часть равна 16. Нарисуй схему.  \\ \nopagebreak \begin{tikzpicture}
 \draw[step=0.7cm ,gray,very thin] (0,0) grid (18,6);
 % Обводка внешнего контура  \draw[black, thick] (0,0) rectangle (18,6);
 \end{tikzpicture}
\end{enumerate}  
\newpage
\subsection*{Вариант 5 для  Гайдадиной Анны Алексеевны. Дроби \\ \today   \currenttime} 
  \begin{enumerate}
\item Найди число, если его $\frac{1}{4}$  часть равна 12. Нарисуй схему.  \\ \nopagebreak \begin{tikzpicture}
 \draw[step=0.7cm ,gray,very thin] (0,0) grid (18,6);
 % Обводка внешнего контура  \draw[black, thick] (0,0) rectangle (18,6);
 \end{tikzpicture}
\item Найди число, если его $\frac{2}{7}$  часть равна 14. Нарисуй схему.  \\ \nopagebreak \begin{tikzpicture}
 \draw[step=0.7cm ,gray,very thin] (0,0) grid (18,6);
 % Обводка внешнего контура  \draw[black, thick] (0,0) rectangle (18,6);
 \end{tikzpicture}
\item Найди число, если его $\frac{10}{18}$  часть равна 180. Нарисуй схему.  \\ \nopagebreak \begin{tikzpicture}
 \draw[step=0.7cm ,gray,very thin] (0,0) grid (18,6);
 % Обводка внешнего контура  \draw[black, thick] (0,0) rectangle (18,6);
 \end{tikzpicture}
\item Найди число, если его  19 \%  равна 19000. Нарисуй схему.  \\ \nopagebreak \begin{tikzpicture}
 \draw[step=0.7cm ,gray,very thin] (0,0) grid (18,6);
 % Обводка внешнего контура  \draw[black, thick] (0,0) rectangle (18,6);
 \end{tikzpicture}
\item Найди число, если его  4 \%  равна 2800. Нарисуй схему.  \\ \nopagebreak \begin{tikzpicture}
 \draw[step=0.7cm ,gray,very thin] (0,0) grid (18,6);
 % Обводка внешнего контура  \draw[black, thick] (0,0) rectangle (18,6);
 \end{tikzpicture}
\item Найди число, если его $\frac{1}{10}$  часть равна 100. Нарисуй схему.  \\ \nopagebreak \begin{tikzpicture}
 \draw[step=0.7cm ,gray,very thin] (0,0) grid (18,6);
 % Обводка внешнего контура  \draw[black, thick] (0,0) rectangle (18,6);
 \end{tikzpicture}
\end{enumerate}  
\newpage
\subsection*{Вариант 6 для  Гринчук Арины. Дроби \\ \today   \currenttime} 
  \begin{enumerate}
\item Найди число, если его $\frac{1}{5}$  часть равна 25. Нарисуй схему.  \\ \nopagebreak \begin{tikzpicture}
 \draw[step=0.7cm ,gray,very thin] (0,0) grid (18,6);
 % Обводка внешнего контура  \draw[black, thick] (0,0) rectangle (18,6);
 \end{tikzpicture}
\item Найди число, если его  19 \%  равна 19000. Нарисуй схему.  \\ \nopagebreak \begin{tikzpicture}
 \draw[step=0.7cm ,gray,very thin] (0,0) grid (18,6);
 % Обводка внешнего контура  \draw[black, thick] (0,0) rectangle (18,6);
 \end{tikzpicture}
\item Найди число, если его $\frac{1}{7}$  часть равна 35. Нарисуй схему.  \\ \nopagebreak \begin{tikzpicture}
 \draw[step=0.7cm ,gray,very thin] (0,0) grid (18,6);
 % Обводка внешнего контура  \draw[black, thick] (0,0) rectangle (18,6);
 \end{tikzpicture}
\item Найди число, если его $\frac{5}{11}$  часть равна 55. Нарисуй схему.  \\ \nopagebreak \begin{tikzpicture}
 \draw[step=0.7cm ,gray,very thin] (0,0) grid (18,6);
 % Обводка внешнего контура  \draw[black, thick] (0,0) rectangle (18,6);
 \end{tikzpicture}
\item Найди число, если его $\frac{9}{20}$  часть равна 180. Нарисуй схему.  \\ \nopagebreak \begin{tikzpicture}
 \draw[step=0.7cm ,gray,very thin] (0,0) grid (18,6);
 % Обводка внешнего контура  \draw[black, thick] (0,0) rectangle (18,6);
 \end{tikzpicture}
\item Найди число, если его  9 \%  равна 3600. Нарисуй схему.  \\ \nopagebreak \begin{tikzpicture}
 \draw[step=0.7cm ,gray,very thin] (0,0) grid (18,6);
 % Обводка внешнего контура  \draw[black, thick] (0,0) rectangle (18,6);
 \end{tikzpicture}
\end{enumerate}  
\newpage
\subsection*{Вариант 7 для  Ермолаевой Софии. Дроби \\ \today   \currenttime} 
  \begin{enumerate}
\item Найди число, если его $\frac{6}{9}$  часть равна 54. Нарисуй схему.  \\ \nopagebreak \begin{tikzpicture}
 \draw[step=0.7cm ,gray,very thin] (0,0) grid (18,6);
 % Обводка внешнего контура  \draw[black, thick] (0,0) rectangle (18,6);
 \end{tikzpicture}
\item Найди число, если его $\frac{8}{17}$  часть равна 136. Нарисуй схему.  \\ \nopagebreak \begin{tikzpicture}
 \draw[step=0.7cm ,gray,very thin] (0,0) grid (18,6);
 % Обводка внешнего контура  \draw[black, thick] (0,0) rectangle (18,6);
 \end{tikzpicture}
\item Найди число, если его  13 \%  равна 24700. Нарисуй схему.  \\ \nopagebreak \begin{tikzpicture}
 \draw[step=0.7cm ,gray,very thin] (0,0) grid (18,6);
 % Обводка внешнего контура  \draw[black, thick] (0,0) rectangle (18,6);
 \end{tikzpicture}
\item Найди число, если его  8 \%  равна 10400. Нарисуй схему.  \\ \nopagebreak \begin{tikzpicture}
 \draw[step=0.7cm ,gray,very thin] (0,0) grid (18,6);
 % Обводка внешнего контура  \draw[black, thick] (0,0) rectangle (18,6);
 \end{tikzpicture}
\item Найди число, если его $\frac{1}{7}$  часть равна 21. Нарисуй схему.  \\ \nopagebreak \begin{tikzpicture}
 \draw[step=0.7cm ,gray,very thin] (0,0) grid (18,6);
 % Обводка внешнего контура  \draw[black, thick] (0,0) rectangle (18,6);
 \end{tikzpicture}
\item Найди число, если его $\frac{1}{7}$  часть равна 63. Нарисуй схему.  \\ \nopagebreak \begin{tikzpicture}
 \draw[step=0.7cm ,gray,very thin] (0,0) grid (18,6);
 % Обводка внешнего контура  \draw[black, thick] (0,0) rectangle (18,6);
 \end{tikzpicture}
\end{enumerate}  
\newpage
\subsection*{Вариант 8 для  Икаева Руслана Алановича. Дроби \\ \today   \currenttime} 
  \begin{enumerate}
\item Найди число, если его $\frac{1}{4}$  часть равна 36. Нарисуй схему.  \\ \nopagebreak \begin{tikzpicture}
 \draw[step=0.7cm ,gray,very thin] (0,0) grid (18,6);
 % Обводка внешнего контура  \draw[black, thick] (0,0) rectangle (18,6);
 \end{tikzpicture}
\item Найди число, если его  20 \%  равна 24000. Нарисуй схему.  \\ \nopagebreak \begin{tikzpicture}
 \draw[step=0.7cm ,gray,very thin] (0,0) grid (18,6);
 % Обводка внешнего контура  \draw[black, thick] (0,0) rectangle (18,6);
 \end{tikzpicture}
\item Найди число, если его $\frac{7}{18}$  часть равна 126. Нарисуй схему.  \\ \nopagebreak \begin{tikzpicture}
 \draw[step=0.7cm ,gray,very thin] (0,0) grid (18,6);
 % Обводка внешнего контура  \draw[black, thick] (0,0) rectangle (18,6);
 \end{tikzpicture}
\item Найди число, если его $\frac{7}{19}$  часть равна 133. Нарисуй схему.  \\ \nopagebreak \begin{tikzpicture}
 \draw[step=0.7cm ,gray,very thin] (0,0) grid (18,6);
 % Обводка внешнего контура  \draw[black, thick] (0,0) rectangle (18,6);
 \end{tikzpicture}
\item Найди число, если его $\frac{1}{3}$  часть равна 15. Нарисуй схему.  \\ \nopagebreak \begin{tikzpicture}
 \draw[step=0.7cm ,gray,very thin] (0,0) grid (18,6);
 % Обводка внешнего контура  \draw[black, thick] (0,0) rectangle (18,6);
 \end{tikzpicture}
\item Найди число, если его  14 \%  равна 4200. Нарисуй схему.  \\ \nopagebreak \begin{tikzpicture}
 \draw[step=0.7cm ,gray,very thin] (0,0) grid (18,6);
 % Обводка внешнего контура  \draw[black, thick] (0,0) rectangle (18,6);
 \end{tikzpicture}
\end{enumerate}  
\newpage
\subsection*{Вариант 9 для  Кострицкой Есении Ивановны. Дроби \\ \today   \currenttime} 
  \begin{enumerate}
\item Найди число, если его  20 \%  равна 28000. Нарисуй схему.  \\ \nopagebreak \begin{tikzpicture}
 \draw[step=0.7cm ,gray,very thin] (0,0) grid (18,6);
 % Обводка внешнего контура  \draw[black, thick] (0,0) rectangle (18,6);
 \end{tikzpicture}
\item Найди число, если его $\frac{1}{7}$  часть равна 21. Нарисуй схему.  \\ \nopagebreak \begin{tikzpicture}
 \draw[step=0.7cm ,gray,very thin] (0,0) grid (18,6);
 % Обводка внешнего контура  \draw[black, thick] (0,0) rectangle (18,6);
 \end{tikzpicture}
\item Найди число, если его  8 \%  равна 2400. Нарисуй схему.  \\ \nopagebreak \begin{tikzpicture}
 \draw[step=0.7cm ,gray,very thin] (0,0) grid (18,6);
 % Обводка внешнего контура  \draw[black, thick] (0,0) rectangle (18,6);
 \end{tikzpicture}
\item Найди число, если его $\frac{9}{12}$  часть равна 108. Нарисуй схему.  \\ \nopagebreak \begin{tikzpicture}
 \draw[step=0.7cm ,gray,very thin] (0,0) grid (18,6);
 % Обводка внешнего контура  \draw[black, thick] (0,0) rectangle (18,6);
 \end{tikzpicture}
\item Найди число, если его $\frac{6}{17}$  часть равна 102. Нарисуй схему.  \\ \nopagebreak \begin{tikzpicture}
 \draw[step=0.7cm ,gray,very thin] (0,0) grid (18,6);
 % Обводка внешнего контура  \draw[black, thick] (0,0) rectangle (18,6);
 \end{tikzpicture}
\item Найди число, если его $\frac{1}{6}$  часть равна 30. Нарисуй схему.  \\ \nopagebreak \begin{tikzpicture}
 \draw[step=0.7cm ,gray,very thin] (0,0) grid (18,6);
 % Обводка внешнего контура  \draw[black, thick] (0,0) rectangle (18,6);
 \end{tikzpicture}
\end{enumerate}  
\newpage
\subsection*{Вариант 10 для  Купцова Даниила Андреевича. Дроби \\ \today   \currenttime} 
  \begin{enumerate}
\item Найди число, если его $\frac{2}{7}$  часть равна 14. Нарисуй схему.  \\ \nopagebreak \begin{tikzpicture}
 \draw[step=0.7cm ,gray,very thin] (0,0) grid (18,6);
 % Обводка внешнего контура  \draw[black, thick] (0,0) rectangle (18,6);
 \end{tikzpicture}
\item Найди число, если его $\frac{1}{5}$  часть равна 45. Нарисуй схему.  \\ \nopagebreak \begin{tikzpicture}
 \draw[step=0.7cm ,gray,very thin] (0,0) grid (18,6);
 % Обводка внешнего контура  \draw[black, thick] (0,0) rectangle (18,6);
 \end{tikzpicture}
\item Найди число, если его $\frac{9}{17}$  часть равна 153. Нарисуй схему.  \\ \nopagebreak \begin{tikzpicture}
 \draw[step=0.7cm ,gray,very thin] (0,0) grid (18,6);
 % Обводка внешнего контура  \draw[black, thick] (0,0) rectangle (18,6);
 \end{tikzpicture}
\item Найди число, если его $\frac{1}{3}$  часть равна 24. Нарисуй схему.  \\ \nopagebreak \begin{tikzpicture}
 \draw[step=0.7cm ,gray,very thin] (0,0) grid (18,6);
 % Обводка внешнего контура  \draw[black, thick] (0,0) rectangle (18,6);
 \end{tikzpicture}
\item Найди число, если его  7 \%  равна 11200. Нарисуй схему.  \\ \nopagebreak \begin{tikzpicture}
 \draw[step=0.7cm ,gray,very thin] (0,0) grid (18,6);
 % Обводка внешнего контура  \draw[black, thick] (0,0) rectangle (18,6);
 \end{tikzpicture}
\item Найди число, если его  11 \%  равна 14300. Нарисуй схему.  \\ \nopagebreak \begin{tikzpicture}
 \draw[step=0.7cm ,gray,very thin] (0,0) grid (18,6);
 % Обводка внешнего контура  \draw[black, thick] (0,0) rectangle (18,6);
 \end{tikzpicture}
\end{enumerate}  
\newpage
\subsection*{Вариант 11 для  Лукашовой Мирославы Евгеньевны. Дроби \\ \today   \currenttime} 
  \begin{enumerate}
\item Найди число, если его  10 \%  равна 5000. Нарисуй схему.  \\ \nopagebreak \begin{tikzpicture}
 \draw[step=0.7cm ,gray,very thin] (0,0) grid (18,6);
 % Обводка внешнего контура  \draw[black, thick] (0,0) rectangle (18,6);
 \end{tikzpicture}
\item Найди число, если его  9 \%  равна 18000. Нарисуй схему.  \\ \nopagebreak \begin{tikzpicture}
 \draw[step=0.7cm ,gray,very thin] (0,0) grid (18,6);
 % Обводка внешнего контура  \draw[black, thick] (0,0) rectangle (18,6);
 \end{tikzpicture}
\item Найди число, если его $\frac{4}{8}$  часть равна 32. Нарисуй схему.  \\ \nopagebreak \begin{tikzpicture}
 \draw[step=0.7cm ,gray,very thin] (0,0) grid (18,6);
 % Обводка внешнего контура  \draw[black, thick] (0,0) rectangle (18,6);
 \end{tikzpicture}
\item Найди число, если его $\frac{7}{18}$  часть равна 126. Нарисуй схему.  \\ \nopagebreak \begin{tikzpicture}
 \draw[step=0.7cm ,gray,very thin] (0,0) grid (18,6);
 % Обводка внешнего контура  \draw[black, thick] (0,0) rectangle (18,6);
 \end{tikzpicture}
\item Найди число, если его $\frac{1}{8}$  часть равна 32. Нарисуй схему.  \\ \nopagebreak \begin{tikzpicture}
 \draw[step=0.7cm ,gray,very thin] (0,0) grid (18,6);
 % Обводка внешнего контура  \draw[black, thick] (0,0) rectangle (18,6);
 \end{tikzpicture}
\item Найди число, если его $\frac{1}{6}$  часть равна 60. Нарисуй схему.  \\ \nopagebreak \begin{tikzpicture}
 \draw[step=0.7cm ,gray,very thin] (0,0) grid (18,6);
 % Обводка внешнего контура  \draw[black, thick] (0,0) rectangle (18,6);
 \end{tikzpicture}
\end{enumerate}  
\newpage
\subsection*{Вариант 12 для  Мухаметшиной Алисы Рифатовны. Дроби \\ \today   \currenttime} 
  \begin{enumerate}
\item Найди число, если его $\frac{8}{19}$  часть равна 152. Нарисуй схему.  \\ \nopagebreak \begin{tikzpicture}
 \draw[step=0.7cm ,gray,very thin] (0,0) grid (18,6);
 % Обводка внешнего контура  \draw[black, thick] (0,0) rectangle (18,6);
 \end{tikzpicture}
\item Найди число, если его $\frac{5}{12}$  часть равна 60. Нарисуй схему.  \\ \nopagebreak \begin{tikzpicture}
 \draw[step=0.7cm ,gray,very thin] (0,0) grid (18,6);
 % Обводка внешнего контура  \draw[black, thick] (0,0) rectangle (18,6);
 \end{tikzpicture}
\item Найди число, если его $\frac{1}{8}$  часть равна 72. Нарисуй схему.  \\ \nopagebreak \begin{tikzpicture}
 \draw[step=0.7cm ,gray,very thin] (0,0) grid (18,6);
 % Обводка внешнего контура  \draw[black, thick] (0,0) rectangle (18,6);
 \end{tikzpicture}
\item Найди число, если его  2 \%  равна 2600. Нарисуй схему.  \\ \nopagebreak \begin{tikzpicture}
 \draw[step=0.7cm ,gray,very thin] (0,0) grid (18,6);
 % Обводка внешнего контура  \draw[black, thick] (0,0) rectangle (18,6);
 \end{tikzpicture}
\item Найди число, если его  9 \%  равна 16200. Нарисуй схему.  \\ \nopagebreak \begin{tikzpicture}
 \draw[step=0.7cm ,gray,very thin] (0,0) grid (18,6);
 % Обводка внешнего контура  \draw[black, thick] (0,0) rectangle (18,6);
 \end{tikzpicture}
\item Найди число, если его $\frac{1}{3}$  часть равна 18. Нарисуй схему.  \\ \nopagebreak \begin{tikzpicture}
 \draw[step=0.7cm ,gray,very thin] (0,0) grid (18,6);
 % Обводка внешнего контура  \draw[black, thick] (0,0) rectangle (18,6);
 \end{tikzpicture}
\end{enumerate}  
\newpage
\subsection*{Вариант 13 для  Нижегородова Даниила Кирилловича. Дроби \\ \today   \currenttime} 
  \begin{enumerate}
\item Найди число, если его  16 \%  равна 35200. Нарисуй схему.  \\ \nopagebreak \begin{tikzpicture}
 \draw[step=0.7cm ,gray,very thin] (0,0) grid (18,6);
 % Обводка внешнего контура  \draw[black, thick] (0,0) rectangle (18,6);
 \end{tikzpicture}
\item Найди число, если его $\frac{7}{17}$  часть равна 119. Нарисуй схему.  \\ \nopagebreak \begin{tikzpicture}
 \draw[step=0.7cm ,gray,very thin] (0,0) grid (18,6);
 % Обводка внешнего контура  \draw[black, thick] (0,0) rectangle (18,6);
 \end{tikzpicture}
\item Найди число, если его $\frac{1}{5}$  часть равна 45. Нарисуй схему.  \\ \nopagebreak \begin{tikzpicture}
 \draw[step=0.7cm ,gray,very thin] (0,0) grid (18,6);
 % Обводка внешнего контура  \draw[black, thick] (0,0) rectangle (18,6);
 \end{tikzpicture}
\item Найди число, если его  9 \%  равна 2700. Нарисуй схему.  \\ \nopagebreak \begin{tikzpicture}
 \draw[step=0.7cm ,gray,very thin] (0,0) grid (18,6);
 % Обводка внешнего контура  \draw[black, thick] (0,0) rectangle (18,6);
 \end{tikzpicture}
\item Найди число, если его $\frac{1}{4}$  часть равна 20. Нарисуй схему.  \\ \nopagebreak \begin{tikzpicture}
 \draw[step=0.7cm ,gray,very thin] (0,0) grid (18,6);
 % Обводка внешнего контура  \draw[black, thick] (0,0) rectangle (18,6);
 \end{tikzpicture}
\item Найди число, если его $\frac{6}{15}$  часть равна 90. Нарисуй схему.  \\ \nopagebreak \begin{tikzpicture}
 \draw[step=0.7cm ,gray,very thin] (0,0) grid (18,6);
 % Обводка внешнего контура  \draw[black, thick] (0,0) rectangle (18,6);
 \end{tikzpicture}
\end{enumerate}  
\newpage
\subsection*{Вариант 14 для  Поландова Максима Ивановича. Дроби \\ \today   \currenttime} 
  \begin{enumerate}
\item Найди число, если его  17 \%  равна 28900. Нарисуй схему.  \\ \nopagebreak \begin{tikzpicture}
 \draw[step=0.7cm ,gray,very thin] (0,0) grid (18,6);
 % Обводка внешнего контура  \draw[black, thick] (0,0) rectangle (18,6);
 \end{tikzpicture}
\item Найди число, если его  17 \%  равна 17000. Нарисуй схему.  \\ \nopagebreak \begin{tikzpicture}
 \draw[step=0.7cm ,gray,very thin] (0,0) grid (18,6);
 % Обводка внешнего контура  \draw[black, thick] (0,0) rectangle (18,6);
 \end{tikzpicture}
\item Найди число, если его $\frac{8}{20}$  часть равна 160. Нарисуй схему.  \\ \nopagebreak \begin{tikzpicture}
 \draw[step=0.7cm ,gray,very thin] (0,0) grid (18,6);
 % Обводка внешнего контура  \draw[black, thick] (0,0) rectangle (18,6);
 \end{tikzpicture}
\item Найди число, если его $\frac{2}{9}$  часть равна 18. Нарисуй схему.  \\ \nopagebreak \begin{tikzpicture}
 \draw[step=0.7cm ,gray,very thin] (0,0) grid (18,6);
 % Обводка внешнего контура  \draw[black, thick] (0,0) rectangle (18,6);
 \end{tikzpicture}
\item Найди число, если его $\frac{1}{3}$  часть равна 21. Нарисуй схему.  \\ \nopagebreak \begin{tikzpicture}
 \draw[step=0.7cm ,gray,very thin] (0,0) grid (18,6);
 % Обводка внешнего контура  \draw[black, thick] (0,0) rectangle (18,6);
 \end{tikzpicture}
\item Найди число, если его $\frac{1}{7}$  часть равна 21. Нарисуй схему.  \\ \nopagebreak \begin{tikzpicture}
 \draw[step=0.7cm ,gray,very thin] (0,0) grid (18,6);
 % Обводка внешнего контура  \draw[black, thick] (0,0) rectangle (18,6);
 \end{tikzpicture}
\end{enumerate}  
\newpage
\subsection*{Вариант 15 для  Поповой Ульяны Дмитриевны. Дроби \\ \today   \currenttime} 
  \begin{enumerate}
\item Найди число, если его $\frac{1}{9}$  часть равна 63. Нарисуй схему.  \\ \nopagebreak \begin{tikzpicture}
 \draw[step=0.7cm ,gray,very thin] (0,0) grid (18,6);
 % Обводка внешнего контура  \draw[black, thick] (0,0) rectangle (18,6);
 \end{tikzpicture}
\item Найди число, если его $\frac{8}{11}$  часть равна 88. Нарисуй схему.  \\ \nopagebreak \begin{tikzpicture}
 \draw[step=0.7cm ,gray,very thin] (0,0) grid (18,6);
 % Обводка внешнего контура  \draw[black, thick] (0,0) rectangle (18,6);
 \end{tikzpicture}
\item Найди число, если его  4 \%  равна 6800. Нарисуй схему.  \\ \nopagebreak \begin{tikzpicture}
 \draw[step=0.7cm ,gray,very thin] (0,0) grid (18,6);
 % Обводка внешнего контура  \draw[black, thick] (0,0) rectangle (18,6);
 \end{tikzpicture}
\item Найди число, если его $\frac{7}{18}$  часть равна 126. Нарисуй схему.  \\ \nopagebreak \begin{tikzpicture}
 \draw[step=0.7cm ,gray,very thin] (0,0) grid (18,6);
 % Обводка внешнего контура  \draw[black, thick] (0,0) rectangle (18,6);
 \end{tikzpicture}
\item Найди число, если его $\frac{1}{9}$  часть равна 45. Нарисуй схему.  \\ \nopagebreak \begin{tikzpicture}
 \draw[step=0.7cm ,gray,very thin] (0,0) grid (18,6);
 % Обводка внешнего контура  \draw[black, thick] (0,0) rectangle (18,6);
 \end{tikzpicture}
\item Найди число, если его  10 \%  равна 21000. Нарисуй схему.  \\ \nopagebreak \begin{tikzpicture}
 \draw[step=0.7cm ,gray,very thin] (0,0) grid (18,6);
 % Обводка внешнего контура  \draw[black, thick] (0,0) rectangle (18,6);
 \end{tikzpicture}
\end{enumerate}  
\newpage
\subsection*{Вариант 16 для  Тильдикова Ярослава Александровича. Дроби \\ \today   \currenttime} 
  \begin{enumerate}
\item Найди число, если его  10 \%  равна 5000. Нарисуй схему.  \\ \nopagebreak \begin{tikzpicture}
 \draw[step=0.7cm ,gray,very thin] (0,0) grid (18,6);
 % Обводка внешнего контура  \draw[black, thick] (0,0) rectangle (18,6);
 \end{tikzpicture}
\item Найди число, если его $\frac{1}{6}$  часть равна 42. Нарисуй схему.  \\ \nopagebreak \begin{tikzpicture}
 \draw[step=0.7cm ,gray,very thin] (0,0) grid (18,6);
 % Обводка внешнего контура  \draw[black, thick] (0,0) rectangle (18,6);
 \end{tikzpicture}
\item Найди число, если его  10 \%  равна 7000. Нарисуй схему.  \\ \nopagebreak \begin{tikzpicture}
 \draw[step=0.7cm ,gray,very thin] (0,0) grid (18,6);
 % Обводка внешнего контура  \draw[black, thick] (0,0) rectangle (18,6);
 \end{tikzpicture}
\item Найди число, если его $\frac{5}{11}$  часть равна 55. Нарисуй схему.  \\ \nopagebreak \begin{tikzpicture}
 \draw[step=0.7cm ,gray,very thin] (0,0) grid (18,6);
 % Обводка внешнего контура  \draw[black, thick] (0,0) rectangle (18,6);
 \end{tikzpicture}
\item Найди число, если его $\frac{1}{11}$  часть равна 88. Нарисуй схему.  \\ \nopagebreak \begin{tikzpicture}
 \draw[step=0.7cm ,gray,very thin] (0,0) grid (18,6);
 % Обводка внешнего контура  \draw[black, thick] (0,0) rectangle (18,6);
 \end{tikzpicture}
\item Найди число, если его $\frac{3}{12}$  часть равна 36. Нарисуй схему.  \\ \nopagebreak \begin{tikzpicture}
 \draw[step=0.7cm ,gray,very thin] (0,0) grid (18,6);
 % Обводка внешнего контура  \draw[black, thick] (0,0) rectangle (18,6);
 \end{tikzpicture}
\end{enumerate}  
\newpage
\subsection*{Вариант 17 для  Топчиева Аарона Рустамовича. Дроби \\ \today   \currenttime} 
  \begin{enumerate}
\item Найди число, если его $\frac{9}{14}$  часть равна 126. Нарисуй схему.  \\ \nopagebreak \begin{tikzpicture}
 \draw[step=0.7cm ,gray,very thin] (0,0) grid (18,6);
 % Обводка внешнего контура  \draw[black, thick] (0,0) rectangle (18,6);
 \end{tikzpicture}
\item Найди число, если его $\frac{6}{11}$  часть равна 66. Нарисуй схему.  \\ \nopagebreak \begin{tikzpicture}
 \draw[step=0.7cm ,gray,very thin] (0,0) grid (18,6);
 % Обводка внешнего контура  \draw[black, thick] (0,0) rectangle (18,6);
 \end{tikzpicture}
\item Найди число, если его  10 \%  равна 4000. Нарисуй схему.  \\ \nopagebreak \begin{tikzpicture}
 \draw[step=0.7cm ,gray,very thin] (0,0) grid (18,6);
 % Обводка внешнего контура  \draw[black, thick] (0,0) rectangle (18,6);
 \end{tikzpicture}
\item Найди число, если его $\frac{1}{11}$  часть равна 88. Нарисуй схему.  \\ \nopagebreak \begin{tikzpicture}
 \draw[step=0.7cm ,gray,very thin] (0,0) grid (18,6);
 % Обводка внешнего контура  \draw[black, thick] (0,0) rectangle (18,6);
 \end{tikzpicture}
\item Найди число, если его  9 \%  равна 13500. Нарисуй схему.  \\ \nopagebreak \begin{tikzpicture}
 \draw[step=0.7cm ,gray,very thin] (0,0) grid (18,6);
 % Обводка внешнего контура  \draw[black, thick] (0,0) rectangle (18,6);
 \end{tikzpicture}
\item Найди число, если его $\frac{1}{3}$  часть равна 24. Нарисуй схему.  \\ \nopagebreak \begin{tikzpicture}
 \draw[step=0.7cm ,gray,very thin] (0,0) grid (18,6);
 % Обводка внешнего контура  \draw[black, thick] (0,0) rectangle (18,6);
 \end{tikzpicture}
\end{enumerate}  
\newpage

	
	\begin{multicols*}{5}
		\subsection*{������. � 1.  \\ \today   \\ \currenttime}   
\begin{enumerate}
\item $2\cdot2\cdot3\cdot 1$
\item $2\cdot2\cdot 5\cdot 13$
   \item ��� $20$ ��� $440$
   \item ��� $36$ ��� $792$
\end{enumerate}
\subsection*{������. � 2.  \\ \today   \\ \currenttime}   
\begin{enumerate}
\item $2\cdot2\cdot2\cdot3\cdot 7$
\item $2\cdot2\cdot2\cdot2\cdot3\cdot3\cdot 5\cdot 11$
   \item ��� $1$ ��� $3960$
   \item ��� $60$ ��� $360$
\end{enumerate}
\subsection*{������. � 3.  \\ \today   \\ \currenttime}   
\begin{enumerate}
\item $2\cdot2\cdot 23$
\item $2\cdot2\cdot2\cdot2\cdot 17$
   \item ��� $660$ ��� $3960$
   \item ��� $1$ ��� $2310$
\end{enumerate}
\subsection*{������. � 4.  \\ \today   \\ \currenttime}   
\begin{enumerate}
\item $2\cdot2\cdot3\cdot3\cdot3\cdot 5\cdot 1$
\item $2\cdot2\cdot3\cdot 17$
   \item ��� $22$ ��� $990$
   \item ��� $1$ ��� $11$
\end{enumerate}
\subsection*{������. � 5.  \\ \today   \\ \currenttime}   
\begin{enumerate}
\item $2\cdot2\cdot3\cdot3\cdot 5\cdot 7$
\item $2\cdot2\cdot3\cdot 5\cdot 13$
   \item ��� $3$ ��� $330$
   \item ��� $9$ ��� $693$
\end{enumerate}
\subsection*{������. � 6.  \\ \today   \\ \currenttime}   
\begin{enumerate}
\item $2\cdot2\cdot3\cdot3\cdot 11$
\item $2\cdot2\cdot2\cdot3\cdot 7$
   \item ��� $11$ ��� $55$
   \item ��� $22$ ��� $396$
\end{enumerate}
\subsection*{������. � 7.  \\ \today   \\ \currenttime}   
\begin{enumerate}
\item $2\cdot2\cdot3\cdot 5\cdot 7$
\item $2\cdot2\cdot3\cdot3\cdot3\cdot 19$
   \item ��� $330$ ��� $1980$
   \item ��� $1$ ��� $2310$
\end{enumerate}
\subsection*{������. � 8.  \\ \today   \\ \currenttime}   
\begin{enumerate}
\item $2\cdot2\cdot3\cdot3\cdot3\cdot 5\cdot 1$
\item $2\cdot2\cdot2\cdot3\cdot3\cdot 5\cdot 13$
   \item ��� $6$ ��� $924$
   \item ��� $24$ ��� $27720$
\end{enumerate}
\subsection*{������. � 9.  \\ \today   \\ \currenttime}   
\begin{enumerate}
\item $2\cdot2\cdot2\cdot 5\cdot 23$
\item $2\cdot2\cdot3\cdot3\cdot 5\cdot 23$
   \item ��� $3$ ��� $660$
   \item ��� $9$ ��� $45$
\end{enumerate}
\subsection*{������. � 10.  \\ \today   \\ \currenttime}   
\begin{enumerate}
\item $2\cdot2\cdot3\cdot3\cdot3\cdot 13$
\item $2\cdot 11$
   \item ��� $3$ ��� $630$
   \item ��� $42$ ��� $420$
\end{enumerate}
\subsection*{������. � 11.  \\ \today   \\ \currenttime}   
\begin{enumerate}
\item $3\cdot3\cdot3\cdot 1$
\item $2\cdot2\cdot2\cdot2\cdot 5\cdot 17$
   \item ��� $7$ ��� $168$
   \item ��� $2$ ��� $9240$
\end{enumerate}
\subsection*{������. � 12.  \\ \today   \\ \currenttime}   
\begin{enumerate}
\item $2\cdot2\cdot2\cdot2\cdot 13$
\item $3\cdot 13$
   \item ��� $30$ ��� $27720$
   \item ��� $99$ ��� $198$
\end{enumerate}
\subsection*{������. � 13.  \\ \today   \\ \currenttime}   
\begin{enumerate}
\item $2\cdot2\cdot 5\cdot 7$
\item $2\cdot2\cdot2\cdot2\cdot 5\cdot 1$
   \item ��� $2$ ��� $27720$
   \item ��� $12$ ��� $1980$
\end{enumerate}

	\end{multicols*}
	
\end{large}

\end{document}
