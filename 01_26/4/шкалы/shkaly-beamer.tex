\documentclass{beamer}

\usetheme{Madrid}
\usecolortheme{default}
\usepackage[utf8]{inputenc}
\usepackage[russian]{babel}
\usepackage{amsmath}
\usepackage{amssymb}
\usepackage{listings}

\title{Шкалы}
\subtitle{Что такое шкала? Использование шкал в математике}
\author{}
\date{2026}

\begin{document}

\frame{\titlepage}

\section{Определение и назначение}

\begin{frame}
\frametitle{Что такое шкала?}

\begin{block}{Определение}
\textbf{Шкала} --- это деления на измерительных инструментах (линейка, термометр, весы и т.д.)
\end{block}

\vspace{0.5cm}

Шкалы помогают нам:
\begin{itemize}
\item Измерять длину и расстояния
\item Определять температуру
\item Узнавать массу предметов
\item Читать показания приборов
\end{itemize}

\end{frame}

\begin{frame}
\frametitle{Примеры шкал в жизни}

\begin{columns}[T]
\column{0.5\textwidth}
\textbf{Линейка}
\begin{itemize}
\item Деления через 1 см
\item Мелкие штрихи (миллиметры)
\item Используется для измерения длины
\end{itemize}

\column{0.5\textwidth}
\textbf{Термометр}
\begin{itemize}
\item Деления через 1\textdegree C
\item От минус к плюсу
\item Показывает температуру
\end{itemize}
\end{columns}

\vspace{0.5cm}

\textbf{Другие примеры:} весы, спидометр, рулетка, измеритель давления

\end{frame}

\section{Цена деления}

\begin{frame}
\frametitle{Цена деления шкалы}

\begin{block}{Определение}
\textbf{Цена деления} --- это стоимость одного деления на шкале (расстояние между соседними числами)
\end{block}

\vspace{0.5cm}

\textbf{Как определить цену деления?}

\begin{enumerate}
\item Найти два соседних числа на шкале
\item Вычислить разность между ними
\item Посчитать количество делений между ними
\item Разделить разность на количество делений
\end{enumerate}

\vspace{0.5cm}

\textbf{Формула:} $\text{Цена деления} = \dfrac{\text{Большее число} - \text{Меньшее число}}{\text{Количество делений}}$

\end{frame}

\begin{frame}
\frametitle{Примеры расчёта цены деления}

\textbf{Пример 1: Линейка}
\begin{itemize}
\item Рассмотрим числа 0 и 10 см
\item Между ними 10 делений (маленьких штрихов)
\item Цена деления = $\dfrac{10 - 0}{10} = 1$ см
\end{itemize}

\vspace{0.5cm}

\textbf{Пример 2: Термометр}
\begin{itemize}
\item Рассмотрим числа 0\textdegree C и 20\textdegree C
\item Между ними 4 деления
\item Цена деления = $\dfrac{20 - 0}{4} = 5$ \textdegree C
\end{itemize}

\vspace{0.5cm}

\textbf{Пример 3: Весы}
\begin{itemize}
\item Рассмотрим числа 0 и 100 г
\item Между ними 5 делений
\item Цена деления = $\dfrac{100 - 0}{5} = 20$ г
\end{itemize}

\end{frame}

\section{Чтение показаний приборов}

\begin{frame}
\frametitle{Как читать показания прибора?}

\textbf{Алгоритм:}

\begin{enumerate}
\item Определить цену деления прибора
\item Найти, какой штрих указывает стрелка (или жидкость)
\item Определить ближайшее меньшее число
\item Посчитать, сколько делений от этого числа до стрелки
\item Умножить количество делений на цену деления
\item Прибавить результат к начальному числу
\end{enumerate}

\vspace{0.5cm}

\textbf{Формула:} 
$$\text{Показание} = \text{Число} + \text{(Количество делений)} \times \text{(Цена деления)}$$

\end{frame}

\begin{frame}
\frametitle{Примеры чтения показаний}

\textbf{Пример 1: Температура на термометре}
\begin{itemize}
\item Цена деления = 2\textdegree C
\item Ближайшее число = 20\textdegree C
\item От 20 до стрелки --- 3 деления
\item Показание = $20 + 3 \times 2 = 20 + 6 = 26$ \textdegree C
\end{itemize}

\vspace{0.5cm}

\textbf{Пример 2: Масса на весах}
\begin{itemize}
\item Цена деления = 100 г
\item Ближайшее число = 2 кг = 2000 г
\item От 2000 до стрелки --- 2 деления
\item Показание = $2000 + 2 \times 100 = 2200$ г = 2 кг 200 г
\end{itemize}

\end{frame}

\section{Практическое применение}

\begin{frame}
\frametitle{Шкалы на измерительных линейках и приборах}

\textbf{Расстояние между числами и смысл деления:}

\vspace{0.3cm}

\begin{tabular}{|l|c|c|c|}
\hline
\textbf{Прибор} & \textbf{Числа} & \textbf{Делений} & \textbf{Цена деления} \\
\hline
Линейка (см) & 0 и 5 & 5 & 1 см \\
\hline
Линейка (мм) & 0 и 10 & 10 & 1 мм \\
\hline
Термометр & 0 и 30 & 3 & 10\textdegree C \\
\hline
Рулетка & 0 и 2 м & 4 & 0,5 м \\
\hline
Весы & 0 и 1000 г & 5 & 200 г \\
\hline
\end{tabular}

\end{frame}

\section{Контрольные вопросы}

\begin{frame}
\frametitle{Проверьте свои знания}

\begin{enumerate}
\item Что называется шкалой?
\item Как называется расстояние между соседними числами на шкале?
\item Напишите формулу для расчёта цены деления
\item Перечислите примеры приборов со шкалами
\item Какие единицы измерения показывают различные шкалы?
\end{enumerate}

\end{frame}

\section{Задание на урок}

\begin{frame}
\frametitle{Задание на урок}

\textbf{Задача 1:} На линейке между числами 0 и 20 см расположено 4 деления. Определите цену деления.

\vspace{0.3cm}

\textbf{Решение:}
Цена деления = $\dfrac{20 - 0}{4} = 5$ см

\vspace{0.5cm}

\textbf{Задача 2:} Термометр показывает 25\textdegree C. Если цена деления 5\textdegree C, и от нулевой отметки до стрелки 5 делений, проверьте, правильно ли показывает прибор.

\vspace{0.3cm}

\textbf{Решение:}
Показание = $0 + 5 \times 5 = 25$ \textdegree C \checkmark

\end{frame}

\begin{frame}
\frametitle{Практические задания}

\begin{enumerate}
\item На рулетке расстояние между числами 0 м и 3 м составляет 6 делений. Найдите цену деления рулетки.

\item На весах между числами 500 г и 1000 г расположено 5 делений. Определите цену деления.

\item На амперметре цена деления 0,2 А. Если стрелка указывает на 5-е деление после числа 1, какой ток показывает прибор?
\end{enumerate}

\end{frame}

\section{Домашнее задание}

\begin{frame}
\frametitle{Домашнее задание}

\textbf{Задача 1:} Рассмотрите градусник в доме. Определите:
\begin{enumerate}
\item Минимальное и максимальное значения на шкале
\item Цену деления
\item Текущую температуру
\end{enumerate}

\vspace{0.3cm}

\textbf{Задача 2:} На шкале между числами 10 и 20 расположено 10 делений. Какова цена деления? Определите показание, если стрелка указывает на 3-е деление после числа 15.

\vspace{0.3cm}

\textbf{Задача 3:} Начертите свою шкалу:
\begin{enumerate}
\item Выберите два числа (например, 0 и 100)
\item Разделите отрезок между ними на 5 равных частей
\item Подпишите все числа
\item Определите цену деления
\end{enumerate}

\vspace{0.3cm}

\textbf{Задача 4:} Найдите в учебнике математики или в интернете примеры шкал (не менее 3). Напишите для каждой:
\begin{itemize}
\item Название и назначение
\item Единицы измерения
\item Цену деления (если указана)
\end{itemize}

\end{frame}

\begin{frame}
\frametitle{Дополнительное задание (для желающих)}

\textbf{Творческое задание:}

Создайте свой собственный прибор со шкалой. Это может быть:
\begin{itemize}
\item Самодельный термометр из пластиковой трубки и воды
\item Весы из линейки и монет
\item Приборная панель для игры
\item Макет спидометра
\end{itemize}

Нарисуйте шкалу и объясните цену деления.

\vspace{0.5cm}

\textbf{Проверка домашнего задания:}

На следующем уроке обсудим задачи 1-4 и посмотрим на творческие работы!

\end{frame}

\begin{frame}
\frametitle{Заключение}

\begin{block}{Ключевые моменты}
\begin{itemize}
\item Шкала --- это деления на измерительных приборах
\item Цена деления вычисляется по формуле
\item Шкалы есть везде в нашей жизни
\item Умение читать показания приборов важно для повседневной жизни
\end{itemize}
\end{block}

\vspace{0.5cm}

\textbf{Спасибо за внимание!}

\end{frame}

\end{document}