\documentclass{beamer}

\usetheme{Madrid}
\usecolortheme{default}
\usepackage[utf8]{inputenc}
\usepackage[russian]{babel}
\usepackage{amsmath}
\usepackage{amssymb}
\usepackage{listings}
\usepackage{tikz}

\title{Числовая прямая}
\subtitle{Координаты, расстояния и движение на числовой прямой}
\author{}
\date{2026}

\begin{document}

\frame{\titlepage}

\section{Определение и назначение}

\begin{frame}
\frametitle{Что такое числовая прямая?}

\begin{block}{Определение}
\textbf{Числовая прямая} --- это прямая линия, на которой отмечены:
\begin{itemize}
\item Точка начала отсчёта (обычно 0)
\item Единица измерения (масштаб)
\item Положительное направление (стрелка вправо)
\end{itemize}
\end{block}

\vspace{0.5cm}

На числовой прямой:
\begin{itemize}
\item Каждому числу соответствует ровно одна точка
\item Каждой точке соответствует ровно одно число
\item Числа увеличиваются слева направо
\item Слева от нуля расположены отрицательные числа
\end{itemize}

\end{frame}

\begin{frame}
\frametitle{Структура числовой прямой}

\textbf{Основные компоненты:}

\begin{itemize}
\item \textbf{Точка начала отсчёта (0)} --- середина прямой
\item \textbf{Единичный отрезок} --- расстояние от 0 до 1
\item \textbf{Положительная часть} --- числа справа от нуля (1, 2, 3, ...)
\item \textbf{Отрицательная часть} --- числа слева от нуля (-1, -2, -3, ...)
\item \textbf{Направление} --- стрелка показывает положительное направление
\end{itemize}

\vspace{0.5cm}

\textbf{Пример:}

\begin{center}
\begin{tikzpicture}[scale=0.8]
\draw[->] (-5,0) -- (5,0);
\foreach \x in {-4,-3,-2,-1,0,1,2,3,4}
  \draw (\x,0.1) -- (\x,-0.1);
\foreach \x in {-4,-3,-2,-1,0,1,2,3,4}
  \node[below] at (\x,-0.3) {\x};
\node[above] at (4.5,0.3) {$\rightarrow$};
\end{tikzpicture}
\end{center}

\end{frame}

\section{Координаты точек}

\begin{frame}
\frametitle{Координаты на числовой прямой}

\begin{block}{Координата}
\textbf{Координата} --- это число, которое указывает положение точки на числовой прямой
\end{block}

\vspace{0.5cm}

\textbf{Как найти координату точки?}

\begin{enumerate}
\item Найти единичный отрезок на прямой
\item Посчитать количество единичных отрезков от нуля до точки
\item Если точка справа от нуля --- число положительное
\item Если точка слева от нуля --- число отрицательное
\end{enumerate}

\vspace{0.5cm}

\textbf{Обозначение:} Точка $A$ с координатой 3 записывается как $A(3)$

\end{frame}

\begin{frame}
\frametitle{Примеры координат}

\textbf{Пример 1: Положительные координаты}

\begin{center}
\begin{tikzpicture}[scale=0.7]
\draw[->] (-2,0) -- (6,0);
\foreach \x in {-1,0,1,2,3,4,5}
  \draw (\x,0.15) -- (\x,-0.15);
\foreach \x in {0,1,2,3,4,5}
  \node[below] at (\x,-0.4) {\x};

\draw[red,fill=red] (2,0) circle (0.15);
\node[above=0.3cm] at (2,0) {$A(2)$};
\draw[blue,fill=blue] (4,0) circle (0.15);
\node[above=0.3cm] at (4,0) {$B(4)$};
\end{tikzpicture}
\end{center}

\textbf{Пример 2: Отрицательные и положительные координаты}

\begin{center}
\begin{tikzpicture}[scale=0.7]
\draw[->] (-5,0) -- (4,0);
\foreach \x in {-4,-3,-2,-1,0,1,2,3}
  \draw (\x,0.15) -- (\x,-0.15);
\foreach \x in {-4,-3,-2,-1,0,1,2,3}
  \node[below] at (\x,-0.4) {\x};

\draw[red,fill=red] (-3,0) circle (0.15);
\node[above=0.3cm] at (-3,0) {$C(-3)$};
\draw[blue,fill=blue] (2,0) circle (0.15);
\node[above=0.3cm] at (2,0) {$D(2)$};
\end{tikzpicture}
\end{center}

\end{frame}

\section{Расстояния на числовой прямой}

\begin{frame}
\frametitle{Расстояние между точками}

\begin{block}{Расстояние}
\textbf{Расстояние} между двумя точками на числовой прямой равно модулю разности их координат
\end{block}

\vspace{0.5cm}

\textbf{Формула:} $d(A,B) = |x_B - x_A|$

\vspace{0.5cm}

\textbf{Правило:} От большего числа вычитаем меньшее число

\vspace{0.3cm}

\textbf{Примеры:}
\begin{itemize}
\item Расстояние от 2 до 5: $|5 - 2| = 3$ единицы
\item Расстояние от -3 до 2: $|2 - (-3)| = |2 + 3| = 5$ единиц
\item Расстояние от -4 до -1: $|-1 - (-4)| = |-1 + 4| = 3$ единицы
\end{itemize}

\end{frame}

\begin{frame}
\frametitle{Примеры расстояний}

\textbf{Пример 1: От 1 до 6}

\begin{center}
\begin{tikzpicture}[scale=0.7]
\draw[->] (-1,0) -- (7,0);
\foreach \x in {0,1,2,3,4,5,6}
  \draw (\x,0.15) -- (\x,-0.15);
\foreach \x in {0,1,2,3,4,5,6}
  \node[below] at (\x,-0.4) {\x};

\draw[red,fill=red] (1,0) circle (0.15);
\draw[blue,fill=blue] (6,0) circle (0.15);
\draw[green,<->] (1,-0.8) -- (6,-0.8);
\node[below] at (3.5,-1.2) {Расстояние = $6 - 1 = 5$};
\end{tikzpicture}
\end{center}

\textbf{Пример 2: От -2 до 3}

\begin{center}
\begin{tikzpicture}[scale=0.7]
\draw[->] (-3,0) -- (5,0);
\foreach \x in {-2,-1,0,1,2,3,4}
  \draw (\x,0.15) -- (\x,-0.15);
\foreach \x in {-2,-1,0,1,2,3,4}
  \node[below] at (\x,-0.4) {\x};

\draw[red,fill=red] (-2,0) circle (0.15);
\draw[blue,fill=blue] (3,0) circle (0.15);
\draw[green,<->] (-2,-0.8) -- (3,-0.8);
\node[below] at (0.5,-1.2) {Расстояние = $3 - (-2) = 5$};
\end{tikzpicture}
\end{center}

\end{frame}

\section{Движение на числовой прямой}

\begin{frame}
\frametitle{Движение на числовой прямой}

\begin{block}{Правило движения}
\begin{itemize}
\item \textbf{Движение вправо (положительное направление)} --- координата увеличивается
\item \textbf{Движение влево (отрицательное направление)} --- координата уменьшается
\end{itemize}
\end{block}

\vspace{0.5cm}

\textbf{Задача:} Точка начинается в позиции $A(2)$. Она движется на 4 единицы вправо. Где она окажется?

\vspace{0.2cm}

\textbf{Решение:} $2 + 4 = 6$

Новая позиция: $B(6)$

\vspace{0.5cm}

\textbf{Задача 2:} Точка начинается в позиции $C(3)$. Она движется на 5 единиц влево. Где она окажется?

\vspace{0.2cm}

\textbf{Решение:} $3 - 5 = -2$

Новая позиция: $D(-2)$

\end{frame}

\begin{frame}
\frametitle{Примеры движения}

\textbf{Пример: Движение вправо на 3 единицы}

\begin{center}
\begin{tikzpicture}[scale=0.7]
\draw[->] (-1,0) -- (7,0);
\foreach \x in {0,1,2,3,4,5,6}
  \draw (\x,0.15) -- (\x,-0.15);
\foreach \x in {0,1,2,3,4,5,6}
  \node[below] at (\x,-0.4) {\x};

\draw[red,fill=red] (1,0) circle (0.15);
\node[above=0.3cm] at (1,0) {Старт: 1};
\draw[blue,fill=blue] (4,0) circle (0.15);
\node[above=0.3cm] at (4,0) {Финиш: 4};
\draw[->,thick,purple] (1.5,0.5) -- (3.5,0.5);
\node[above] at (2.5,0.8) {$+3$};
\end{tikzpicture}
\end{center}

\textbf{Пример: Движение влево на 4 единицы}

\begin{center}
\begin{tikzpicture}[scale=0.7]
\draw[->] (-4,0) -- (4,0);
\foreach \x in {-3,-2,-1,0,1,2,3}
  \draw (\x,0.15) -- (\x,-0.15);
\foreach \x in {-3,-2,-1,0,1,2,3}
  \node[below] at (\x,-0.4) {\x};

\draw[red,fill=red] (2,0) circle (0.15);
\node[above=0.3cm] at (2,0) {Старт: 2};
\draw[blue,fill=blue] (-2,0) circle (0.15);
\node[above=0.3cm] at (-2,0) {Финиш: -2};
\draw[<-,thick,purple] (1.5,0.5) -- (-1.5,0.5);
\node[above] at (0,-0.2) {$-4$};
\end{tikzpicture}
\end{center}

\end{frame}

\section{Сравнение чисел}

\begin{frame}
\frametitle{Сравнение чисел на числовой прямой}

\begin{block}{Правило сравнения}
На числовой прямой число, которое расположено \textbf{левее}, всегда \textbf{меньше} числа, которое расположено \textbf{правее}
\end{block}

\vspace{0.5cm}

\textbf{Примеры:}
\begin{itemize}
\item $-3 < -1$ (так как -3 левее, чем -1)
\item $0 < 2$ (так как 0 левее, чем 2)
\item $-5 < 0$ (так как -5 левее, чем 0)
\item $-4 < 3$ (так как -4 левее, чем 3)
\end{itemize}

\vspace{0.5cm}

\textbf{Визуально на прямой:}

\begin{center}
\begin{tikzpicture}[scale=0.7]
\draw[->] (-5,0) -- (5,0);
\foreach \x in {-4,-3,-2,-1,0,1,2,3,4}
  \draw (\x,0.15) -- (\x,-0.15);
\foreach \x in {-4,-3,-2,-1,0,1,2,3,4}
  \node[below] at (\x,-0.4) {\x};
\node[above] at (-4.5,0.3) {Меньше};
\node[above] at (4.5,0.3) {Больше};
\end{tikzpicture}
\end{center}

\end{frame}

\section{Таблица и резюме}

\begin{frame}
\frametitle{Основные операции на числовой прямой}

\begin{tabular}{|l|c|c|}
\hline
\textbf{Операция} & \textbf{Правило} & \textbf{Пример} \\
\hline
Координата точки & Расстояние от нуля & $A(3)$ или $B(-2)$ \\
\hline
Расстояние & $|x_B - x_A|$ & От 1 до 5: $|5-1|=4$ \\
\hline
Движение вправо & Прибавить & $2 + 3 = 5$ \\
\hline
Движение влево & Вычесть & $5 - 3 = 2$ \\
\hline
Сравнение & Левее = меньше & $-2 < 1$ \\
\hline
\end{tabular}

\end{frame}

\section{Контрольные вопросы}

\begin{frame}
\frametitle{Проверьте свои знания}

\begin{enumerate}
\item Что такое числовая прямая?
\item Как обозначается координата точки?
\item Как найти расстояние между двумя точками?
\item Что произойдёт с координатой при движении вправо?
\item Какое число больше: -5 или -2?
\item Чему равно расстояние от -3 до 2?
\item Где на числовой прямой расположены отрицательные числа?
\end{enumerate}

\end{frame}

\section{Задание на урок}

\begin{frame}
\frametitle{Задание на урок}

\textbf{Задача 1:} На числовой прямой отмечены точки $A(2)$, $B(5)$, $C(-1)$. Найдите расстояния:
\begin{enumerate}
\item От $A$ до $B$: $|5 - 2| = 3$
\item От $C$ до $A$: $|-1 - 2| = 3$ (или $|2 - (-1)| = 3$)
\item От $C$ до $B$: $|-1 - 5| = 6$ (или $|5 - (-1)| = 6$)
\end{enumerate}

\vspace{0.3cm}

\textbf{Задача 2:} Точка движется из позиции $A(3)$. Сначала она движется на 4 единицы вправо, затем на 2 единицы влево. Где она окажется?

\vspace{0.3cm}

\textbf{Решение:}
\begin{align*}
1) \quad 3 + 4 &= 7 \text{ (после движения вправо)}\\
2) \quad 7 - 2 &= 5 \text{ (после движения влево)}
\end{align*}

\textbf{Ответ: } Точка окажется в позиции 5

\end{frame}

\begin{frame}
\frametitle{Практические задания}

\begin{enumerate}
\item Отметьте на числовой прямой точки: $A(-3)$, $B(2)$, $C(0)$, $D(-1)$. Запишите их в порядке возрастания.

\item Найдите координату точки $M$, если она находится на расстоянии 5 единиц от точки $N(2)$ слева.

\item Точка начинает движение из $P(-2)$. Сначала движется на 3 единицы вправо, потом на 4 единицы влево. Найдите финальную позицию.

\item На числовой прямой расстояние между точками $X$ и $Y$ равно 7 единиц. Если $X(1)$, найдите возможные координаты точки $Y$.
\end{enumerate}

\end{frame}

\section{Домашнее задание}

\begin{frame}
\frametitle{Домашнее задание}

\textbf{Задача 1:} Рассмотрите числовую прямую. Ответьте на вопросы:
\begin{enumerate}
\item Где расположены положительные числа?
\item Где расположены отрицательные числа?
\item Какое число находится между -2 и 0?
\end{enumerate}

\vspace{0.3cm}

\textbf{Задача 2:} Начертите числовую прямую и отметьте точки:
$$A(-4), \quad B(-1), \quad C(2), \quad D(5)$$

Найдите расстояния между всеми соседними точками.

\vspace{0.3cm}

\textbf{Задача 3:} Решите задачу движения:
\begin{itemize}
\item Старт: позиция -3
\item Шаг 1: движение на 5 единиц вправо
\item Шаг 2: движение на 2 единицы влево
\item Шаг 3: движение на 4 единицы вправо
\item Где финиш?
\end{itemize}

\end{frame}

\begin{frame}
\frametitle{Дополнительное задание}

\textbf{Задача 4:} Сравните числа (поставьте $<$, $>$ или $=$):
\begin{itemize}
\item $-5 \quad ? \quad -2$
\item $0 \quad ? \quad -1$
\item $-3 \quad ? \quad 3$
\item $-10 \quad ? \quad -5$
\end{itemize}

\vspace{0.3cm}

\textbf{Задача 5:} Творческое задание:

Нарисуйте числовую прямую в масштабе 1 см = 1 единица. Отметьте на ней:
\begin{itemize}
\item Точку $O(0)$ --- начало отсчёта
\item Пять положительных чисел
\item Пять отрицательных чисел
\item Расстояние 8 единиц (отметьте две точки на этом расстоянии)
\end{itemize}

\vspace{0.3cm}

\textbf{Задача 6:} Найдите закономерность:
$$-5, -3, -1, 1, 3, 5, \ldots$$

Напишите следующие три числа в последовательности.

\end{frame}

\begin{frame}
\frametitle{Заключение}

\begin{block}{Ключевые моменты}
\begin{itemize}
\item Числовая прямая --- это способ визуализации чисел
\item Каждому числу соответствует ровно одна точка
\item Расстояние между точками вычисляется по формуле $|x_B - x_A|$
\item Движение вправо --- прибавление, влево --- вычитание
\item На числовой прямой левее = меньше, правее = больше
\end{itemize}
\end{block}

\vspace{0.5cm}

\textbf{Спасибо за внимание!}

\end{frame}

\end{document}