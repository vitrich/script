\documentclass{beamer}

\usetheme{Madrid}
\usecolortheme{default}
\usepackage[utf8]{inputenc}
\usepackage[russian]{babel}
\usepackage{amsmath}
\usepackage{amssymb}
\usepackage{listings}
\usepackage{tikz}

\title{Числовая прямая}
\subtitle{Натуральные числа и ноль на числовой прямой}
\author{}
\date{2026}

\begin{document}

\frame{\titlepage}

\section{Определение и назначение}

\begin{frame}
\frametitle{Что такое числовая прямая?}

\begin{block}{Определение}
\textbf{Числовая прямая} --- это прямая линия, на которой отмечены:
\begin{itemize}
\item Точка начала отсчёта (0)
\item Единица измерения (масштаб)
\item Направление (стрелка вправо)
\end{itemize}
\end{block}

\vspace{0.5cm}

На числовой прямой:
\begin{itemize}
\item Каждому числу соответствует ровно одна точка
\item Каждой точке соответствует ровно одно число
\item Числа увеличиваются слева направо: 0, 1, 2, 3, 4, ...
\item Это натуральные числа и число ноль
\end{itemize}

\end{frame}

\begin{frame}
\frametitle{Структура числовой прямой}

\textbf{Основные компоненты:}

\begin{itemize}
\item \textbf{Точка начала отсчёта (0)} --- левая граница прямой
\item \textbf{Единичный отрезок} --- расстояние от 0 до 1
\item \textbf{Натуральные числа} --- 1, 2, 3, 4, 5, ... (справа от нуля)
\item \textbf{Направление} --- стрелка показывает возрастание чисел
\end{itemize}

\vspace{0.5cm}

\textbf{Пример:}

\begin{center}
\begin{tikzpicture}[scale=0.8]
\draw[->] (-0.5,0) -- (6,0);
\foreach \x in {0,1,2,3,4,5}
  \draw (\x,0.1) -- (\x,-0.1);
\foreach \x in {0,1,2,3,4,5}
  \node[below] at (\x,-0.3) {\x};
\end{tikzpicture}
\end{center}

\textbf{Направление:} 0 $<$ 1 $<$ 2 $<$ 3 $<$ 4 $<$ 5 ...

\end{frame}

\section{Координаты точек}

\begin{frame}
\frametitle{Координаты на числовой прямой}

\begin{block}{Координата}
\textbf{Координата} --- это число, которое указывает положение точки на числовой прямой
\end{block}

\vspace{0.5cm}

\textbf{Как найти координату точки?}

\begin{enumerate}
\item Найти нулевую отметку (начало отсчёта)
\item Найти единичный отрезок на прямой
\item Посчитать количество единичных отрезков от нуля до точки
\item Это число и есть координата точки
\end{enumerate}

\vspace{0.5cm}

\textbf{Обозначение:} Точка $A$ с координатой 3 записывается как $A(3)$

\end{frame}

\begin{frame}
\frametitle{Примеры координат}

\textbf{Пример 1: Найти координаты точек}

\begin{center}
\begin{tikzpicture}[scale=0.8]
\draw[->] (-0.5,0) -- (7,0);
\foreach \x in {0,1,2,3,4,5,6}
  \draw (\x,0.15) -- (\x,-0.15);
\foreach \x in {0,1,2,3,4,5,6}
  \node[below] at (\x,-0.4) {\x};

\draw[red,fill=red] (2,0) circle (0.15);
\node[above=0.4cm] at (2,0) {$A(2)$};
\draw[blue,fill=blue] (4,0) circle (0.15);
\node[above=0.4cm] at (4,0) {$B(4)$};
\draw[green,fill=green] (6,0) circle (0.15);
\node[above=0.4cm] at (6,0) {$C(6)$};
\end{tikzpicture}
\end{center}

\vspace{0.3cm}

\textbf{Ответы:}
\begin{itemize}
\item Точка $A$ находится на расстоянии 2 единицы от нуля $\Rightarrow$ координата 2
\item Точка $B$ находится на расстоянии 4 единицы от нуля $\Rightarrow$ координата 4
\item Точка $C$ находится на расстоянии 6 единиц от нуля $\Rightarrow$ координата 6
\end{itemize}

\end{frame}

\section{Расстояния на числовой прямой}

\begin{frame}
\frametitle{Расстояние между точками}

\begin{block}{Расстояние}
\textbf{Расстояние} между двумя точками на числовой прямой равно разности их координат (из большего числа вычитаем меньшее)
\end{block}

\vspace{0.5cm}

\textbf{Формула:} $d = x_{\text{большее}} - x_{\text{меньшее}}$

\vspace{0.5cm}

\textbf{Примеры:}
\begin{itemize}
\item Расстояние от 2 до 5: $5 - 2 = 3$ единицы
\item Расстояние от 1 до 7: $7 - 1 = 6$ единиц
\item Расстояние от 0 до 4: $4 - 0 = 4$ единицы
\end{itemize}

\end{frame}

\begin{frame}
\frametitle{Примеры расстояний}

\textbf{Пример 1: Расстояние от 1 до 5}

\begin{center}
\begin{tikzpicture}[scale=0.8]
\draw[->] (-0.5,0) -- (6.5,0);
\foreach \x in {0,1,2,3,4,5,6}
  \draw (\x,0.15) -- (\x,-0.15);
\foreach \x in {0,1,2,3,4,5,6}
  \node[below] at (\x,-0.4) {\x};

\draw[red,fill=red] (1,0) circle (0.15);
\draw[blue,fill=blue] (5,0) circle (0.15);
\draw[green,<->] (1,-0.9) -- (5,-0.9);
\node[below] at (3,-1.3) {Расстояние = $5 - 1 = 4$ единицы};
\end{tikzpicture}
\end{center}

\textbf{Пример 2: Расстояние от 2 до 8}

\begin{center}
\begin{tikzpicture}[scale=0.8]
\draw[->] (-0.5,0) -- (9,0);
\foreach \x in {0,1,2,3,4,5,6,7,8}
  \draw (\x,0.15) -- (\x,-0.15);
\foreach \x in {0,1,2,3,4,5,6,7,8}
  \node[below] at (\x,-0.4) {\x};

\draw[red,fill=red] (2,0) circle (0.15);
\draw[blue,fill=blue] (8,0) circle (0.15);
\draw[green,<->] (2,-0.9) -- (8,-0.9);
\node[below] at (5,-1.3) {Расстояние = $8 - 2 = 6$ единиц};
\end{tikzpicture}
\end{center}

\end{frame}

\section{Движение на числовой прямой}

\begin{frame}
\frametitle{Движение вправо на числовой прямой}

\begin{block}{Правило движения вправо}
\textbf{Движение вправо} --- это движение в сторону больших чисел. При этом число увеличивается.
$$\text{Новая позиция} = \text{Старая позиция} + \text{Количество шагов}$$
\end{block}

\vspace{0.3cm}

\textbf{Пример 1:} Точка находится на 2. Движется на 3 единицы вправо.

\vspace{0.2cm}

Решение: $2 + 3 = 5$

\vspace{0.3cm}

\textbf{Пример 2:} Точка находится на 4. Движется на 5 единиц вправо.

\vspace{0.2cm}

Решение: $4 + 5 = 9$

\end{frame}

\begin{frame}
\frametitle{Движение влево на числовой прямой}

\begin{block}{Правило движения влево}
\textbf{Движение влево} --- это движение в сторону меньших чисел. При этом число уменьшается.
$$\text{Новая позиция} = \text{Старая позиция} - \text{Количество шагов}$$
\end{block}

\vspace{0.3cm}

\textbf{Пример 1:} Точка находится на 8. Движется на 3 единицы влево.

\vspace{0.2cm}

Решение: $8 - 3 = 5$

\vspace{0.3cm}

\textbf{Пример 2:} Точка находится на 6. Движется на 4 единицы влево.

\vspace{0.2cm}

Решение: $6 - 4 = 2$

\end{frame}

\begin{frame}
\frametitle{Примеры движения}

\textbf{Пример: Движение вправо на 3 единицы}

\begin{center}
\begin{tikzpicture}[scale=0.8]
\draw[->] (-0.5,0) -- (7.5,0);
\foreach \x in {0,1,2,3,4,5,6,7}
  \draw (\x,0.15) -- (\x,-0.15);
\foreach \x in {0,1,2,3,4,5,6,7}
  \node[below] at (\x,-0.4) {\x};

\draw[red,fill=red] (1,0) circle (0.15);
\node[above=0.4cm] at (1,0) {Старт: 1};
\draw[blue,fill=blue] (4,0) circle (0.15);
\node[above=0.4cm] at (4,0) {Финиш: 4};
\draw[->,thick,purple] (1.5,0.8) -- (3.5,0.8);
\node[above] at (2.5,1.1) {$+3$};
\end{tikzpicture}
\end{center}

\textbf{Пример: Движение влево на 4 единицы}

\begin{center}
\begin{tikzpicture}[scale=0.8]
\draw[->] (-0.5,0) -- (8,0);
\foreach \x in {0,1,2,3,4,5,6,7}
  \draw (\x,0.15) -- (\x,-0.15);
\foreach \x in {0,1,2,3,4,5,6,7}
  \node[below] at (\x,-0.4) {\x};

\draw[red,fill=red] (6,0) circle (0.15);
\node[above=0.4cm] at (6,0) {Старт: 6};
\draw[blue,fill=blue] (2,0) circle (0.15);
\node[above=0.4cm] at (2,0) {Финиш: 2};
\draw[<-,thick,purple] (5.5,0.8) -- (2.5,0.8);
\node[above] at (4,-0.3) {$-4$};
\end{tikzpicture}
\end{center}

\end{frame}

\section{Сравнение чисел}

\begin{frame}
\frametitle{Сравнение чисел на числовой прямой}

\begin{block}{Правило сравнения}
На числовой прямой число, которое находится \textbf{левее}, всегда \textbf{меньше}, чем число, которое находится \textbf{правее}
\end{block}

\vspace{0.5cm}

\textbf{Примеры:}
\begin{itemize}
\item $2 < 5$ (так как 2 левее, чем 5)
\item $0 < 3$ (так как 0 левее, чем 3)
\item $1 < 4$ (так как 1 левее, чем 4)
\item $7 > 4$ (так как 7 правее, чем 4)
\end{itemize}

\vspace{0.5cm}

\textbf{На числовой прямой:}

\begin{center}
\begin{tikzpicture}[scale=0.8]
\draw[->] (-0.5,0) -- (7,0);
\foreach \x in {0,1,2,3,4,5,6}
  \draw (\x,0.15) -- (\x,-0.15);
\foreach \x in {0,1,2,3,4,5,6}
  \node[below] at (\x,-0.4) {\x};
\node[above] at (-0.3,0.5) {Меньше};
\node[above] at (6.3,0.5) {Больше};
\end{tikzpicture}
\end{center}

\end{frame}

\section{Таблица и резюме}

\begin{frame}
\frametitle{Основные операции на числовой прямой}

\begin{tabular}{|l|c|c|}
\hline
\textbf{Операция} & \textbf{Правило} & \textbf{Пример} \\
\hline
Координата & Расстояние от нуля & $A(3)$, $B(5)$ \\
\hline
Расстояние & Вычтем меньшее из большего & От 2 до 6: $6-2=4$ \\
\hline
Движение вправо & Прибавить & $3 + 4 = 7$ \\
\hline
Движение влево & Вычесть & $7 - 3 = 4$ \\
\hline
Сравнение & Левее = меньше & $2 < 5$ \\
\hline
\end{tabular}

\end{frame}

\section{Контрольные вопросы}

\begin{frame}
\frametitle{Проверьте свои знания}

\begin{enumerate}
\item Что такое числовая прямая?
\item Как обозначается координата точки?
\item Какие числа мы рассматриваем на числовой прямой в 4 классе?
\item Как найти расстояние между двумя точками?
\item Что произойдёт с числом при движении вправо?
\item Что произойдёт с числом при движении влево?
\item Какое число больше: 5 или 8?
\end{enumerate}

\end{frame}

\section{Задание на урок}

\begin{frame}
\frametitle{Задание на урок}

\textbf{Задача 1:} На числовой прямой отмечены точки $A(3)$, $B(7)$, $C(5)$. Найдите расстояния:

\begin{enumerate}
\item От $A$ до $B$: $7 - 3 = 4$ единицы
\item От $A$ до $C$: $5 - 3 = 2$ единицы
\item От $C$ до $B$: $7 - 5 = 2$ единицы
\end{enumerate}

\vspace{0.3cm}

\textbf{Задача 2:} Точка движется из позиции 4. Сначала движется на 3 единицы вправо, затем на 2 единицы влево. Где она окажется?

\vspace{0.2cm}

\textbf{Решение:}
\begin{align*}
1) \quad 4 + 3 &= 7 \text{ (после движения вправо)}\\
2) \quad 7 - 2 &= 5 \text{ (после движения влево)}
\end{align*}

\textbf{Ответ: } Точка окажется в позиции 5

\end{frame}

\begin{frame}
\frametitle{Практические задания}

\begin{enumerate}
\item Отметьте на числовой прямой точки: $A(2)$, $B(6)$, $C(4)$, $D(1)$. Запишите их в порядке возрастания.

\item Найдите расстояние между точками $M(3)$ и $N(9)$.

\item Точка начинает движение из позиции 5. Движется на 4 единицы влево. Найдите финальную позицию.

\item На числовой прямой расстояние между точками $X$ и $Y$ равно 6 единиц. Если $X(2)$, найдите координату точки $Y$ (правее точки $X$).
\end{enumerate}

\end{frame}

\section{Домашнее задание}

\begin{frame}
\frametitle{Домашнее задание}

\textbf{Задача 1:} Рассмотрите числовую прямую. Ответьте на вопросы:
\begin{enumerate}
\item С какого числа начинается числовая прямая?
\item В каком направлении увеличиваются числа?
\item Какое число стоит между 5 и 7?
\end{enumerate}

\vspace{0.3cm}

\textbf{Задача 2:} Начертите числовую прямую и отметьте точки:
$$A(1), \quad B(3), \quad C(6), \quad D(8)$$

Найдите расстояния между всеми соседними точками.

\vspace{0.3cm}

\textbf{Задача 3:} Решите задачу движения:
\begin{itemize}
\item Старт: позиция 2
\item Шаг 1: движение на 4 единицы вправо
\item Шаг 2: движение на 3 единицы влево
\item Где финиш?
\end{itemize}

\end{frame}

\begin{frame}
\frametitle{Дополнительное задание}

\textbf{Задача 4:} Сравните числа (поставьте $<$, $>$ или $=$):
\begin{itemize}
\item $3 \quad ? \quad 7$
\item $9 \quad ? \quad 5$
\item $4 \quad ? \quad 4$
\item $2 \quad ? \quad 8$
\end{itemize}

\vspace{0.3cm}

\textbf{Задача 5:} Творческое задание:

Нарисуйте числовую прямую в масштабе 1 см = 1 единица. Отметьте на ней:
\begin{itemize}
\item Ноль --- начало отсчёта
\item Числа от 0 до 10
\item Отметьте расстояние 5 единиц (две точки на этом расстоянии)
\end{itemize}

\vspace{0.3cm}

\textbf{Задача 6:} Найдите закономерность:
$$0, 2, 4, 6, 8, \ldots$$

Напишите следующие три числа в последовательности.

\end{frame}

\begin{frame}
\frametitle{Заключение}

\begin{block}{Ключевые моменты}
\begin{itemize}
\item Числовая прямая --- это способ визуализации натуральных чисел и нуля
\item Каждому числу соответствует ровно одна точка
\item Расстояние между точками: из большего числа вычитаем меньшее
\item Движение вправо --- прибавление, влево --- вычитание
\item На числовой прямой левее = меньше, правее = больше
\end{itemize}
\end{block}

\vspace{0.5cm}

\textbf{Спасибо за внимание!}

\end{frame}

\end{document}