\documentclass[a4paper,12pt]{article}
\usepackage[utf8]{inputenc}
\usepackage[russian]{babel}
\usepackage{amsmath}
\usepackage{geometry}
\usepackage{fancyhdr}

\geometry{margin=1cm}

\pagestyle{fancy}
\fancyhf{}
\rhead{\thepage}
\cfoot{\small Ответы к самостоятельным работам}

\title{\textbf{ОТВЕТЫ К САМОСТОЯТЕЛЬНЫМ РАБОТАМ}}
\subtitle{Нахождение части числа и числа по его части}
\date{}
\author{}

\begin{document}

\maketitle

\section*{САМОСТОЯТЕЛЬНАЯ РАБОТА 1: Нахождение части числа}

\subsection*{Вариант 1}
\begin{enumerate}
  \item \textbf{Ответ:} 80 учебников
  
  Решение: $280 \cdot \frac{2}{7} = \frac{280 \cdot 2}{7} = \frac{560}{7} = 80$
  
  \item \textbf{Ответ:} 40 км осталось
  
  Решение: $24 : \frac{3}{8} = 24 \cdot \frac{8}{3} = 64$ км весь маршрут; $64 - 24 = 40$ км
  
  \item \textbf{Ответ:} 20 человек
  
  Решение: $60 \cdot \frac{1}{3} = 20$
  
  \item \textbf{Ответ:} 18 тетрадей осталось
  
  Решение: $48 \cdot \frac{5}{8} = 30$ тетрадей потратили; $48 - 30 = 18$
  
  \item \textbf{Ответ:} 10 га не засеяно
  
  Решение: $50 \cdot \frac{2}{5} = 20$ га пшеницей; $50 \cdot \frac{1}{5} = 10$ га рожью; $50 - 20 - 10 = 20$ га не засеяно
\end{enumerate}

\subsection*{Вариант 2}
\begin{enumerate}
  \item \textbf{Ответ:} 90 игрушек
  
  Решение: $360 \cdot \frac{1}{4} = 90$
  
  \item \textbf{Ответ:} 72 км всего
  
  Решение: $32 : \frac{4}{9} = 32 \cdot \frac{9}{4} = 72$
  
  \item \textbf{Ответ:} 24 человека
  
  Решение: $40 \cdot \frac{3}{5} = 24$
  
  \item \textbf{Ответ:} 24 коровы
  
  Решение: $72 \cdot \frac{1}{3} = 24$
  
  \item \textbf{Ответ:} 50 кубометров остаётся
  
  Решение: $80 \cdot \frac{3}{8} = 30$ кубометров используют; $80 - 30 = 50$
\end{enumerate}

\newpage

\section*{САМОСТОЯТЕЛЬНАЯ РАБОТА 2: Целое число минус часть}

\subsection*{Вариант 3}
\begin{enumerate}
  \item \textbf{Ответ:} 30 л осталось
  
  Решение: $120 \cdot \frac{3}{4} = 90$ л использовали; $120 - 90 = 30$ л
  
  \item \textbf{Ответ:} 180 кирпичей осталось
  
  Решение: $300 \cdot \frac{2}{5} = 120$ использовали; $300 - 120 = 180$
  
  \item \textbf{Ответ:} 56 груш
  
  Решение: $84 \cdot \frac{1}{3} = 28$ яблонь; $84 - 28 = 56$ груш
  
  \item \textbf{Ответ:} 15 м осталось
  
  Решение: $75 \cdot \frac{4}{5} = 60$ м потратила; $75 - 60 = 15$ м
  
  \item \textbf{Ответ:} 300 рублей осталось
  
  Решение: $1000 \cdot \frac{7}{10} = 700$ р потратили; $1000 - 700 = 300$ р
\end{enumerate}

\subsection*{Вариант 4}
\begin{enumerate}
  \item \textbf{Ответ:} 60 сладостей осталось
  
  Решение: $160 \cdot \frac{5}{8} = 100$ съели; $160 - 100 = 60$
  
  \item \textbf{Ответ:} 30 свободных мест
  
  Решение: $90 \cdot \frac{2}{3} = 60$ занято; $90 - 60 = 30$
  
  \item \textbf{Ответ:} 80 яиц осталось
  
  Решение: $200 \cdot \frac{3}{5} = 120$ продал; $200 - 120 = 80$
  
  \item \textbf{Ответ:} 100 рублей осталось
  
  Решение: $500 \cdot \frac{4}{5} = 400$ р на еду; $500 - 400 = 100$ р
  
  \item \textbf{Ответ:} 36 учебников на полках
  
  Решение: $144 \cdot \frac{3}{4} = 108$ выданы; $144 - 108 = 36$
\end{enumerate}

\newpage

\section*{САМОСТОЯТЕЛЬНАЯ РАБОТА 3: Нахождение числа по его части}

\subsection*{Вариант 5}
\begin{enumerate}
  \item \textbf{Ответ:} 60
  
  Решение: $15 : \frac{1}{4} = 15 \cdot 4 = 60$
  
  \item \textbf{Ответ:} 30 рыб
  
  Решение: $18 : \frac{3}{5} = 18 \cdot \frac{5}{3} = 30$
  
  \item \textbf{Ответ:} 24 участника
  
  Решение: $8 : \frac{1}{3} = 8 \cdot 3 = 24$
  
  \item \textbf{Ответ:} 36 рублей
  
  Решение: $24 : \frac{2}{3} = 24 \cdot \frac{3}{2} = 36$
  
  \item \textbf{Ответ:} 300 учеников
  
  Решение: $180 : \frac{3}{5} = 180 \cdot \frac{5}{3} = 300$
\end{enumerate}

\subsection*{Вариант 6}
\begin{enumerate}
  \item \textbf{Ответ:} 50
  
  Решение: $20 : \frac{2}{5} = 20 \cdot \frac{5}{2} = 50$
  
  \item \textbf{Ответ:} 105 страниц
  
  Решение: $35 : \frac{1}{3} = 35 \cdot 3 = 105$
  
  \item \textbf{Ответ:} 48 покупателей
  
  Решение: $12 : \frac{1}{4} = 12 \cdot 4 = 48$
  
  \item \textbf{Ответ:} 48 деталей (норма)
  
  Решение: $36 : \frac{3}{4} = 36 \cdot \frac{4}{3} = 48$
  
  \item \textbf{Ответ:} 80 страниц
  
  Решение: $50 : \frac{5}{8} = 50 \cdot \frac{8}{5} = 80$
\end{enumerate}

\newpage

\section*{САМОСТОЯТЕЛЬНАЯ РАБОТА 4: Нахождение части как дроби}

\subsection*{Вариант 7}
\begin{enumerate}
  \item \textbf{Ответ:} $\frac{2}{5}$ или 0,4
  
  Решение: $\frac{12}{30} = \frac{2}{5} = 0,4$
  
  \item \textbf{Ответ:} $\frac{1}{5}$ или 0,2
  
  Решение: $\frac{5}{25} = \frac{1}{5} = 0,2$
  
  \item \textbf{Ответ:} $\frac{1}{3}$ или $\approx 0,33$
  
  Решение: $\frac{16}{48} = \frac{1}{3} \approx 0,33$
  
  \item \textbf{Ответ:} $\frac{3}{4}$ или 0,75
  
  Решение: $\frac{15}{20} = \frac{3}{4} = 0,75$
  
  \item \textbf{Ответ:} $\frac{1}{4}$ или 0,25
  
  Решение: $\frac{10}{40} = \frac{1}{4} = 0,25$
\end{enumerate}

\subsection*{Вариант 8}
\begin{enumerate}
  \item \textbf{Ответ:} $\frac{1}{4}$ или 0,25
  
  Решение: $\frac{9}{36} = \frac{1}{4} = 0,25$
  
  \item \textbf{Ответ:} $\frac{1}{3}$ или $\approx 0,33$
  
  Решение: $\frac{8}{24} = \frac{1}{3} \approx 0,33$
  
  \item \textbf{Ответ:} $\frac{1}{5}$ или 0,2
  
  Решение: $\frac{12}{60} = \frac{1}{5} = 0,2$
  
  \item \textbf{Ответ:} $\frac{1}{3}$ или $\approx 0,33$
  
  Решение: $\frac{15}{45} = \frac{1}{3} \approx 0,33$
  
  \item \textbf{Ответ:} $\frac{1}{4}$ или 0,25
  
  Решение: $\frac{8}{32} = \frac{1}{4} = 0,25$
\end{enumerate}

\newpage

\section*{САМОСТОЯТЕЛЬНАЯ РАБОТА 5: Комплексные задачи}

\subsection*{Вариант 9}
\begin{enumerate}
  \item \textbf{Ответ:} 50 апельсинов осталось
  
  Решение: $120 \cdot \frac{1}{3} = 40$ в школу; $120 \cdot \frac{1}{4} = 30$ в детский сад; $120 - 40 - 30 = 50$
  
  \item \textbf{Ответ:} 100 рублей остаётся
  
  Решение: $600 \cdot \frac{1}{2} = 300$ р на еду; $600 \cdot \frac{1}{3} = 200$ р на транспорт; $600 - 300 - 200 = 100$
  
  \item \textbf{Ответ:} 60 книг на полках
  
  Решение: $360 \cdot \frac{1}{2} = 180$ в 4 классы; $360 \cdot \frac{1}{3} = 120$ в 5 классы; $360 - 180 - 120 = 60$
  
  \item \textbf{Ответ:} 200 граммов осталось
  
  Решение: 1 кг = 1000 г; $1000 \cdot \frac{3}{5} = 600$ г на салат; $1000 \cdot \frac{1}{5} = 200$ г на засолку; $1000 - 600 - 200 = 200$ г
  
  \item \textbf{Ответ:} 200 рублей осталось
  
  Решение: $500 \cdot \frac{2}{5} = 200$ р на еду; $500 \cdot \frac{1}{5} = 100$ р на транспорт; $500 - 200 - 100 = 200$
\end{enumerate}

\subsection*{Вариант 10}
\begin{enumerate}
  \item \textbf{Ответ:} 120 метров осталось
  
  Решение: $240 \cdot \frac{1}{3} = 80$ м продали; $240 - 80 = 160$ м остаток; $160 \cdot \frac{1}{4} = 40$ м потом продали; $160 - 40 = 120$ м
  
  \item \textbf{Ответ:} 80 рублей осталось
  
  Решение: $300 \cdot \frac{2}{5} = 120$ р на сладости; $300 \cdot \frac{1}{3} = 100$ р на книги; $300 - 120 - 100 = 80$
  
  \item \textbf{Ответ:} 60 кг осталось
  
  Решение: $200 \cdot \frac{1}{4} = 50$ кг на консервацию; $200 \cdot \frac{2}{5} = 80$ кг на продажу; $200 - 50 - 80 = 70$ кг
  
  Примечание: Проверить условие — может быть опечатка. Если $\frac{2}{5} = 80$ кг, то осталось 70 кг.
  
  \item \textbf{Ответ:} 2 часа свободного времени
  
  Решение: $8 \cdot \frac{1}{4} = 2$ ч на еду; $8 \cdot \frac{1}{2} = 4$ ч программа; $8 - 2 - 4 = 2$ ч свободного
  
  \item \textbf{Ответ:} 250 учеников не ходят
  
  Решение: $500 \cdot \frac{3}{10} = 150$ на футбол; $500 \cdot \frac{1}{5} = 100$ на баскетбол; $500 - 150 - 100 = 250$
\end{enumerate}

\newpage

\section*{Примечания для учителя}

\begin{itemize}
  \item \textbf{Печать:} Каждый лист (Вариант 1 и 2, затем 3 и 4 и т.д.) печатается на одной странице формата А4, затем разрезается пополам для получения формата А5.
  
  \item \textbf{Раздача:} На каждого ученика даётся по 2 варианта (например, Вариант 1 и Вариант 2, или Вариант 3 и Вариант 4).
  
  \item \textbf{Уникальность:} В каждой самостоятельной работе используются разные задачи, не повторяющиеся в других работах и в презентации.
  
  \item \textbf{Чередование типов:} Задачи чередуют разные типы для развития навыков работы с дробями.
  
  \item \textbf{Сложность:} Постепенно увеличивается от Работы 1 к Работе 5.
\end{itemize}

\end{document}
