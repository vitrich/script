\documentclass[a4paper,12pt,landscape]{article}

\usepackage[utf8]{inputenc}
\usepackage[russian]{babel}
\usepackage{amsmath}
\usepackage[landscape,margin=0.7cm]{geometry}
\usepackage{fancyhdr}
\usepackage{multicol}

\pagestyle{fancy}
\fancyhf{}
\rfoot{\small\thepage}
\setlength{\parindent}{0pt}
\setlength{\parskip}{2pt}
\setlength{\columnsep}{1cm}

\title{}
\author{}
\date{}

\begin{document}

% ===== ЛИСТ 1: ВАРИАНТЫ 1 и 2 (ДВЕ ПОЛОВИНЫ А5) =====

\noindent
\begin{minipage}[t]{0.48\textwidth}

\section*{\small Самостоятельная работа. Вариант 1}

\textbf{Тип 1: Нахождение части числа}

\begin{enumerate}
    \item В автобусе 51 место. $\dfrac{2}{3}$ мест заняты. Сколько мест свободно?
    
    \item Купили 5 кг 600 г сахара и израсходовали $\dfrac{1}{2}$ на варенье. Сколько г сахара осталось?
\end{enumerate}

\textbf{Тип 2: Нахождение числа по части}

\begin{enumerate}
    \setcounter{enumi}{2}
    \item Вася загадал число. 12 составляет $\dfrac{3}{8}$ этого числа. Какое число загадал Вася?
    
    \item На приобретение костюма потратили $\dfrac{3}{5}$ денег. Костюм стоил 120 р. Сколько было денег?
\end{enumerate}

\textbf{Тип 3: Нахождение отношения}

\begin{enumerate}
    \setcounter{enumi}{4}
    \item В гараже 30 зелёных машин из 120. Какую часть составляют зелёные? (Выразите дробью и десятичной)
    
    \item Около дома 8 машин, из них 2 серые, остальные синие. Какую часть составляют синие? (Выразите десятичной дробью)
\end{enumerate}

\end{minipage}
\hfill
\begin{minipage}[t]{0.48\textwidth}

\section*{\small Самостоятельная работа. Вариант 2}

\textbf{Тип 1: Нахождение части числа}

\begin{enumerate}
    \item На огороде собрали 42 кг огурцов и $\dfrac{5}{6}$ засолили. Сколько кг осталось свежих?
    
    \item На базу доставили 22 собаки. Из $\dfrac{3}{4}$ составили упряжку. Сколько собак не вошло в упряжку?
\end{enumerate}

\textbf{Тип 2: Нахождение числа по части}

\begin{enumerate}
    \setcounter{enumi}{2}
    \item До обеда выгрузили $\dfrac{7}{10}$ зерна из вагона — это 42 т. Сколько т было всего?
    
    \item Иван не работал 15 дней в апреле. Какую часть апреля он работал? (Выразите десятичной дробью, апрель = 30 дней)
\end{enumerate}

\textbf{Тип 3: Нахождение отношения}

\begin{enumerate}
    \setcounter{enumi}{4}
    \item На уроке 45 минут. На задачу ушло 9 минут. Какая часть урока? (Выразите дробью и десятичной)
    
    \item Продано $\dfrac{2}{3}$ всех билетов из 90. Сколько билетов осталось продать?
\end{enumerate}

\end{minipage}

\newpage

% ===== ЛИСТ 2: ВАРИАНТЫ 3 и 4 =====

\noindent
\begin{minipage}[t]{0.48\textwidth}

\section*{\small Самостоятельная работа. Вариант 3}

\textbf{Тип 1: Нахождение части числа}

\begin{enumerate}
    \item Мастерская получила 700 м шёлка. Из $\dfrac{2}{5}$ сшили халаты, из $\dfrac{3}{5}$ платья. Сколько м осталось?
    
    \item В классе 32 учащихся. $\dfrac{3}{4}$ каталось на лыжах. Сколько не каталось?
\end{enumerate}

\textbf{Тип 2: Нахождение числа по части}

\begin{enumerate}
    \setcounter{enumi}{2}
    \item Прочитали 35 страниц. Осталось прочитать $\dfrac{2}{7}$ книги. Сколько страниц в книге?
    
    \item Длина дороги 20 км. Заасфальтировали $\dfrac{3}{5}$ дороги. Сколько км осталось?
\end{enumerate}

\textbf{Тип 3: Нахождение отношения}

\begin{enumerate}
    \setcounter{enumi}{4}
    \item В кинозале 90 мест. Продано $\dfrac{2}{3}$ билетов. Сколько ещё можно продать?
    
    \item В драмкружке 24 девочки и число мальчиков составляет $\dfrac{3}{8}$ девочек. Сколько всего в кружке?
\end{enumerate}

\end{minipage}
\hfill
\begin{minipage}[t]{0.48\textwidth}

\section*{\small Самостоятельная работа. Вариант 4}

\textbf{Тип 1: Нахождение части числа}

\begin{enumerate}
    \item Папа имел 3500 руб. потратил $\dfrac{5}{7}$ денег. Сколько осталось?
    
    \item В тетради 24 страницы. Записи занимают $\dfrac{5}{8}$ страниц. Сколько чистых страниц?
\end{enumerate}

\textbf{Тип 2: Нахождение числа по части}

\begin{enumerate}
    \setcounter{enumi}{2}
    \item Какова сумма, если 12 руб. составляют $\dfrac{3}{4}$ суммы?
    
    \item За 1 час автобус проходит $\dfrac{1}{6}$ расстояния. За сколько часов пройдёт всё?
\end{enumerate}

\textbf{Тип 3: Нахождение отношения}

\begin{enumerate}
    \setcounter{enumi}{4}
    \item В классе 40 человек. 10 не сдали нормы ГТО. Какая часть сдала? (Выразите десятичной)
    
    \item В баке 18 л бензина — это $\dfrac{1}{4}$ полного бака. Сколько л помещается в бак?
\end{enumerate}

\end{minipage}

\end{document}