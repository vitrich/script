\documentclass[a4paper,12pt]{article}

\usepackage[utf8]{inputenc}
\usepackage[russian]{babel}
\usepackage{amsmath}
\usepackage[margin=2cm]{geometry}
\usepackage{fancyhdr}
\usepackage{xcolor}

\pagestyle{fancy}
\fancyhf{}
\chead{\textbf{ОТВЕТЫ: Доли и части числа}}
\rfoot{\small\thepage}

\title{}
\author{}
\date{}

\setlength{\parindent}{0pt}

\begin{document}

\section*{Вариант 1}

\textbf{Тип 1: Нахождение части числа}

\begin{enumerate}
    \item В автобусе 51 место. $\dfrac{2}{3}$ мест заняты. Сколько мест свободно?
    
    \textcolor{blue}{Решение:}
    \begin{align*}
        1) \quad 51 \times \frac{2}{3} &= 34 \text{ (мест занято)}\\
        2) \quad 51 - 34 &= 17 \text{ (мест свободно)}
    \end{align*}
    \textcolor{red}{\textbf{Ответ: 17 мест}}
    
    \item Купили 5 кг 600 г сахара и израсходовали $\dfrac{1}{2}$ на варенье. Сколько г осталось?
    
    \textcolor{blue}{Решение:}
    \begin{align*}
        1) \quad 5600 \text{ г} \times \frac{1}{2} &= 2800 \text{ г (израсходовано)}\\
        2) \quad 5600 - 2800 &= 2800 \text{ г}
    \end{align*}
    \textcolor{red}{\textbf{Ответ: 2800 г}}
\end{enumerate}

\textbf{Тип 2: Нахождение числа по части}

\begin{enumerate}
    \setcounter{enumi}{2}
    \item Вася загадал число. 12 составляет $\dfrac{3}{8}$ этого числа. Какое число загадал Вася?
    
    \textcolor{blue}{Решение:}
    \[
        x = 12 \div \frac{3}{8} = 12 \times \frac{8}{3} = \frac{96}{3} = 32
    \]
    \textcolor{red}{\textbf{Ответ: 32}}
    
    \item На приобретение костюма потратили $\dfrac{3}{5}$ денег. Костюм стоил 120 р. Сколько было денег?
    
    \textcolor{blue}{Решение:}
    \[
        x = 120 \div \frac{3}{5} = 120 \times \frac{5}{3} = \frac{600}{3} = 200 \text{ р}
    \]
    \textcolor{red}{\textbf{Ответ: 200 рублей}}
\end{enumerate}

\textbf{Тип 3: Нахождение отношения}

\begin{enumerate}
    \setcounter{enumi}{4}
    \item В гараже 30 зелёных машин из 120. Какую часть составляют зелёные? (Выразите дробью и десятичной)
    
    \textcolor{blue}{Решение:}
    \[
        \frac{30}{120} = \frac{1}{4} = 0{,}25
    \]
    \textcolor{red}{\textbf{Ответ: } $\dfrac{1}{4}$ или 0,25}
    
    \item Около дома 8 машин, из них 2 серые, остальные синие. Какую часть составляют синие? (Выразите десятичной)
    
    \textcolor{blue}{Решение:}
    \begin{align*}
        1) \quad 8 - 2 &= 6 \text{ (синих машин)}\\
        2) \quad \frac{6}{8} &= \frac{3}{4} = 0{,}75
    \end{align*}
    \textcolor{red}{\textbf{Ответ: 0,75}}
\end{enumerate}

\vspace{1cm}
\hrule
\vspace{1cm}

\section*{Вариант 2}

\textbf{Тип 1: Нахождение части числа}

\begin{enumerate}
    \item На огороде собрали 42 кг огурцов и $\dfrac{5}{6}$ засолили. Сколько кг осталось свежих?
    
    \textcolor{blue}{Решение:}
    \begin{align*}
        1) \quad 42 \times \frac{5}{6} &= 35 \text{ (кг засолили)}\\
        2) \quad 42 - 35 &= 7 \text{ (кг свежих)}
    \end{align*}
    \textcolor{red}{\textbf{Ответ: 7 кг}}
    
    \item На базу доставили 22 собаки. Из $\dfrac{3}{4}$ составили упряжку. Сколько собак не вошло в упряжку?
    
    \textcolor{blue}{Решение:}
    \begin{align*}
        1) \quad 22 \times \frac{3}{4} &= 16{,}5 \approx 16 \text{ (собак в упряжке)}\\
        2) \quad 22 - 16 &= 6 \text{ (собак не вошло)}
    \end{align*}
    \textcolor{red}{\textbf{Ответ: 6 собак}} (или 5-6 в зависимости от округления)
\end{enumerate}

\textbf{Тип 2: Нахождение числа по части}

\begin{enumerate}
    \setcounter{enumi}{2}
    \item До обеда выгрузили $\dfrac{7}{10}$ зерна из вагона — это 42 т. Сколько т было всего?
    
    \textcolor{blue}{Решение:}
    \[
        x = 42 \div \frac{7}{10} = 42 \times \frac{10}{7} = 60 \text{ т}
    \]
    \textcolor{red}{\textbf{Ответ: 60 тонн}}
    
    \item Иван не работал 15 дней в апреле. Какую часть апреля он работал? (Апрель = 30 дней)
    
    \textcolor{blue}{Решение:}
    \begin{align*}
        1) \quad 30 - 15 &= 15 \text{ (дней работал)}\\
        2) \quad \frac{15}{30} &= \frac{1}{2} = 0{,}5
    \end{align*}
    \textcolor{red}{\textbf{Ответ: } $\dfrac{1}{2}$ или 0,5}
\end{enumerate}

\textbf{Тип 3: Нахождение отношения}

\begin{enumerate}
    \setcounter{enumi}{4}
    \item На уроке 45 минут. На задачу ушло 9 минут. Какая часть урока? (Выразите дробью и десятичной)
    
    \textcolor{blue}{Решение:}
    \[
        \frac{9}{45} = \frac{1}{5} = 0{,}2
    \]
    \textcolor{red}{\textbf{Ответ: } $\dfrac{1}{5}$ или 0,2}
    
    \item Продано $\dfrac{2}{3}$ всех билетов из 90. Сколько билетов осталось продать?
    
    \textcolor{blue}{Решение:}
    \begin{align*}
        1) \quad 90 \times \frac{2}{3} &= 60 \text{ (продано)}\\
        2) \quad 90 - 60 &= 30 \text{ (осталось)}
    \end{align*}
    \textcolor{red}{\textbf{Ответ: 30 билетов}}
\end{enumerate}

\vspace{1cm}
\hrule
\vspace{1cm}

\section*{Вариант 3}

\textbf{Тип 1: Нахождение части числа}

\begin{enumerate}
    \item Мастерская получила 700 м шёлка. Из $\dfrac{2}{5}$ сшили халаты, из $\dfrac{3}{5}$ платья. Сколько м осталось?
    
    \textcolor{blue}{Решение:}
    \begin{align*}
        1) \quad 700 \times \frac{2}{5} &= 280 \text{ (м халатов)}\\
        2) \quad 700 \times \frac{3}{5} &= 420 \text{ (м платьев)}\\
        3) \quad 280 + 420 &= 700 \text{ (м использовано)}\\
        4) \quad 700 - 700 &= 0 \text{ (м осталось)}
    \end{align*}
    \textcolor{red}{\textbf{Ответ: 0 м}}
    
    \item В классе 32 учащихся. $\dfrac{3}{4}$ каталось на лыжах. Сколько не каталось?
    
    \textcolor{blue}{Решение:}
    \begin{align*}
        1) \quad 32 \times \frac{3}{4} &= 24 \text{ (каталось)}\\
        2) \quad 32 - 24 &= 8 \text{ (не каталось)}
    \end{align*}
    \textcolor{red}{\textbf{Ответ: 8 учащихся}}
\end{enumerate}

\textbf{Тип 2: Нахождение числа по части}

\begin{enumerate}
    \setcounter{enumi}{2}
    \item Прочитали 35 страниц. Осталось прочитать $\dfrac{2}{7}$ книги. Сколько страниц в книге?
    
    \textcolor{blue}{Решение:}
    \begin{align*}
        1) \quad \text{Прочитали: } 1 - \frac{2}{7} = \frac{5}{7}\\
        2) \quad x = 35 \div \frac{5}{7} = 35 \times \frac{7}{5} = 49 \text{ страниц}
    \end{align*}
    \textcolor{red}{\textbf{Ответ: 49 страниц}}
    
    \item Длина дороги 20 км. Заасфальтировали $\dfrac{3}{5}$ дороги. Сколько км осталось?
    
    \textcolor{blue}{Решение:}
    \begin{align*}
        1) \quad 20 \times \frac{3}{5} &= 12 \text{ (км заасф.)}\\
        2) \quad 20 - 12 &= 8 \text{ (км осталось)}
    \end{align*}
    \textcolor{red}{\textbf{Ответ: 8 км}}
\end{enumerate}

\textbf{Тип 3: Нахождение отношения}

\begin{enumerate}
    \setcounter{enumi}{4}
    \item В кинозале 90 мест. Продано $\dfrac{2}{3}$ билетов. Сколько ещё можно продать?
    
    \textcolor{blue}{Решение:}
    \begin{align*}
        1) \quad 90 \times \frac{2}{3} &= 60 \text{ (продано)}\\
        2) \quad 90 - 60 &= 30 \text{ (можно продать)}
    \end{align*}
    \textcolor{red}{\textbf{Ответ: 30 билетов}}
    
    \item В драмкружке 24 девочки и число мальчиков составляет $\dfrac{3}{8}$ девочек. Сколько всего в кружке?
    
    \textcolor{blue}{Решение:}
    \begin{align*}
        1) \quad 24 \times \frac{3}{8} &= 9 \text{ (мальчиков)}\\
        2) \quad 24 + 9 &= 33 \text{ (всего)}
    \end{align*}
    \textcolor{red}{\textbf{Ответ: 33 учащихся}}
\end{enumerate}

\vspace{1cm}
\hrule
\vspace{1cm}

\section*{Вариант 4}

\textbf{Тип 1: Нахождение части числа}

\begin{enumerate}
    \item Папа имел 3500 руб. и потратил $\dfrac{5}{7}$ денег. Сколько осталось?
    
    \textcolor{blue}{Решение:}
    \begin{align*}
        1) \quad 3500 \times \frac{5}{7} &= 2500 \text{ (р потратил)}\\
        2) \quad 3500 - 2500 &= 1000 \text{ (р осталось)}
    \end{align*}
    \textcolor{red}{\textbf{Ответ: 1000 рублей}}
    
    \item В тетради 24 страницы. Записи занимают $\dfrac{5}{8}$ страниц. Сколько чистых страниц?
    
    \textcolor{blue}{Решение:}
    \begin{align*}
        1) \quad 24 \times \frac{5}{8} &= 15 \text{ (страниц с записями)}\\
        2) \quad 24 - 15 &= 9 \text{ (чистых страниц)}
    \end{align*}
    \textcolor{red}{\textbf{Ответ: 9 страниц}}
\end{enumerate}

\textbf{Тип 2: Нахождение числа по части}

\begin{enumerate}
    \setcounter{enumi}{2}
    \item Какова сумма, если 12 руб. составляют $\dfrac{3}{4}$ суммы?
    
    \textcolor{blue}{Решение:}
    \[
        x = 12 \div \frac{3}{4} = 12 \times \frac{4}{3} = 16 \text{ р}
    \]
    \textcolor{red}{\textbf{Ответ: 16 рублей}}
    
    \item За 1 час автобус проходит $\dfrac{1}{6}$ расстояния. За сколько часов пройдёт всё?
    
    \textcolor{blue}{Решение:}
    \[
        1 \div \frac{1}{6} = 6 \text{ часов}
    \]
    \textcolor{red}{\textbf{Ответ: за 6 часов}}
\end{enumerate}

\textbf{Тип 3: Нахождение отношения}

\begin{enumerate}
    \setcounter{enumi}{4}
    \item В классе 40 человек. 10 не сдали нормы ГТО. Какая часть сдала? (Выразите десятичной)
    
    \textcolor{blue}{Решение:}
    \begin{align*}
        1) \quad 40 - 10 &= 30 \text{ (сдали)}\\
        2) \quad \frac{30}{40} &= \frac{3}{4} = 0{,}75
    \end{align*}
    \textcolor{red}{\textbf{Ответ: 0,75}}
    
    \item В баке 18 л бензина — это $\dfrac{1}{4}$ полного бака. Сколько л помещается в бак?
    
    \textcolor{blue}{Решение:}
    \[
        x = 18 \div \frac{1}{4} = 18 \times 4 = 72 \text{ л}
    \]
    \textcolor{red}{\textbf{Ответ: 72 литра}}
\end{enumerate}

\end{document}